\chapter[Markov Chains]{Markov Chains}

% Introduction
\chapterinitial{M}{any} real world situations have some level of
unpredictability through randomness: the flip of a coin, the number of orders of
coffee in a shop, the winning numbers of the lottery. However, mathematics can
in fact let us make predictions about what we expect to happen. One tool used to
understand randomness is Markov chains, an area of mathematics sitting at the
intersection of probability and linear algebra.

\section{Problem}\label{sec:problem}

- Barber shop
- 2 chairs and 2 barbers
- Room for 4 people to wait
- Add room for 2 people to wait or add 1 more barber + chair?

\section{Theory}\label{sec:theory}

- System can be described using a Markov chain which is a collection of States,
transitions etc...
- In this instance, it is a type of birth-death process that can be drawn...
- Has a Matrix representation: Q
- We can calculate the expected smallest time for a change to occur.
- There is a mathematical process that discretises this system to give us P and
the corresponding picture...
- In Delta t time what is the probability...
- Now we can think about P pi, ..., P to a high power will give us...  this will
correspond to pi P = pi (this is an eigenvalue problem).
- Return to continuous system.

\section{Solving with Python}\label{sec:solving-with-python}

- numpy array
- function to generate Q
- function to discretise Q
- multiple P by vectors
- raise P to high power
- write function to solve matrix equation and return pfull - eigenvalue approach

- Run for both scenarios


\section{Solving with R}\label{sec:solving-with-R}

- R matrix
- function to generate Q
- function to discretise Q
- multiple P by vectors
- raise P to high power
- write function to solve matrix equation and return pfull - eigenvalue approach

- Run for both scenarios

\section{Research}\label{sec:research}

TBA
