\chapter[Discrete Event Simulation]{Discrete Event Simulation}\label{chp:discrete_event_simulation}

% Introduction
\chapterinitial{C}{omplex} situations further compounded by randomness appear
throughout daily lives. Examples include data flowing through a computer
network, patients being treated at an emergency services, and daily commutes to
work.
Mathematics can be used to understand these complex situations so as to
make predictions which in turn can be used to make improvements. One tool used
to do this, is to let a computer create a dynamic virtual representation of the
scenario in question, a particular approach we are going to cover here is called
Discrete Event Simulation.

\section{Problem}\label{sec:discrete_event_simulation_problem}

A bicycle repair shop would like reconfigure
in order to guarantee that all bicycles processed take a maximum of 30 minutes.
Their current set-up is as follows:

\begin{itemize}
  \item Bicycles arrive randomly at the shop at a rate of 15 per hour.
  \item They wait in line to be seen at an inspection counter, staffed by one
  member of staff who can inspect one bicycle at a time. On average an
  inspection takes around 3 minutes.
  \item Around 20\% of bicycles do not need
  repair after inspection, and they are then ready for collection.
  \item Around 80\% of bicycles go on to be
  repaired after inspection. These then wait in line outside the repair
  workshop, which is staffed by two members of staff who can each repair
  one bicycle at a time. On average
  a repair takes around 6 minutes.
  \item After repair the bicycles are ready for collection.
\end{itemize}

A diagram of the system is shown in Figure~\ref{fig:bike-shop}.

\begin{figure}
    \begin{center}
        \includestandalone[width=\textwidth]{./assets/bike-repair-shop}
    \end{center}
    \caption{
        Diagrammatic representation of the bicycle repair shop as a queuing
        system.
    }
    \label{fig:bike-shop}
\end{figure}

An assumption of infinite capacity at the bicycle repair shop
for waiting bicycles is made.
The shop will hire an extra member of staff in order to meet their target of a
maximum time in the system of 30 minutes. They would like to know if they
should work on the inspection counter or in the repair workshop?

\section{Theory}\label{sec:discrete_event_simulation_theory}

A number of aspects of the bicycle shop above are
probabilistic. For example the
times that bicycles arrive at the shop, the duration of the inspection and
repairs, and whether the bicycle would need to go on to be repaired or not.
When a number of these probabilistic events are linked together
such as the bicycle shop a method to model this situation is
\textit{Discrete Event Simulation}.

Consider one probabilistic event, rolling a six sided die where each side is
equally likely to land.  Therefore the probability of rolling a 1 is
\(\frac{1}{6}\), the probability of rolling a 2 is \(\frac{1}{6}\), and so on.
This means that that if the die is rolled a large number of times,
\(\frac{1}{6}\) of those rolls would be expected to be a 1.

Consider a random process in which the actual values of the probability
of events occurring are not known. Consider rolling a weighted die, in this case a
die in which the probability of obtaining one number is much greater than the
others. How can probability of obtaining a 1 on this die be estimated?

Rolling the weighted die once does not give much information.
However due to a theorem called the law of large numbers,
this die can be rolled a number of times
and find the proportion of those rolls which gave a 1. The more times we roll
the die, the closer this proportion approaches the actual value of the
probability of obtaining a 1.

For a complex system such as the bicycle shop the goal is to estimate the
proportion of bicycles that take longer than 30 minutes to be processed. As it
is a complex system it is difficult to obtain an exact value. So,
like the weighted die, the system will be observed a number of times and
the overall proportions
of bicycles spending longer than 30 minutes in the shop will converge to
the exact value.
Unlike rolling a weighted die, it is costly to observe this shop over a
number of days with identical conditions. In this case it is costly in terms of
time, as the repair shop already exists. However some scenarios, for example the
scenario where the repair shop hires an additional member of staff, do not yet
exist, so observing this would be costly in terms of money also.
It is possible to build a virtual representation of this complex system on a
computer, and observe a virtual day of work much more quickly and with much less
cost, similar to a video game.

In order to do this, the computer needs to be able to generate random outcomes
of each of the smaller events that make up the large complex system. Generating
random events are essentially doing things with random numbers, these need to be
generated.

Computers are deterministic, therefore true randomness is in itself a
challenging mathematical problem.
They can however generate pseudorandom numbers: sequences of numbers that look
like random numbers, but are entirely determined from the
previous numbers in the sequence\footnote{An early
discussion of pseudo random numbers is~\autocite{von195113} where the author
claimed: ``Anyone who considers arithmetical methods of producing random digits
is, of course, in a state of sin.'' A number of different pseudo random
number generators exist, at the time of writing the state of the art is the
Mersenne Twister described in~\autocite{matsumoto1998mersenne}. }.
Most programming languages have methods of
doing this.

In order to simulate an event the law of large numbers can be used.
Let \(X \sim U(0, 1)\), a uniformly pseudorandom variable between 0 and 1.
Let \(D\) be the outcome of a roll of an unbiased die. Then \(D\) can be defined as:

\begin{equation}
D =
  \begin{cases}
    1 & \text{if } 0 \leq X < \frac{1}{6}\\
    2 & \text{if } \frac{1}{6} \leq X < \frac{2}{6}\\
    3 & \text{if } \frac{2}{6} \leq X < \frac{3}{6}\\
    4 & \text{if } \frac{3}{6} \leq X < \frac{4}{6}\\
    5 & \text{if } \frac{4}{6} \leq X < \frac{5}{6}\\
    6 & \text{if } \frac{5}{6} \leq X < 1
  \end{cases}
\end{equation}

The bicycle repair shop is a system of interactions of
random events. This can be thought of as many interactions of random
variables, each generated using pseudorandom numbers.

In this case the fundamental random events that need to be generated are:

\begin{itemize}
  \item the time each bicycle arrives to the repair shop,
  \item the time each bicycle spends at the inspection counter,
  \item whether each bicycle needs to go on to the repair workshop,
  \item the time those bicycles spend being repaired.
\end{itemize}

As the simulation progresses these events will be generated, and will interact
together as described in Section~\ref{sec:discrete_event_simulation_problem}.
The proportion of customers spending longer than 30 minutes in the shop can then
be counted. This proportion itself is a random variable, and so like the
weighted die, running this simulation once does not give much information.
The simulation can be run many times and to give an average proportion.

The process outlined above is a particular implementation of Monte Carlo
simulation called \textit{Discrete Event Simulation}, which is a generic term
for generating pseudorandom numbers and observes the emergent interactions. In
practice there are two main approaches to simulating complex probabilistic
systems such as this one: \textit{event scheduling} and
\textit{process based} simulation.  It so happens that the main
implementations in Python and R use each of these approaches respectively.

% Event scheduling
\subsection{Event Scheduling Approach}

When using the event scheduling approach, the `virtual representation' of the
system is the collection of facilities that the bicycles use, shown in
Figure~\ref{fig:bike-shop}. Then the entities (the bicycles) interact with these
facilities. It is these facilities that determine how the entities behave.

In a simulation that uses an event scheduling approach, a key concept is that
when events occur this causes further events to occur in the future, either
immediately or after a delay.
In the bicycle shop examples of such events include a bicycle joining a queue, a
bicycle beginning service, and a bicycle finishing service. At each event the
event list is updated, and the clock then jumps forward to the next event in
this updated list.

% Process based
\subsection{Process Based Simulation}

When using process based simulation, the `virtual
representation' of the system is the sequence of actions that each entity
(the bicycles) must take, and these sequences of actions might contain delays as
a number of entities seize and release a finite amount of resources. It is the
sequence of these actions that determine how the entities behave.

For the bicycle repair shop an example of one possible sequence of actions would
be:

\textit{arrive
        \(\rightarrow\)
        seize inspection counter
        \(\rightarrow\)
        delay
        \(\rightarrow\)
        release inspection counter
        \(\rightarrow\)
        seize repair shop
        \(\rightarrow\)
        delay
        \(\rightarrow\)
        release repair shop
        \(\rightarrow\)
        leave
        }

The scheduled delays in this sequence of events correspond to the time spend
being inspected and the time spend being repaired. Waiting in line for service
at these facilities are not included in the sequence of events; that is implicit
by the `seize' and `release' actions, as an entity will wait for a free resource
before seizing one.
Therefore in process based simulations, in addition to defining a sequence of
events, resource types and their numbers also need to be defined.


\section{Solving with Python}\label{sec:discrete_event_simulation_solving-with-python}

In this book the Ciw library will be used in order to conduct Discrete Event
Simulation in Python.
Ciw uses the event scheduling approach, which means the system's
facilities are defined, and customers then interact with them.

In this case there are two facilities to define: the inspection desk and the
repair workshop. For each of these the following need to be defined:

\begin{itemize}
  \item the distribution of times between consecutive bicycles arriving,
  \item the distribution of times the bicycles spend in service,
  \item the number of servers available,
  \item the probability of routing to each of the other facilities after
  service.
\end{itemize}

In this case the time between consecutive arrivals will be assumed to follow an
exponential distribution, as will the service time.
These are common assumptions for this sort of queueing system~\autocite{stewart2009probability}.

In Ciw, these are defined as part of a Network object, created using the
\mintinline{python}{ciw.create_network} function. The function below creates a
Network object that defines the system for a given set of parameters bicycle
repair shop:

\begin{pyin}
import ciw


def build_network_object(
    num_inspectors=1,
    num_repairers=2,
):
    """Returns a Network object that defines the repair shop.

    Args:
        num_inspectors: a positive integer (default: 1)
        num_repairers: a positive integer (default: 2)

    Returns:
        a Ciw network object
    """
    arrival_rate = 15
    inspection_rate = 20
    repair_rate = 10
    prob_need_repair = 0.8
    N = ciw.create_network(
        arrival_distributions=[
            ciw.dists.Exponential(arrival_rate),
            ciw.dists.NoArrivals(),
        ],
        service_distributions=[
            ciw.dists.Exponential(inspection_rate),
            ciw.dists.Exponential(repair_rate),
        ],
        number_of_servers=[num_inspectors, num_repairers],
        routing=[[0.0, prob_need_repair], [0.0, 0.0]],
    )
    return N
\end{pyin}

A Network object is used by Ciw to access system parameters. For example one
piece of information it holds is the number of nodes of the system:

\begin{pyin}
N = build_network_object()
print(N.number_of_nodes)
\end{pyin}

which gives:

\begin{pyout}
2
\end{pyout}

Now that the system is defined a Simulation object can be created.
Once this is built the simulation can be run, that is observe it for one
virtual day. The following function does this:

\begin{pyin}
def run_simulation(network, seed=0):
    """Builds a simulation object and runs it for 8 time units.

    Args:
        network: a Ciw network object
        seed: a float (default: 0)

    Returns:
        a Ciw simulation object after a run of the simulation
    """
    max_time = 8
    ciw.seed(seed)
    Q = ciw.Simulation(network)
    Q.simulate_until_max_time(max_time)
    return Q
\end{pyin}

Notice here a random seed is set. This is because there is randomness in running
the simulation, setting a seed ensures reproducible results\footnote{
Pseudo random number generators give a sequence of numbers that obey a series of
properties. A seed is necessary to obtain a starting point for a given sequence.
This has the benefit of ensuring that given sequences can be reproduced.
}.
Notice also that the simulation always begins with an empty system, so the first
bicycle to arrive will never wait for service. Depending on the situation this
may be an unwanted feature, though not in this case as it is reasonable to
assume that the bicycle shop will begin the day with no customers.

To count the number of bicycles that have finished service, and to
count the number of those whose entire journey through the system lasted longer
than 0.5 hours the pandas
library will be used:

\begin{pyin}
import pandas as pd


def get_proportion(Q):
    """Returns the proportion of bicycles spending over a given
    limit at the repair shop.

    Args:
        Q: a Ciw simulation object after a run of the
           simulation

    Returns:
        a real
    """
    limit = 0.5
    inds = Q.nodes[-1].all_individuals
    recs = pd.DataFrame(
        dr for ind in inds for dr in ind.data_records
    )
    recs["total_time"] = recs["exit_date"] - recs["arrival_date"]
    total_times = recs.groupby("id_number")["total_time"].sum()
    return (total_times > limit).mean()
\end{pyin}

Altogether these functions can define the system, run one day of the system, and
then find the proportion of bicycles spending over half an hour in the shop:

\begin{pyin}
N = build_network_object()
Q = run_simulation(N)
p = get_proportion(Q)
print(round(p, 6))
\end{pyin}

This gives:

\begin{pyout}
0.261261
\end{pyout}

meaning 26.13\% of all bicycles spent longer than half an hour at the repair
shop.

However this particular day may have contained a number of extreme events.
For a more accurate proportion this experiment should be repeated a number of
times, and an average proportion taken.
The following function returns an average proportion:

\begin{pyin}
def get_average_proportion(num_inspectors=1, num_repairers=2):
    """Returns the average proportion of bicycles spending over a
    given limit at the repair shop.

    Args:
        num_inspectors: a positive integer (default: 1)
        num_repairers: a positive integer (default: 2)

    Returns:
        a real
    """
    num_trials = 100
    N = build_network_object(
        num_inspectors=num_inspectors,
        num_repairers=num_repairers,
    )
    proportions = []
    for trial in range(num_trials):
        Q = run_simulation(N, seed=trial)
        proportion = get_proportion(Q=Q)
        proportions.append(proportion)
    return sum(proportions) / num_trials
\end{pyin}

This can be used to find the average proportion over 100 trials for the current
system of one inspector and two repair people:

\begin{pyin}
p = get_average_proportion(num_inspectors=1, num_repairers=2)
print(round(p, 6))
\end{pyin}

which gives:

\begin{pyout}
0.159354
\end{pyout}

that is, on average 15.94\% of bicycles will spend longer than 30 minutes at the
repair shop.

Now consider the two possible future scenarios: hiring an
extra member of staff to serve at the inspection desk, or hiring an extra member
of staff at the repair workshop. Which scenario yields a smaller proportion of
bicycles spending over 30 minutes at the shop? First look
the situation where the additional member of staff works at the inspection desk
is considered:

\begin{pyin}
p = get_average_proportion(num_inspectors=2, num_repairers=2)
print(round(p, 6))
\end{pyin}

which gives:

\begin{pyout}
0.038477
\end{pyout}

that is 3.85\% of bicycles.

Now look at the situation where the additional member of staff works at the
repair workshop:

\begin{pyin}
p = get_average_proportion(num_inspectors=1, num_repairers=3)
print(round(p, 6))
\end{pyin}

which gives:

\begin{pyout}
0.103591
\end{pyout}

that is 10.36\% of bicycles.

Therefore an additional member of staff at the inspection desk would be more
beneficial than an additional member of staff at the repair workshop.


\section{Solving with R}\label{sec:discrete_event_simulation_solving-with-R}

In this book we will use the Simmer package in order to conduct discrete event
simulation in R.  Simmer uses the process based approach, which means that each
bicycle's sequence of actions must be defined, and then generate a number of
bicycles with these sequences.

In Simmer these sequences of actions are made up of trajectories. The
diagram in Figure~\ref{fig:processbased_diagram} shows the branched
trajectories than a bicycle would take at the repair shop:

\begin{figure}
\begin{center}
\includestandalone[width=0.75\textwidth]{./assets/process_based}
\caption{Diagrammatic representation of the forked trajectories a bicycle can take}
\label{fig:processbased_diagram}
\end{center}
\end{figure}

The function below defines a simmer object that describes these trajectories:

\begin{Rin}
library(simmer)

#' Returns a simmer trajectory object outlining the bicycles
#' path through the repair shop
#'
#' @return A simmer trajectory object
define_bicycle_trajectories <- function() {
  inspection_rate <- 20
  repair_rate <- 10
  prob_need_repair <- 0.8
  bicycle <-
    trajectory("Inspection") %>%
    seize("Inspector") %>%
    timeout(function() {
      rexp(1, inspection_rate)
    }) %>%
    release("Inspector") %>%
    branch(
      function() (runif(1) < prob_need_repair),
      continue = c(F),
      trajectory("Repair") %>%
        seize("Repairer") %>%
        timeout(function() {
          rexp(1, repair_rate)
        }) %>%
        release("Repairer"),
      trajectory("Out")
    )
  return(bicycle)
}
\end{Rin}

These trajectories are not very useful alone, we are yet to define the resources
used, or a way to generate bicycles with these trajectories. This is done in the
function below, which begins by defining a \mintinline{R}{repair_shop} with one
resource labelled ``Inspector'', and two resources labelled ``Repairer''.
Once this is built the simulation can be run, that is observe it for one
virtual day. The following function does all this:

\begin{Rin}
#' Runs one trial of the simulation.
#'
#' @param bicycle a simmer trajectory object
#' @param num_inspectors positive integer (default: 1)
#' @param num_repairers positive integer (default: 2)
#' @param seed a float (default: 0)
#'
#' @return A simmer simulation object after one run of
#'         the simulation
run_simulation <- function(bicycle,
                           num_inspectors = 1,
                           num_repairers = 2,
                           seed = 0) {
  arrival_rate <- 15
  max_time <- 8
  repair_shop <-
    simmer("Repair Shop") %>%
    add_resource("Inspector", num_inspectors) %>%
    add_resource("Repairer", num_repairers) %>%
    add_generator(
      "Bicycle", bicycle, function() {
        rexp(1, arrival_rate)
      }
    )
  set.seed(seed)
  repair_shop %>% run(until = max_time)
  return(repair_shop)
}
\end{Rin}

Notice here a random seed is set. This is because there are elements of
randomness when running the simulation, setting a seed ensures reproducible
results\footnote{
Pseudo random number generators give a sequence of numbers that obey a series of
properties. A seed is necessary to obtain a starting point for a given sequence.
This has the benefit of ensuring that given sequences can be reproduced.
}.
Notice also that the simulation always begins with an empty system, so the first
bicycle to arrive will never wait for service. Depending on the situation this
may be an unwanted feature, though not in this case as it is reasonable to
assume that the bicycle shop will begin the day with no customers.

To count the number of bicycles that have finished service, and to
count the number of those whose entire journey through the system lasted longer
than 0.5 hours, Simmer's \mintinline{R}{get_mon_arrivals()} function gives a
data frame that can be manipulated:

\begin{Rin}
#' Returns the proportion of bicycles spending over 30
#' minutes in the repair shop
#'
#' @param repair_shop a simmer simulation object
#'
#' @return a float between 0 and 1
get_proportion <- function(repair_shop) {
  limit <- 0.5
  recs <- repair_shop %>% get_mon_arrivals()
  total_times <- recs$end_time - recs$start_time
  return(mean(total_times > limit))
}
\end{Rin}

Altogether these functions can define the system, run one day of the system, and
then find the proportion of bicycles spending over half an hour in the shop:

\begin{Rin}
bicycle <- define_bicycle_trajectories()
repair_shop <- run_simulation(bicycle = bicycle)
print(get_proportion(repair_shop = repair_shop))
\end{Rin}

This piece of code gives

\begin{Rout}
[1] 0.1343284
\end{Rout}

meaning 13.43\% of all bicycles spent longer than half an hour at the repair
shop.

However this particular day may have contained a number of extreme events.  For
a more accurate proportion this experiment should be repeated a number of times,
and an average proportion taken.  In order to do so, the following is a function
that performs the above experiment over a number of trials, then finds an
average proportion:

\begin{Rin}
#' Returns the average proportion of bicycles spending over
#' a given limit at the repair shop.
#'
#' @param num_inspectors positive integer (default: 1)
#' @param num_repairers positive integer (default: 2)

#' @return a float between 0 and 1
get_average_proportion <- function(num_inspectors = 1,
                                   num_repairers = 2) {
  num_trials <- 100
  bicycle <- define_bicycle_trajectories()
  proportions <- c()
  for (trial in 1:num_trials) {
    repair_shop <- run_simulation(
      bicycle = bicycle,
      num_inspectors = num_inspectors,
      num_repairers = num_repairers,
      seed = trial
    )
    proportion <- get_proportion(
      repair_shop = repair_shop
    )
    proportions[trial] <- proportion
  }
  return(mean(proportions))
}
\end{Rin}

This can be used to find the average proportion over 100 trials:

\begin{Rin}
print(
  get_average_proportion(
    num_inspectors = 1,
    num_repairers = 2)
)
\end{Rin}

which gives:

\begin{Rout}
[1] 0.1635779
\end{Rout}

that is, on average 16.36\% of bicycles will spend longer than 30 minutes at the
repair shop.

Now consider the two possible future scenarios: hiring an
extra member of staff to serve at the inspection desk, or hiring an extra member
of staff at the repair workshop. Which scenario yields a smaller proportion of
bicycles spending over 30 minutes at the shop? First consider the
the situation where the additional member of staff works at the inspection desk:

\begin{Rin}
print(
  get_average_proportion(
    num_inspectors = 2,
    num_repairers = 2
  )
)
\end{Rin}

which gives:

\begin{Rout}
[1] 0.04221602
\end{Rout}

that is 4.22\% of bicycles.

Now look at the situation where the additional member of staff works at the
repair workshop:

\begin{Rin}
print(
  get_average_proportion(
    num_inspectors = 1,
    num_repairers = 3
  )
)
\end{Rin}

which gives:

\begin{Rout}
[1] 0.1224761
\end{Rout}

that is 12.25\% of bicycles.

Therefore an additional member of staff at the inspection desk would be more
beneficial than an additional member of staff at the repair workshop.

\section{Wider context}\label{sec:discrete_event_simulation_wider_context}

The concepts shown in this chapter cover some theoretical aspects of discrete
event simulation. There are a number of further topics that can be vital to
creating valid models of real life scenarios. These include time dependent rates
and rostering servers. An overview of the theory of discrete event simulation is
given in~\parencite{robinson2004simulation}.

One particular use of discrete event simulation is as part of a wider
optimisation exercise. For example, a systematic search for an optimal service
configuration can use a discrete event simulation model as a replacement for a
mathematical objective function. Another approach is to integrate an
optimisation procedure\footnote{For more information on optimisation see
Chapters~\ref{chp:linear_programming} and~\ref{chp:heuristics}.} within a
discrete event simulation model so as to iteratively simulate optimal
configurations. This is done in~\parencite{osorio2017simulation} to be able to
bring together strategic configuration and overall flow in the blood supply
chain. A general review and taxonomy of different uses of discrete event
simulation with optimisation techniques is given
in~\parencite{figueira2014hybrid}.

One domain where simulation is popularly used is in modelling healthcare systems.
A general overview is given in~\parencite{brailsford2009analysis} where uses
include resource utilisation, human behaviour, and workforce management.

In order to be able to fully capture all the relevant details of the system to
be modelled, an extension of discrete event simulation is to combine the
methodology with systems dynamics (Chapter~\ref{chp:system_dynamics}) in order
to model continuous aspects of the system and/or agent based modelling
(Chapter~\ref{chp:agent_based_simulation}) in order to observe emergent or
learned behaviours. This is known as hybrid simulation, and an overview is given
in~\parencite{brailsford2019hybrid}. There are a number of ways of combining
these methodologies, from comparison to full integration.
