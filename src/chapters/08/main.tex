\chapter[Linear programming]{Linear Programming}

% Introduction
\chapterinitial{F}{inding} the best configuration of some system can be
challenging, especially when there is a seemingly endless amount of possible
solutions. Optimisation techniques are a way to mathematically derive solutions
that maximise or minimise some objective function, subject to a number of
feasibility constraints. When all components of the problem can be written in a
linear way, then linear programming is one technique that can be used to find
the solution.

\section{Problem}\label{sec:problem}
A university runs 26 modules over four subjects: Art, Biology, Chemistry, and
Dutch. Each subject runs core modules and optional modules.
Table~\ref{tab:modules} gives the module numbers for each of these.

\begin{table}
\begin{center}
\begin{tabular}{llll}
\toprule
\textbf{Art Core} & \textbf{Biology Core} & \textbf{Chemistry Core} & \textbf{Dutch Core}\\
\midrule
M00 & M08 & M16 & M22 \\
M01 & M09 & M17 & M23 \\
M02 & M10 & M18 &     \\
    & M11 & M19 &     \\
    & M12 &     &     \\
\midrule
\textbf{Art Optional} & \textbf{Biology Optional} & \textbf{Chemistry Optional} & \textbf{Dutch Optional}\\
\midrule
M03 & M13 & M20 & M24 \\
M04 & M14 & M21 & M25 \\
M05 & M15 &     &     \\
M06 &     &     &     \\
M07 &     &     &     \\
\bottomrule
\end{tabular}
\end{center}
\caption{List of modules on offer at the university.}
\label{tab:modules}
\end{table}

The university is required to schedule examinations for each of these modules.
The university would like the exams to be scheduled using the least amount of
time slots possible. However not all modules can be scheduled at the same time
as they share some students:

\begin{itemize}
  \item All art modules share students,
  \item All biology modules share students,
  \item All chemistry modules share students,
  \item All Dutch modules share students,
  \item Art students have the option of taking the core Dutch modules, and so
  all art modules may share students with core Dutch modules,
  \item Biology students have the option of taking optional modules from
  chemistry, so all biology modules may share students with optional chemistry
  modules,
  \item Chemistry students have the option of taking optional modules from
  biology, so all chemistry modules may share students with optional biology
  modules,
  \item Biology students have the option of taking core art modules, and so all
  biology modules may share students with core art modules.
\end{itemize}

What is the least number of exam time slots required to hold all 26 exams with
no clashes?


\section{Theory}\label{sec:theory}
Linear programming is a method that solves an optimisation problem of $n$
variables by defining all constraints as planes in $n$-dimensional space. These
planes combine to create a convex region where all feasible solutions (those
that satisfy the constrains) lie within the convex region, and all infeasible
solutions (those that break at least one constraint) lie outside this convex
region.

As we are interested in optimising some linear function, that is either
minimising or maximising some function, the solution must lie at the very edge
of the feasible convex region. That is we have improved so much that if we were
to improve any further we would lie outside the feasible region - hence the
optimum lies on the edge.

Linear programming employs algorithms such as the Simplex method to
mathematically traverse the edges of the feasible convex region, and stops at
the optimum. Therefore we only need to define out objective function and
constraints in a linear fashion, and then apply appropriate algorithms.

Consider a 2-dimensional example: I am able to make $\pounds 50$ profit on each
tonne of paint A I produce, and $\pounds 60$ profit on each tonne of paint B I
produce. A tonne of paint A needs 4 tonnes of ingredient X and 5 tonnes of
ingredient Y. A tonne of paint B needs 6 tonnes of ingredient X and 4 tonnes of
ingredient Y. Only 24 tonnes of X and 20 tonnes of Y available per day. How much
of paint A and paint B should I produce daily to maximise profit?

This is formulated as a linear objective function (total profit) to maximise,
and two linear constraints (availability of ingredients X and Y). They are
written as:

\begin{align}
\text{Maximise: } 50 A + 60 B & \\\label{eqn:objective}
\text{Subject to: } & \nonumber \\
4 A + 6 B &\leq 24 \\\label{eqn:ingredientX}
5 A + 4 B &\leq 20 \label{eqn:ingredientY}
\end{align}

Now we have a linear system in 2-dimensional space with coordinates A and B.
These are called the decision variables, whose values we wish to find that
optimises the objective given by expression~\ref{eqn:objective}.
Inequalities~\ref{eqn:ingredientX} and~\ref{eqn:ingredientY} correspond to the
amount of ingredient X and Y available per day. These, along with the additional
constraints that we cannot produce a negative amount of paint
($A \geq 0$ and $B \geq 0$), form the convex feasible region shown in
Figure~\ref{fig:paint_lp}.

\begin{figure}
\begin{center}
\includestandalone[width=\textwidth]{assets/paint_LP}
\end{center}
\caption{Visual representation of the paint linear program. The feasible convex
region is shaded in grey; the objective function with arbitrary value is shown
in a dashed line.}
\label{fig:paint_lp}
\end{figure}

Expression~\ref{eqn:objective} corresponds to the total profit, which is the
expression we are trying to maximise. As a line in the 2-dimensional space, this
expression fixes its gradient, but its value determines the size of the
$y$-intercept. Therefore optimising this function corresponds to pushing a line
with that gradient to its furthest extreme within the feasible region,
demonstrated in Figure~\ref{paint_lp}.
Therefore for this problem the optimum occurs in a particular vertex of the
feasible region, at $A = \frac{12}{7}$ and $B = \frac{20}{7}$.

This works well as $A$ and $B$ can take any real value in the feasible region.
It is common however to formulate Integer Linear Programmes where the decision
variables are restricted to integers. There are a number of methods that can
help us adapt a real solution to an integer solution. These include cutting
planes, which introduce new constraints around the real solution to force an
integer value; and branch and bound methods, where we iteratively convert
decision variables to their closest two integers and remove any infeasible
solutions.

Both Python and R have libraries that carry out the linear and integer
programming algorithms for us. When solving these kinds of problems, formulating
them as linear systems is the most important challenge.

Consider again the exam scheduling problem from Section~\ref{sec:problem}.
There are eight distinct sets of modules in which exams cannot be scheduled
simultaneously: $A_c$, $A_o$ representing core and optional art modules
respectively; $B_c$, $B_o$ representing core and optional biology modules
respectively; $C_c$, $C_o$ representing core and optional chemistry modules
respectively; and $D_c$, $D_o$ representing core and optional Dutch modules
respectively.

\begin{figure}
\begin{center}
\includestandalone[width=0.75\textwidth]{assets/clashes}
\end{center}
\caption{Visualisation of sets of modules with shared students.}
\label{fig:clashes}
\end{figure}

Additionally there are further clashes between these sets, this can be
visualised as a graph, shown in Figure~\ref{fig:clashes} where edges between
sets represent shared students.
This shows there are five cliques (shown in red, blue, green, orange and
yellow), that is five larger sets of modules that may share students:
$C_o \cup C_c \cup B_o$, $C_o \cup B_o \cup B_c$, $B_c \cup B_o \cup A_c$,
$A_c \cup A_o \cup D_c$, and $D_c \cup D_o$. These sets of modules will form the
basis for out constraints, Inequalities~\ref{eqn:clique1} to~\ref{eqn:clique5}.

Define $M$ as the set of all modules to be scheduled, that is
$M = A_c \cup A_o \cup B_c \cup B_o \cup C_c \cup C_o \cup D_c \cup D_o$.
Define also $T$ as the set of possible time slots. At worst each exam is
scheduled for a different day, thus $|T| = |M| = 26$ in this case.
Let $\{X_{mt} \text{ for } m \in M \text{ and } t \in T\}$ be a set of binary
decision variables, that is $X_{mt}$ is 1 if module $m$ is scheduled for time
$t$, and 0 otherwise.
Let's also define $\{Y_t \text{ for } t \in T\}$ as a set of auxiliary binary
decision variables, where $Y_t$ is 1 if time slot $t$ is being used. This is
enforced by Inequality~\ref{eqn:auxiliary}.

Finally we have one final constraint, Inequality~\ref{eqn:schedule_all_modules},
which ensures all modules are scheduled once and once only.
Thus altogether our integer program becomes:

\begin{align}
\text{Minimise: } \sum_{t \in T} Y_j & \\\label{eqn:objective}
\text{Subject to: } & \nonumber \\
\frac{1}{|M|} \sum_{m \in M} X_{mt} &\leq Y_j \text{ for all } j \in T\\\label{eqn:auxiliary}
\sum_{t \in T} X_{mt} &= 1 \text{ for all } m \in M\\\label{eqn:schedule_all_modules}
\sum_{m \in C_o \cup C_c \cup B_o} X_{mt} &\leq 1 \text{ for all } t \in T\\\label{eqn:clique1}
\sum_{m \in C_o \cup B_o \cup B_c} X_{mt} &\leq 1 \text{ for all } t \in T\\\label{eqn:clique2}
\sum_{m \in B_c \cup B_o \cup A_c} X_{mt} &\leq 1 \text{ for all } t \in T\\\label{eqn:clique3}
\sum_{m \in A_c \cup A_o \cup D_c} X_{mt} &\leq 1 \text{ for all } t \in T\\\label{eqn:clique4}
\sum_{m \in D_c \cup D_o} X_{mt} & \leq 1 \text{ for all } t \in T\\\label{eqn:clique5}
\end{align}


\section{Solving with Python}\label{sec:solving-with-python}
In this book we will use the Python library \mintinline{python}{PuLP} to
formulate and solve the integer program. First let's define all the sets we
will use to formulate the problem.

\begin{pyin}
Ac = [0, 1, 2]
Ao = [3, 4, 5, 6, 7]
Bc = [8, 9, 10, 11, 12]
Bo = [13, 14, 15]
Cc = [16, 17, 18, 19]
Co = [20, 21]
Dc = [22, 23]
Do = [24, 25]
modules = Ac + Ao + Bc + Bo + Cc + Co + Dc + Do
time_slots = range(26)
\end{pyin}

Now let's begin by defining an empty problem:

\begin{pyin}
import pulp
prob = pulp.LpProblem("ExamScheduling", pulp.LpMinimize)
\end{pyin}

We also need to define our sets of binary decision variables:

\begin{pyin}
x = pulp.LpVariable.dicts("X", (modules, time_slots), cat=pulp.LpBinary)
y = pulp.LpVariable.dicts('Y', time_slots, cat=pulp.LpBinary)
\end{pyin}

Now \mintinline{python}{y} is a dictionary of binary decision variables, with
keys as elements of the list \mintinline{python}{time_slots}. Let's look at
$Y_3$ corresponding to the third day:

\begin{pyin}
print(y[3])
\end{pyin}

\begin{pyout}
Y_3
\end{pyout}

While \mintinline{python}{x} is a dictionary of dictionaries of binary decision
variables, with keys as elements of the lists \mintinline{python}{modules} and
\mintinline{python}{time_slots}. Let's look at $X_{2,5}$, the variable
corresponing to module 2 being scheduled on day 5:

\begin{pyin}
print(x[2][5])
\end{pyin}

\begin{pyout}
X_2_5
\end{pyout}

Now we have an empty problem, all relevant sets, and all decision variables
defined, we can go ahead and add the objective function and constraints to the
problem.

For the objective function:

\begin{pyin}
objective_function = sum([y[day] for day in time_slots])
prob += objective_function
\end{pyin}

Now the constraints:

\begin{pyin}
for day in time_slots:
    prob += (1/len(modules)) * sum([x[module][day] for module in modules]) <= y[day]
    prob += sum([x[module][day] for module in Ac+Ao+Dc]) <= 1
    prob += sum([x[module][day] for module in Bc+Bo+Co]) <= 1
    prob += sum([x[module][day] for module in Bc+Bo+Ac]) <= 1
    prob += sum([x[module][day] for module in Cc+Co+Bo]) <= 1
    prob += sum([x[module][day] for module in Dc+Do]) <= 1

for module in modules:
    prob += sum([x[module][day] for day in time_slots]) == 1
\end{pyin}

At this stage we could print the \mintinline{python}{prob} object, which would
explicitly give all constraints written out fully. This can be used to error
check is the need arises.

Now we can go ahead and solve the problem:

\begin{pyin}
prob.solve()
\end{pyin}

This returns a status code: 1 for `optimal', 0 for `not solved', -1 for
`infeasible', -2 for `unbounded', or -3 for `undefined'. A problem is
successfully solved is this methods returns 1:

\begin{pyout}
1
\end{pyout}

This method has also assigned values to our decision variables. These can be
inspected, lets check if module 2 was scheduled for day 5:

\begin{pyin}
print(x[2][5])
\end{pyin}

\begin{pyout}
0.0
\end{pyout}

This was assigned the value 0, and so module 2 was not scheduled for that day.
Let's check if module 2 was scheduled for day 14:

\begin{pyin}
print(x[2][14])
\end{pyin}

\begin{pyout}
1.0
\end{pyout}

This was assigned a value of 1, and so module 2 was scheduled for that day.

We can iterate through all decision variables and make a print solutions in
order to read off the schedule easier:

\begin{pyin}
for day in time_slots:
    if y[day].value() == 1:
        schedule = f"Day {day}: "
        for module in modules:
            if x[module][day].value() == 1:
                 schedule += f"{module}, "
        print(schedule)
\end{pyin}

giving:

\begin{pyout}
Day 0: 7, 9, 25,
Day 1: 6, 12, 17, 24,
Day 3: 3, 14,
Day 4: 10, 19,
Day 10: 4, 8, 16,
Day 11: 13, 22,
Day 12: 11, 23,
Day 13: 5, 15,
Day 14: 2, 20,
Day 17: 0, 18,
Day 18: 1, 21,
\end{pyout}

Now the order of the days do not matter here, but we can see that 11 days are
required in order to schedule all exams with no clashes. Most days (8) have two
exams scheduled, 2 days have 3 exams scheduled, and it was possible on 1 day to
have 4 exams scheduled simultaneously.


\section{Solving with R}\label{sec:solving-with-R}

In R we will use the R package \mintinline{R}{ROI}, the R Optimization Infrastructure.
This is a library of code that acts as a front end to a number of other solvers that need to be installed externally, allowing a range of optimisation problems to be solved with a number of different solvers, using similar problem structures and syntax.
The solver that we will use here is called the CBC MILP Solver, which needs to be installed as well as \mintinline{R}{rcbc} package.

In Section~\ref{sec:theory} the exam scheduling problem was formulated with two arrays of decision variables, a one-dimensional array $Y$ and a two dimensional array $X$.
In \mintinline{R}{ROI} we need our decision variables to be a one-dimensional array, so we `flatten' these by first ordering by $X$ then $Y$, within that ordering by time slot, then within that ordering by module number.
Call these variables $Z_i$ for $i \in 1, 2, ... 702$.
For example $Z_{29}$ would correspond to $X_{3, 2}$, the decision variable representing whether module number 3 is scheduled on day 2; $Z_{700}$ would correspond to $Y_{24}$, the decision variable representing whether there's an exam scheduled for day 24.

Once this array is defined, then the objective function is an ordered array of the coefficients of each of these variables:

\begin{Rin}
#' Writes the row of coefficients for the objective function
#'
#' @param n_modules: the number of modules to schedule
#' @param n_days: the maximum number of days to schedule
#'
#' @return the objective function row to minimise
write_objective <- function(n_modules, n_days){
  all_days <- rep(0, n_modules * n_days)
  Ys <- rep(1, n_days)
  append(all_days, Ys)
}
\end{Rin}

For 3 modules and 3 days:

\begin{Rin}
write_objective(3, 3)
\end{Rin}

Which gives the following array, corresponding the the coefficients of the array $Z$ for Equation~\ref{eqn:objective}.

\begin{Rout}
[1] 0 0 0 0 0 0 0 0 0 1 1 1
\end{Rout}

Now the constraints have three components: the coefficients matrix, the right hand side array, and the array of directions.
The coefficients matrix is a matrix of the coefficients of the left hand side of inequalities, where again columns represent the coefficients of the array $Z$, and rows represent each separate constraint.
The following function is used to write one row of that coefficients matrix, for a given day, for a given set of clashes, corresponding to Inequalities~\ref{eqn:clique1} to \ref{eqn:clique5}:

\begin{Rin}
#' Writes the constraint row dealing with clashes
#'
#' @param clashes: a vector of module indices that all cannot
#'                 be scheduled at the same time
#' @param day: an integer representing the day
#'
#' @return the constraint row corresponding to that set of
#'         clashes on that day
write_X_clashes <- function(clashes, day, n_days, n_modules){
  today <- rep(0, n_modules)
  today[clashes] = 1
  before_today <- rep(0, n_modules * (day - 1))
  after_today <- rep(0, n_modules * (n_days - day))
  all_days <- c(before_today, today, after_today)
  full_coeffs <- c(all_days, rep(0, n_days))
  full_coeffs
}
\end{Rin}

where \mintinline{clashes} is an array containing the module numbers of a set of modules that may all share students.
The following function is used to write one row of the coefficients matrix, for each module, ensuring that each module is scheduled on one day and one day only, corresponding to Equation~\ref{eqn:schedule_all_modules}:

\begin{Rin}
#' Writes the constraint row to ensure that every module is
#' scheduled once and only one
#'
#' @param module: an integer representing the module
#'
#' @return the constraint row corresponding to scheduling a
#'         module on only one day
write_X_requirements <- function(module, n_days, n_modules){
  today <- rep(0, n_modules)
  today[module] = 1
  all_days <- rep(today, n_days)
  full_coeffs <- c(all_days, rep(0, n_days))
  full_coeffs
}
\end{Rin}

The following function is used to write one row of the coefficients matrix corresponding to the auxiliary constraints of Inequalities~\ref{ref:auxiliary}:

\begin{Rin}
#' Writes the constraint row representing the Y variable,
#' whether at least one exam is scheduled on that day
#'
#' @param day: an integer representing the day
#'
#' @return the constraint row corresponding to creating Y
write_Y_constraints <- function(day, n_days, n_modules){
  today <- rep(1, n_modules)
  before_today <- rep(0, n_modules * (day - 1))
  after_today <- rep(0, n_modules * (n_days - day))
  all_days <- c(before_today, today, after_today)
  all_Ys <- rep(0, n_days)
  all_Ys[day] = -n_modules
  full_coeffs <- append(all_days, all_Ys)
  full_coeffs
}
\end{Rin}

Finally the following function uses them all to assemble a coefficients matrix.
It loops though the parameters for each constraint row required, uses the appropriate function to create the row of the coefficients matrix, sets the appropriate inequality direction ($\leq$, $=$, $\geq$), and the value of the right hand side.
It returns all three components:

\begin{Rin}
#' Writes all the constraints as a matrix, column of
#' inequalities, and right hand side column.
#'
#' @param list_clashes: a list of vectors with sets of modules
#         that cannot be scheduled at the same time
#'
#' @return f.con the LHS of the constraints as a matrix
#' @return f.dir a vector of directions of the inequalities
#' @return f.rhs a vector of the values of the RHS of the inequalities
write_constraints <- function(list_clashes, n_days, n_modules){
  all_rows <- c()
  all_dirs <- c()
  all_rhss <- c()
  n_rows <- 0
  
  for (clash in list_clashes){
    for (day in 1:n_days){
      clashes <- write_X_clashes(clash, day, n_days, n_modules)
      all_rows <- append(all_rows, clashes)
      all_dirs <- append(all_dirs, "<=")
      all_rhss <- append(all_rhss, 1)
      n_rows <- n_rows + 1
    }
  }
  
  for (module in 1:n_modules){
    reqs <- write_X_requirements(module, n_days, n_modules)
    all_rows <- append(all_rows, reqs)
    all_dirs <- append(all_dirs, "==")
    all_rhss <- append(all_rhss, 1)
    n_rows <- n_rows + 1
  }
  
  for (day in 1:n_days){
    Yconstraints <- write_Y_constraints(day, n_days, n_modules)
    all_rows <- append(all_rows, Yconstraints)
    all_dirs <- append(all_dirs, "<=")
    all_rhss <- append(all_rhss, 0)
    n_rows <- n_rows + 1
  }
  
  f.con <- matrix(all_rows, nrow = n_rows, byrow = TRUE)
  f.dir <- all_dirs
  f.rhs <- all_rhss
  list(f.con, f.dir, f.rhs)
}
\end{Rin}

For demonstration, if we had two modules and two possible days, with the single constraint that both modules cannot be scheduled at the same time, then:

\begin{Rin}
write_constraints(list_clashes = list(c(1, 2)),
                  n_days = 2,
                  n_modules = 2)
\end{Rin}

This would give three components, a coefficient matrix of the left hand side of the constraints (rows 1 and 2 corresponding to the clash on days 1 and 2, row 3 ensuring module 1 is scheduled on one day only, row 4 ensuring module 2 is scheduled on one day only, and rows 5 and 5 defining the Y decision variables), an array of direction of the constraint inequalities, and an array of the right hand side values of the constraints.

\begin{Rout}
[[1]]
     [,1] [,2] [,3] [,4] [,5] [,6]
[1,]    1    1    0    0    0    0
[2,]    0    0    1    1    0    0
[3,]    1    0    1    0    0    0
[4,]    0    1    0    1    0    0
[5,]    1    1    0    0   -2    0
[6,]    0    0    1    1    0   -2

[[2]]
[1] "<=" "<=" "==" "==" "<=" "<="

[[3]]
[1] 1 1 1 1 0 0
\end{Rout}

Now we are ready to use these to solve the exam scheduling problem.
First we define some parameters, including the sets of modules that all share students, that is the list of clashes:

\begin{Rin}
n_modules = 26
n_days = 26

Ac <- c(1, 2, 3)
Ao <- c(4, 5, 6, 7, 8)
Bc <- c(9, 10, 11, 12, 13)
Bo <- c(14, 15, 16)
Cc <- c(17, 18, 19, 20)
Co <- c(21, 22)
Dc <- c(23, 24)
Do <- c(25, 26)

list_clashes <- list(
  c(Ac, Ao, Dc),
  c(Bc, Bo, Co),
  c(Bc, Bo, Ac),
  c(Cc, Co, Bo),
  c(Dc, Do)
)
\end{Rin}

Then we can use the functions defined above to create the objective function and the three elements of the constraints:

\begin{Rin}
constraints <- write_constraints(list_clashes = list_clashes,
                                 n_days = n_days,
                                 n_modules = n_modules)
f.con <- constraints[[1]]
f.dir <- constraints[[2]]
f.rhs <- constraints[[3]]
f.obj <- write_objective(n_modules = n_modules, n_days = n_days)
\end{Rin}

Finally, once these objects are in place, we can use the \mintinline{ROI} library to construct an optimisation problem object:

\begin{Rin}
library(ROI)

milp <- OP(objective = L_objective(f.obj),
           constraints = L_constraint(L = f.con,
                                      dir = f.dir,
                                      rhs = f.rhs),
           types = rep("B", length(f.obj)), 
           maximum = FALSE)
\end{Rin}

This creates an \mintinline{R}{OP} object from our objective row \mintinline{R}{f.obj}, and our constraints which are made up from the three components \mintinline{R}{f.con}, \mintinline{R}{f.dir} and \mintinline{R}{f.rhs}.
When creating this object we also denote the \mintinline{R}{types} as binary variables (an array o \mintinline{R}{"B"} for each decision variable), and we want to minimise out objective function so we set \mintinline{R}{maximum = FALSE}.

Now to solve:

\begin{Rin}
sol <- ROI_solve(milp)
\end{Rin}

The solver will not output information about the solve process and runtime.
We can now print the solution:

\begin{Rin}
print(sol$solution)
\end{Rin}

\begin{Rout}
  [1] 0 0 0 0 0 0 0 0 0 0 0 0 0 0 0 0 0 0 0 0 0 0 0 0 0 0 0 0 0
 [30] 0 0 0 0 0 0 0 0 0 0 0 0 0 0 0 0 0 0 0 0 0 0 0 0 0 0 0 0 0
 [59] 0 0 0 0 0 0 0 0 0 0 0 0 0 0 0 0 0 0 0 0 0 0 0 0 0 0 0 0 0
 [88] 0 0 0 0 0 0 0 0 0 0 0 0 0 0 0 0 0 0 0 0 0 0 0 0 0 0 0 0 0
[117] 0 0 0 0 0 0 0 0 0 0 0 0 0 0 0 1 0 0 0 0 0 0 0 0 0 0 0 0 0
[146] 0 0 0 0 0 0 1 0 0 0 0 0 0 0 0 0 0 0 0 0 0 0 0 0 0 0 0 0 0
[175] 0 0 0 0 0 0 0 0 0 0 0 0 0 0 0 1 1 0 0 0 0 0 0 0 0 0 1 0 0
[204] 0 0 0 0 0 0 0 0 0 0 0 0 0 0 0 0 0 0 0 0 0 0 0 0 0 0 0 0 0
[233] 0 0 0 0 0 0 1 0 0 0 0 0 0 0 1 0 0 0 0 0 0 1 0 0 0 0 0 0 0
[262] 0 1 0 0 0 0 0 0 0 0 0 0 0 0 0 0 0 0 0 1 0 0 0 0 1 0 0 0 0
[291] 0 0 0 0 0 0 0 0 0 0 0 0 0 0 0 0 0 0 0 0 0 0 0 0 0 0 0 0 0
[320] 0 0 0 0 0 0 0 0 1 0 0 0 0 0 0 0 0 0 0 0 0 0 0 0 1 0 0 0 0
[349] 0 0 0 0 1 0 0 0 0 0 0 0 0 0 0 0 0 0 0 0 0 0 0 0 0 0 0 0 0
[378] 0 0 0 0 0 0 0 0 0 0 0 0 0 0 0 0 0 0 0 0 0 0 0 0 0 0 0 0 0
[407] 0 0 0 0 0 0 0 0 0 0 0 0 0 1 0 0 0 0 0 0 0 0 0 1 0 0 0 0 0
[436] 0 0 0 0 0 1 0 0 0 0 0 0 0 1 0 0 0 1 0 0 0 0 0 0 0 0 0 0 0
[465] 0 0 0 0 0 0 0 0 0 0 0 0 0 0 0 0 0 0 0 0 0 0 0 0 0 0 0 0 0
[494] 0 0 0 0 0 0 0 0 0 0 0 0 0 0 0 0 0 0 0 0 0 0 0 0 0 0 0 0 0
[523] 0 0 0 0 0 0 0 0 0 0 0 0 0 0 0 0 0 0 0 0 0 0 0 0 0 0 0 0 0
[552] 0 0 0 0 0 0 1 0 0 0 0 1 0 0 0 0 0 0 1 0 0 0 0 0 0 0 0 0 0
[581] 0 1 0 0 0 0 0 0 0 0 0 0 0 0 1 0 0 0 0 0 0 0 0 0 0 0 0 0 0
[610] 0 0 0 0 0 0 0 0 0 0 0 0 0 0 0 1 0 0 0 0 0 0 0 0 0 0 0 0 0
[639] 0 0 0 1 0 0 0 0 0 0 0 0 0 0 0 0 0 0 0 0 0 0 0 0 0 0 0 0 0
[668] 0 0 0 0 0 0 0 0 0 0 0 0 0 0 1 0 1 0 1 1 0 1 1 0 0 1 1 0 0
[697] 0 1 1 0 1 0
\end{Rout}

This gives the values of each of the $Z$ decision variables.
We know the structure of this, that is the first 26 variables are the modules scheduled for day 1, and so on.
The following code prints a readable schedule:

\begin{Rin}
for (day in 1:n_days){
  if (sol$solution[(n_days*n_modules) + day] == 1){
    schedule <- paste("Day", day, ":")
    for (module in 1:n_modules){
      var <- ((day - 1) * n_modules) + module
      if (sol$solution[var] == 1){
        schedule <- paste(schedule, module)
      }
    }
    print(schedule)
  }
}
\end{Rin}

\begin{Rout}
[1] "Day 6 : 2 22"
[1] "Day 8 : 8 9 19"
[1] "Day 10 : 5 13 20"
[1] "Day 11 : 3 21 26"
[1] "Day 13 : 16"
[1] "Day 14 : 6 15"
[1] "Day 17 : 4 14 25"
[1] "Day 18 : 7 11"
[1] "Day 22 : 12 17 24"
[1] "Day 23 : 10 23"
[1] "Day 25 : 1 18"
\end{Rout}

Giving that 11 days are the minimum required to schedule the 26 exams without clashes.

\section{Research}\label{sec:research}
