\chapter[Linear programming]{Linear Programming}\label{chp:linear_programming}


% Introduction
\chapterinitial{F}{inding} the best configuration of some system can be
challenging, especially when there is a seemingly endless amount of possible
solutions. Optimisation techniques are a way to mathematically derive solutions
that maximise or minimise some objective function, subject to a number of
feasibility constraints. When all components of the problem can be written in a
linear way, then linear programming is one technique that can be used to find
the solution.

\section{Problem}\label{sec:problem}

A university runs 14 modules over three subjects: Art, Biology, and Chemistry.
Each subject runs core modules and optional modules.
Table~\ref{tab:modules} gives the module numbers for each of these.

\begin{table}
\begin{center}
\begin{tabular}{lll}
\toprule
\textbf{Art Core} & \textbf{Biology Core} & \textbf{Chemistry Core} \\
\midrule
M00 & M05 & M09 \\
M01 & M06 & M10 \\
\midrule
\textbf{Art Optional} & \textbf{Biology Optional} & \textbf{Chemistry Optional}\\
\midrule
M02 & M07 & M11 \\
M03 & M08 & M12 \\
M04 &     & M13 \\
\bottomrule
\end{tabular}
\end{center}
\caption{List of modules on offer at the university.}
\label{tab:modules}
\end{table}

The university is required to schedule examinations for each of these modules.
The university would like the exams to be scheduled using the least amount of
time slots possible. However not all modules can be scheduled at the same time
as they share some students:

\begin{itemize}
  \item All art modules share students,
  \item All biology modules share students,
  \item All chemistry modules share students,
  \item Biology students have the option of taking optional modules from
  chemistry, so all biology modules may share students with optional chemistry
  modules,
  \item Chemistry students have the option of taking optional modules from
  biology, so all chemistry modules may share students with optional biology
  modules,
  \item Biology students have the option of taking core art modules, and so all
  biology modules may share students with core art modules.
\end{itemize}

How can every exam be scheduled with no clashes, using the least amount of
time slots?

\section{Theory}\label{sec:theory}

Linear programming is a method that solves a type of optimisation problem of a
number of variables by making use of some concepts of higher dimensional
geometry\autocite{conforti2014integer}.
Optimisation here refers to finding the variable that gives either the maximum
or minimum of some linear function, called the objective function.

Linear programming employs algorithms such as the Simplex
method
to efficiently search some feasible convex region, stopping at the optimum. To
do this, an objective function function and constraints need to be defined.

To illustrate this a classic 2-dimensional example will be used:
\(\pounds 50\) of profit can be made on each
tonne of paint A produced, and \(\pounds 60\) profit on each tonne of paint B
produced. A tonne of paint A needs 4 tonnes of component X and 5 tonnes of
component Y. A tonne of paint B needs 6 tonnes of component X and 4 tonnes of
component Y. Only 24 tonnes of X and 20 tonnes of Y are available per day. How
much of paint A and paint B should be produced to maximise profit?

This is formulated as a linear objective function, representing total profit,
that is to be maximised; and two linear constraints, representing the
availability of components X and Y. They are written as:

\begin{align}
\text{Maximise: } 50 A + 60 B & \label{eqn:objective_paint} \\
\text{Subject to: } & \nonumber \\
4 A + 6 B &\leq 24 \label{eqn:ingredientX} \\
5 A + 4 B &\leq 20 \label{eqn:ingredientY}
\end{align}

Now this is a linear system in 2-dimensional space with coordinates A and B.
These are called the decision variables, what is required are the values of A
and B that optimises the objective function given by
expression~\ref{eqn:objective_paint}.

Inequalities~\ref{eqn:ingredientX} and~\ref{eqn:ingredientY} correspond to the
amount of component X and Y available per day. These, along with the additional
constraints that a negative amount of paint cannot be produced (\(A \geq 0\) and
\(B \geq 0\)), form a convex region, shown in Figure~\ref{fig:paint_lp}.
This shaded region shows the pairs of values of \(A\) and \(B\) which are
feasible, that is they satisfy the constraints.

\begin{figure}
\begin{center}
\includestandalone[width=\textwidth]{assets/paint_LP}
\end{center}
\caption{Visual representation of the paint linear program. The feasible convex
region is shaded in grey; the objective function with arbitrary value is shown
in a dashed line.}
\label{fig:paint_lp}
\end{figure}

Expression~\ref{eqn:objective_paint} corresponds to the total profit, which is
the value to be maximised. As a line in 2-dimensional space,
this expression fixes its gradient, but its value determines the size of the
\(y\)-intercept. Therefore optimising this function corresponds to pushing a line
with that gradient to its furthest extreme within the feasible region,
demonstrated in Figure~\ref{fig:paint_lp}.
Therefore for this problem the optimum occurs in a particular vertex of the
feasible region, at \(A = \frac{12}{7}\) and \(B = \frac{20}{7}\).

This works well as \(A\) and \(B\) can take any real value in the feasible region.
Some problems must be formulated as integer linear programs where the decision
variables are restricted to integers. There are a number of methods that can
help adapt a real solution to an integer solution. These include cutting
planes, which introduce new constraints around the real solution to force an
integer value; and branch and bound methods, where we iteratively convert
decision variables to their closest two integers and remove any infeasible
solutions\autocite{conforti2014integer}.

Both Python and R have libraries that carry out the linear and integer
programming algorithms. When solving these kinds of problems, formulating
them as linear systems is the most important challenge.

Consider again the exam scheduling problem from Section~\ref{sec:problem} which
will now be formulated as an integer linear program.
Define \(M\) as the set of all modules to be scheduled, and define \(T\) as the set
of possible time slots. At worst each exam is
scheduled for a different day, thus \(|T| = |M| = 14\) in this case.
Let \(\{X_{mt} \text{ for } m \in M \text{ and } t \in T\}\) be a set of binary
decision variables, that is \(X_{mt} = 1\) if module \(m\) is scheduled for time
\(t\), and \(0\) otherwise.

There are six distinct sets of modules in which exams cannot be scheduled
simultaneously: \(A_c\), \(A_o\) representing core and optional art modules
respectively; \(B_c\), \(B_o\) representing core and optional biology modules
respectively; and \(C_c\), \(C_o\) representing core and optional chemistry modules
respectively.
Therefore \(M = A_c \cup A_o \cup B_c \cup B_o \cup C_c \cup C_o\).

Additionally there are further clashes between these sets:
\begin{itemize}
  \item No modules in \(A_c \cup A_o\) can be scheduled together as they may
  share students, this is given by the constraint in inequality~\ref{eqn:clique1}.
  \item No modules in \(B_c \cup B_o \cup A_c\), can be scheduled together as
  they may share students, given by inequality~\ref{eqn:clique2}.
  \item No modules in \(B_c \cup B_o \cup C_o\), can be scheduled together as
  they may share students, given by inequality~\ref{eqn:clique3}.
  \item No modules in \(B_o \cup C_c \cup C_o\), can be scheduled together as
  they may share students, given by inequality~\ref{eqn:clique4}.
\end{itemize}

Define \(\{Y_t \text{ for } t \in T\}\) as a set of auxiliary binary
decision variables, where \(Y_t\) is 1 if time slot \(t\) is being used. This is
enforced by Inequality~\ref{eqn:auxiliary}.

Equation~\ref{eqn:schedule_all_modules},
ensures all modules are scheduled once and once only.
Thus altogether the integer program becomes:

\begin{align}
\text{Minimise: } \sum_{t \in T} Y_j & \label{eqn:objective_modules} \\
\text{Subject to: } & \nonumber \\
\frac{1}{|M|} \sum_{m \in M} X_{mt} &\leq Y_j \text{ for all } j \in T \label{eqn:auxiliary} \\
\sum_{t \in T} X_{mt} &= 1 \text{ for all } m \in M \label{eqn:schedule_all_modules} \\
\sum_{m \in A_c \cup A_o} X_{mt} &\leq 1 \text{ for all } t \in T \label{eqn:clique1} \\
\sum_{m \in B_c \cup B_o \cup A_c} X_{mt} &\leq 1 \text{ for all } t \in T \label{eqn:clique2} \\
\sum_{m \in B_c \cup B_o \cup C_o} X_{mt} &\leq 1 \text{ for all } t \in T \label{eqn:clique3} \\
\sum_{m \in B_o \cup C_c \cup C_o} X_{mt} &\leq 1 \text{ for all } t \in T \label{eqn:clique4}
\end{align}

Another common way to define this linear program is by representing the
coefficients of the constraints as a matrix.
That is:

\begin{align}
\text{Minimise: } c^T Z & \\
\text{Subject to: } & \nonumber \\
A Z & \star b
\end{align}

where \(Z\) is a vector representing the decision variables, \(c\) is the
coefficients of the \(Z\) in the objective function, \(A\) is the matrix of the
coefficients of \(Z\) in the constraints, \(b\) is the vector of the right hand
side of the constraints, and \(\star\) represents either \(\leq\), \(=\) or
\(\geq\) as required.

As \(Z\) is a one-dimensional vector of decisions variables,
the matrix \(X\) and the vector \(Y\) can be `flattened' together to form this new variable.
This is done by first ordering \(X\) then \(Y\), within that
ordering by time slot, then within that ordering by module number.
Therefore:

\begin{align}
Z_{|M|t + m} &= X_{mt}\\
Z_{|M|^2 + m} &= Y_m
\end{align}

where \(t\) and \(m\) are indices starting at 0.
For example \(Z_{17}\) would correspond to \(X_{3, 2}\), the decision variable
representing whether module number 4 is scheduled on day 3; \(Z_{208}\) would
correspond to \(Y_{12}\), the decision variable representing whether there is an
exam scheduled for day 12.

Parameters \(c\), \(A\), and \(b\) can be determined by using this same conversion
from the model in Equations~\ref{eqn:objective_modules} to \ref{eqn:clique4}.
The vector \(c\) would be \(|M|^2\) zeroes followed by \(|M|\) ones.
The vector \(b\) would be zeroes for all the rows representing
Equation~\ref{eqn:auxiliary}, and ones for all other constraints.

\section{Solving with Python}\label{sec:solving-with-python}

In this book the Python library Pulp will be used to
formulate and solve the integer program. First a function to create the
binary problem variables for a given set of times and modules is needed:

\begin{pyin}
import pulp


def get_variables(modules, times):
    """Returns the binary variables for a given timetabling
    problem.

    Args:
        modules: The complete collection of modules to be
                 timetabled.
        times: The collection of available time slots.

    Returns:
        A tuple containing the decision variables x and y.
    """
    xshape = (modules, times)
    x = pulp.LpVariable.dicts("X", xshape, cat=pulp.LpBinary)
    y = pulp.LpVariable.dicts("Y", times, cat=pulp.LpBinary)
    return x, y
\end{pyin}

The specific modules and times relating to the problem can now be used to
obtain the corresponding variables:

\begin{pyin}
Ac = [0, 1]
Ao = [2, 3, 4]
Bc = [5, 6]
Bo = [7, 8]
Cc = [9, 10]
Co = [11, 12, 13]
modules = Ac + Ao + Bc + Bo + Cc + Co
times = range(14)

x, y = get_variables(modules=modules, times=times)
\end{pyin}

Now \mintinline{python}{y} is a dictionary of binary decision variables, with
keys as elements of the list \mintinline{python}{times}. \(Y_3\) corresponds to
the third day:

\begin{pyin}
print(y[3])
\end{pyin}

\begin{pyout}
Y_3
\end{pyout}

While \mintinline{python}{x} is a dictionary of dictionaries of binary decision
variables, with keys as elements of the lists \mintinline{python}{modules} and
\mintinline{python}{times}. \(X_{2,5}\) is the variable corresponding
to module 2 being scheduled on day 5:

\begin{pyin}
print(x[2][5])
\end{pyin}

\begin{pyout}
X_2_5
\end{pyout}


The next step is to create a specific program with the corresponding variables,
objective function, constraints and solve it. This is done with the following
function:

%% TODO remove message supressor
\begin{pyin}
def get_solution(Ac, Ao, Bc, Bo, Cc, Co, times):
    """Returns the binary variables corresponding to the
    solution of given timetabling problem.

    Args:
        Ac: The set of core art modules
        Ao: The set of optional art modules
        Bc: The set of core biology modules
        Bo: The set of optional biology modules
        Cc: The set of core chemistry modules
        Co: The set of optional chemistry modules
        times: The collection of available time slots.

    Returns:
        A tuple containing the decision variables x and y.
    """
    modules = Ac + Ao + Bc + Bo + Cc + Co
    x, y = get_variables(modules=modules, times=times)

    prob = pulp.LpProblem("ExamScheduling", pulp.LpMinimize)

    objective_function = sum([y[day] for day in times])
    prob += objective_function

    M = 1 / len(modules)
    for day in times:
        prob += M * sum(x[m][day] for m in modules) <= y[day]
        prob += sum([x[mod][day] for mod in Ac + Ao]) <= 1
        prob += sum([x[mod][day] for mod in Bc + Bo + Co]) <= 1
        prob += sum([x[mod][day] for mod in Bc + Bo + Ac]) <= 1
        prob += sum([x[mod][day] for mod in Cc + Co + Bo]) <= 1

    for mod in modules:
        prob += sum(x[mod][day] for day in times) == 1

    prob.solve(pulp.apis.PULP_CBC_CMD(msg=False))

    return x, y
\end{pyin}

Using this, the solution \(x\) of the original problem can be obtained:

%% TODO remove message supressor
\begin{pyin}
x, y = get_solution(
    Ac=Ac, Ao=Ao, Bc=Bc, Bo=Bo, Cc=Cc, Co=Co, times=times
)
\end{pyin}

These can be
inspected, for example \(x_{25}\) is a boolean variable relating to if module 2
is scheduled on the 5th day.

\begin{pyin}
print(x[2][5].value())
\end{pyin}

\begin{pyout}
0.0
\end{pyout}

This was assigned the value 0, and so module 2 was not scheduled for that day.
However, module 2 was scheduled for day 9:

\begin{pyin}
print(x[2][9].value())
\end{pyin}

\begin{pyout}
1.0
\end{pyout}

This was assigned a value of 1, and so module 2 was scheduled for that day.

The following function creates a readable schedule:

\begin{pyin}
def get_schedule(x, y, Ac, Ao, Bc, Bo, Cc, Co, times):
    """Returns a human readable schedule corresponding to the
    solution of given timetabling problem.

    Args:
        Ac: The set of core art modules
        Ao: The set of optional art modules
        Bc: The set of core biology modules
        Bo: The set of optional biology modules
        Cc: The set of core chemistry modules
        Co: The set of optional chemistry modules
        times: The collection of available time slots.

    Returns:
        A string with the schedule
    """
    modules = Ac + Ao + Bc + Bo + Cc + Co

    schedule = ""
    for day in times:
        if y[day].value() == 1:
            schedule += f"\nDay {day}: "
            for mod in modules:
                if x[mod][day].value() == 1:
                    schedule += f"{mod}, "
    return schedule
\end{pyin}

Thus:

\begin{pyin}
schedule = get_schedule(
    x=x,
    y=y,
    times=times,
    Ac=Ac,
    Ao=Ao,
    Bc=Bc,
    Bo=Bo,
    Cc=Cc,
    Co=Co,
)
print(schedule)
\end{pyin}

gives:

\begin{pyout}

Day 0: 1, 12,
Day 5: 0, 13,
Day 6: 11,
Day 7: 4, 6, 10,
Day 8: 3, 5, 9,
Day 9: 2, 7,
Day 13: 8,
\end{pyout}

The order of the days do not matter here, but we 7 days are required in order to
schedule all exams with no clashes, with at most three exams scheduled each day.

\section{Solving with R}\label{sec:solving-with-R}

The R package ROI, the R Optimization
Infrastructure will be used here.
This is a library of code that acts as a front end to a number of other solvers
that need to be installed externally, allowing a range of optimisation problems
to be solved with a number of different solvers.
The solver that will be used here is called the CBC MILP Solver, which needs to
be installed as well as the R \mintinline{R}{rcbc} package.
% TODO Do we need to point at installation documentation for this?

The ROI package requires that the linear program is
represented in its matrix form, with a one-dimensional array of decision
variables. Therefore the form of the model described at the end of
Section~\ref{sec:theory} will be used.
Functions that define the objective function \(c\), the coefficient
matrix \(A$, the vector of the right hand side of the constraints $b\), and the
vector of equality or inequalities directions \(\star\) are needed.

First the objective function:

\begin{Rin-no-test}
#' Writes the row of coefficients for the objective function
#'
#' @param n_modules: the number of modules to schedule
#' @param n_days: the maximum number of days to schedule
#'
#' @return the objective function row to minimise
write_objective <- function(n_modules, n_days){
  all_days <- rep(0, n_modules * n_days)
  Ys <- rep(1, n_days)
  append(all_days, Ys)
}
\end{Rin-no-test}

For 3 modules and 3 days:

\begin{Rin-no-test}
write_objective(n_modules = 3, n_days = 3)
\end{Rin-no-test}

Which gives the following array, corresponding to the coefficients of the
array \(Z\) for Equation~\ref{eqn:objective_modules}.

\begin{Rout-no-test}
[1] 0 0 0 0 0 0 0 0 0 1 1 1
\end{Rout-no-test}

The following function is used to write one row of that coefficients matrix, for
a given day, for a given set of clashes, corresponding to
Inequalities~\ref{eqn:clique1} to \ref{eqn:clique4}:

\begin{Rin-no-test}
#' Writes the constraint row dealing with clashes
#'
#' @param clashes: a vector of module indices that all cannot
#'                 be scheduled at the same time
#' @param day: an integer representing the day
#'
#' @return the constraint row corresponding to that set of
#'         clashes on that day
write_X_clashes <- function(clashes, day, n_days, n_modules){
  today <- rep(0, n_modules)
  today[clashes] = 1
  before_today <- rep(0, n_modules * (day - 1))
  after_today <- rep(0, n_modules * (n_days - day))
  all_days <- c(before_today, today, after_today)
  full_coeffs <- c(all_days, rep(0, n_days))
  full_coeffs
}
\end{Rin-no-test}

where \mintinline{R}{clashes} is an array containing the module numbers of a set
of modules that may all share students.

The following function is used to write one row of the coefficients matrix, for
each module, ensuring that each module is scheduled on one day and one day only,
corresponding to Equation~\ref{eqn:schedule_all_modules}:

\begin{Rin-no-test}
#' Writes the constraint row to ensure that every module is
#' scheduled once and only one
#'
#' @param module: an integer representing the module
#'
#' @return the constraint row corresponding to scheduling a
#'         module on only one day
write_X_requirements <- function(module, n_days, n_modules){
  today <- rep(0, n_modules)
  today[module] = 1
  all_days <- rep(today, n_days)
  full_coeffs <- c(all_days, rep(0, n_days))
  full_coeffs
}
\end{Rin-no-test}

The following function is used to write one row of the coefficients matrix
corresponding to the auxiliary constraints of Inequality~\ref{eqn:auxiliary}:

\begin{Rin-no-test}
#' Writes the constraint row representing the Y variable,
#' whether at least one exam is scheduled on that day
#'
#' @param day: an integer representing the day
#'
#' @return the constraint row corresponding to creating Y
write_Y_constraints <- function(day, n_days, n_modules){
  today <- rep(1, n_modules)
  before_today <- rep(0, n_modules * (day - 1))
  after_today <- rep(0, n_modules * (n_days - day))
  all_days <- c(before_today, today, after_today)
  all_Ys <- rep(0, n_days)
  all_Ys[day] = -n_modules
  full_coeffs <- append(all_days, all_Ys)
  full_coeffs
}
\end{Rin-no-test}

Finally the following function uses all previous functions
to assemble a coefficients matrix.
It loops though the parameters for each constraint row required, uses the
appropriate function to create the row of the coefficients matrix, sets the
appropriate inequality direction (\(\leq$, $=$, $\geq\)), and the value of the
right hand side.
It returns all three components:

\begin{Rin-no-test}
#' Writes all the constraints as a matrix, column of
#' inequalities, and right hand side column.
#'
#' @param list_clashes: a list of vectors with sets of modules
#         that cannot be scheduled at the same time
#'
#' @return f.con the LHS of the constraints as a matrix
#' @return f.dir the directions of the inequalities
#' @return f.rhs the values of the RHS of the inequalities
write_constraints <- function(list_clashes, n_days, n_modules){
  all_rows <- c()
  all_dirs <- c()
  all_rhss <- c()
  n_rows <- 0

  for (clash in list_clashes){
    for (day in 1:n_days){
      clashes <- write_X_clashes(clash, day, n_days, n_modules)
      all_rows <- append(all_rows, clashes)
      all_dirs <- append(all_dirs, "<=")
      all_rhss <- append(all_rhss, 1)
      n_rows <- n_rows + 1
    }
  }

  for (module in 1:n_modules){
    reqs <- write_X_requirements(module, n_days, n_modules)
    all_rows <- append(all_rows, reqs)
    all_dirs <- append(all_dirs, "==")
    all_rhss <- append(all_rhss, 1)
    n_rows <- n_rows + 1
  }

  for (day in 1:n_days){
    Yconstraints <- write_Y_constraints(day, n_days, n_modules)
    all_rows <- append(all_rows, Yconstraints)
    all_dirs <- append(all_dirs, "<=")
    all_rhss <- append(all_rhss, 0)
    n_rows <- n_rows + 1
  }

  f.con <- matrix(all_rows, nrow = n_rows, byrow = TRUE)
  f.dir <- all_dirs
  f.rhs <- all_rhss
  list(f.con, f.dir, f.rhs)
}
\end{Rin-no-test}

For demonstration, with 2 modules and 2 possible days, with the single
constraint that both modules cannot be scheduled at the same time, then:

\begin{Rin-no-test}
write_constraints(list_clashes = list(c(1, 2)),
                  n_days = 2,
                  n_modules = 2)
\end{Rin-no-test}

This would give 3 components:

\begin{itemize}
  \item a coefficient matrix of the left hand side of the constraints, \(A\), (rows 1
  and 2 corresponding to the clash on days 1 and 2, row 3 ensuring module 1 is
  scheduled on one day only, row 4 ensuring module 2 is scheduled on one day
  only, and rows 5 and 6 defining the decision variables \(Y\)),
  \item an array of direction of the constraint inequalities, \(\star\),
  \item and an array of the right hand side values of the constraints, \(b\).
\end{itemize}

\begin{Rout-no-test}
[[1]]
     [,1] [,2] [,3] [,4] [,5] [,6]
[1,]    1    1    0    0    0    0
[2,]    0    0    1    1    0    0
[3,]    1    0    1    0    0    0
[4,]    0    1    0    1    0    0
[5,]    1    1    0    0   -2    0
[6,]    0    0    1    1    0   -2

[[2]]
[1] "<=" "<=" "==" "==" "<=" "<="

[[3]]
[1] 1 1 1 1 0 0
\end{Rout-no-test}

Now, the problem will be solved.
First some parameters, including the sets of modules that all share
students, that is the list of clashes are needed:

\begin{Rin-no-test}
n_modules = 14
n_days = 14

Ac <- c(0, 1)
Ao <- c(2, 3, 4)
Bc <- c(5, 6)
Bo <- c(7, 8)
Cc <- c(9, 10)
Co <- c(11, 12, 13)

list_clashes <- list(
  c(Ac, Ao),
  c(Bc, Bo, Co),
  c(Bc, Bo, Ac),
  c(Bo, Cc, Co)
)
\end{Rin-no-test}

Then, the functions defined above are used to create the objective function and
the 3 elements of the constraints:

\begin{Rin-no-test}
constraints <- write_constraints(list_clashes = list_clashes,
                                 n_days = n_days,
                                 n_modules = n_modules)
f.con <- constraints[[1]]
f.dir <- constraints[[2]]
f.rhs <- constraints[[3]]
f.obj <- write_objective(n_modules = n_modules, n_days = n_days)
\end{Rin-no-test}

Finally, once these objects are in place, the
\mintinline{R}{ROI} library is used to construct an optimisation problem object:

\begin{Rin-no-test}
library(ROI)

milp <- OP(objective = L_objective(f.obj),
           constraints = L_constraint(L = f.con,
                                      dir = f.dir,
                                      rhs = f.rhs),
           types = rep("B", length(f.obj)),
           maximum = FALSE)
\end{Rin-no-test}

This creates an \mintinline{R}{OP} object from our objective row
\mintinline{R}{f.obj}, and our constraints which are made up from the three
components \mintinline{R}{f.con}, \mintinline{R}{f.dir} and
\mintinline{R}{f.rhs}.
When creating this object the \mintinline{R}{types} as binary
variables are indicated (an array of \mintinline{R}{"B"} for each decision
variable).
The objective function is to be minimised so
\mintinline{R}{maximum = FALSE} is used.

Now to solve:

\begin{Rin-no-test}
sol <- ROI_solve(milp)
\end{Rin-no-test}

The solver will output information about the solve process and runtime.

\begin{Rin-no-test}
print(sol$solution)
\end{Rin-no-test}

\begin{Rout-no-test}
  [1] 0 0 0 0 0 0 0 0 0 0 0 0 0 0 0 0 0 1 0 0 0 0 0 0 1 0 0 0 0
 [30] 0 0 0 0 0 0 0 0 0 0 0 0 0 0 0 0 0 0 0 0 0 0 0 0 0 0 0 0 0
 [59] 0 0 0 0 0 0 0 0 0 0 0 0 1 0 0 0 0 0 0 0 0 0 0 1 0 0 0 0 0
 [88] 0 0 0 0 0 0 0 0 0 0 0 0 0 0 0 0 0 1 0 0 0 0 0 0 0 0 0 0 0
[117] 0 0 0 0 0 0 0 0 0 0 0 0 0 0 0 0 0 1 0 0 0 0 0 0 0 0 1 0 0
[146] 0 0 0 0 0 0 0 1 0 0 1 0 0 0 1 0 0 1 0 0 0 0 1 0 0 0 0 0 0
[175] 0 0 0 0 0 0 0 0 0 0 0 0 1 0 0 0 0 1 0 0 0 0 0 1 0 0 0 1 0
[204] 1 0 1 1 1 0 1
\end{Rout-no-test}

This gives the values of each of the \(Z\) decision variables.
We know the structure of this, that is the first 14 variables are the modules
scheduled for day 1, and so on.
The following code prints a readable schedule:

\begin{Rin-no-test}
#' Gives a human readable schedule corresponding to the
#' solution of a given timetable problem.
#'
#' @param sol: a solution to the timetabling problem
#' @param n_modules: the number of modules to schedule
#' @param n_days: the maximum number of days to schedule
#'
#' @return A string with the schedule
get_schedule <- function(sol, n_days, n_modules){
    schedule = ""
    for (day in 1:n_days){
      if (sol$solution[(n_days * n_modules) + day] == 1){
        schedule <- paste(schedule, "\n", "Day", day, ":")
        for (module in 1:n_modules){
          var <- ((day - 1) * n_modules) + module
          if (sol$solution[var] == 1){
            schedule <- paste(schedule, module)
          }
        }
      }
    }
    schedule
}
\end{Rin-no-test}

Thus:

\begin{Rin-no-test}
schedule <- get_schedule(
    sol = sol,
    n_days = n_days,
    n_modules = n_modules
    )
cat(schedule)
\end{Rin-no-test}

gives:

\begin{Rout-no-test}

 "Day 2 : 4 11"
 "Day 6 : 1 12"
 "Day 8 : 7"
 "Day 10 : 8"
 "Day 11 : 3 13"
 "Day 12 : 2 6 9 14"
 "Day 14 : 5 10"
\end{Rout-no-test}

This gives that 7 days are the minimum required to schedule the 14 exams without
clashes, with either 1, 2 or 4 exams scheduled on each day.

\section{Wider context}\label{sec:lp_wider_context}

The overview given here on linear programming covers a wide breath of the subject
although not much depth. For specific algorithmic approaches to the underlying
algorithms and problem types, such as branch and bound and cutting plane methods
as well as some minor extensions see~\parencite{conforti2014integer,
sultan2014linear}.

The efficiency of a linear programme as well as the ability to model
linear situations imply that it is often used for a variety of applications.
Theatre scheduling as one such application is given
in~\parencite{guerriero2011operational}.
However, scheduling is a wide ranging sub application of linear programming
which can also be used to schedule sport
seasons~\parencite{duran2007scheduling}.

Other applications include the transportation problem~\parencite{diaz2014survey}
which can be used to find a best allocation of a fleet of delivery vehivles;
fire station location problem~\parencite{schreuder1981application} used to
minimise travel times to at-risk areas; and the bin packing
problem~\parencite{hifi2010linear} in which a number of, possibly irregular,
shapes are packed into the smallest possible number of bins.
