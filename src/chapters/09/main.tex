\chapter[Heuristics]{Heuristics}

% Introduction
\chapterinitial{I}{t} is often necessary to find the most desirable choice from
a large, or indeed, infinite set of options. Sometimes this can be done using
exact techniques but often this is not possible and we finding an almost perfect
choice quickly is just as good. This is where the field of heuristics comes in
to play.

\section{Problem}\label{sec:problem}

Consider a delivery company that needs to find itineraries for its 3 drivers. In
the past, the management team has noticed that drivers will often drive to
whichever next stop is closest but this often makes for longer deliveries.

The stops of each driver are represented in Figure~\ref{fig:tsp}.

\begin{figure}
\begin{center}
\includegraphics[width=.8\textwidth]{./assets/tsp/main.pdf}
\end{center}
\caption{Diagrammatic representation of the action sets and payoff matrices for
    the game.}
\label{fig:tsp}
\end{figure}

The distance matrices are given in equations (\ref{eqn:tsp-0}-\ref{eqn:tsp-2}).

\begin{landscape}
    \footnotesize{
        \chapter[Introduction]{Introduction}

\chapterinitial{T}{hank} you for starting to read this book. This book aims to
bring together two fascinating topics:

\begin{itemize}
\item
  Problems that can be solved using mathematics;
\item
  Software that is free to use and change.
\end{itemize}

What we mean by both of those things will become clear through reading
this chapter and the rest of the book.

\section{Who is this book for?}\label{sec:who-is-this-book-for}

Anyone who is interested in using mathematics and computers to solve
problems will hopefully find this book helpful.

%TODO Include VENN Diagram

If you are a student of a mathematical discipline, a graduate student of
a subject like operational research, a hobbyist who enjoys solving the
travelling salesman problem or even if you get paid to do this stuff:
this book is for you. We will introduce you to the world of open source
software that allows you to do all these things freely.

If you are a student learning to write code, a graduate student using
databases for their research, an enthusiast who programmes applications
to help coordinate the neighbourhood watch, or even if you get paid to
write software: this book is for you. We will introduce you to a world
of problems that can be solved using your skill sets.

It would be helpful for the reader of this book to:

\begin{itemize}
\item
  Have access to a computer and be able to connect to the internet (at
  least once) to be able to download the relevant software.
\item
  Be prepared to read some mathematics. Technically you do not need to
  understand the specific mathematics to be able to use the tools in this book.
  The topics covered use some algebra, calculus and probability.
\end{itemize}

\section{What do we mean by applied mathematics?}\label{sec:what-do-we-mean-by-applied-mathematics}

We consider this book to be a book on applied mathematics. This is not
however a universal term, for some applied mathematics is the study of
mechanics and involves modelling projectiles being fired out of canons.
We will use the term a bit more freely here and mean any type of real
world problem that can be tackled using mathematical tools. This is
sometimes referred to as operational research, operations research,
mathematical modelling or indeed just mathematics.

One of the authors, Vince, used mathematics to plan the sitting plan at
his wedding. Using a particular area of mathematics call graph theory he
was able to ensure that everyone sat next to someone they liked and/or
knew.

The other author, Geraint, used mathematics to find the best team of
Pokemon. Using an area of mathematics call linear programming which is
based on linear algebra he was able to find the best makeup of pokemon.

Here, applied mathematics is the type of mathematics that helps us
answer questions that the real world asks.

\section{What is open source
software}\label{sec:what-is-open-source-software}

Strictly speaking open source software is software with source code that
anyone can read, modify and improve. In practice this means that you do
not need to pay to use it which is often one of the first attractions.
This financial aspect can also be one of the reasons that someone will
not use a particular piece of software due to a confusion between cost
and value: if something is free is it really going to be any good?

In practice open source software is used all of the world and powers
some of the most important infrastructure around. For example, one
should never use any cryptographic software that is not open source: if
you cannot open up and read things than you should not trust it (this is
indeed why most cryptographic systems used are open source).
% TODO Add a reference to back up this claim.

Today, open source software is a lot more than a licensing agreement:
it is a community of practice. Bugs are fixed faster, research is
implemented immediately and knowledge is spread more widely thanks to
open source software. Bugs are fixed faster because anyone can read and
inspect the source code. Most open source software projects also have a
clear mechanisms for communicating with the developers and even
reviewing and accepting code contributions from the general public.
Research is implemented immediately because when new algorithms are
discovered they are often added directly to the software by the
researchers who found them. This all contributes to the spread of
knowledge: open source software is the modern should of giants that we
all stand on.

Open source software is software that, like scientific knowledge is not
restricted in its use.

\section{How to get the most out of this
book}\label{sec:how-to-get-the-most-out-of-this-book}

The book itself is open source. You can find the source files for this
book online at \texttt{github.com/drvinceknight/ampwoss}. There will will
also find a number of \emph{Jupyter notebooks} and \emph{R markdown
files} that include code snippets that let you follow along.

We feel that you can choose to read the book from cover to cover,
writing out the code examples as you go; or it could also be used as a
reference text when faced with particular problem and wanting to know
where to start.

The book is made up of 10 chapters that are paired in two 4 parts. Each
part corresponds to a particular area of mathematics, for example
``Emergent Behaviour''. Two chapters are paired together for each
chapter, usually these two chapters correspond to the same area of
mathematics but from a slightly different scale that correspond to
different ways of tackling the problem.

Every chapter has the following structure:

\begin{enumerate}
\item
  Introduction - a brief overview of a given problem type. Here we will
  describe the problem at hand in general terms.
\item
  An Example problem. This will provide a tangible example problem that
  offers the reader some intuition for the rest of the discussion.
\item
  Solving with Python. We will describe the mathematical tools available
  to us in a programming language called Python to solve the problem.
\item
  Solving with R. Here we will do the same with the R programming
  language.
\item
  Brief theoretic background with pointers to reference texts. Some
  readers might like to delve in to the mathematics of the problem a bit
  further, we will include those details here.
\item
  Examples of research using these methods. Finally, some readers might
  even be interested in finding out a bit more of what mathematicians
  are doing on these problems. Often this will include some descriptions
  of the problem considered but perhaps at a much larger scale than the
  one presented in the example.
\end{enumerate}

For a given reader, not all sections of a chapter will be of interest.
Perhaps a reader is only interested in R and finding out more about the
research. Please do take from the book what you find useful.

    }
\end{landscape}

The value \(d^{(k)}_{ij}\) gives the travel distance for itinerary \(k\) between
stops \(i\) and \(j\). For example, \(d^{2}_{23}=103.3\) % TODO If the distance matrix changes, this value need to be updated
indicates that the distance between the 2nd and 3rd stop in the third itinerary
is given 103.3. % TODO If the distance matrix changes, this value need to be updated

Given these parameters, we aim to find a \emph{sufficiently good} set of
itineraries that gives a low total amount of travel.

The emphasis on needing a good solution, but not necessarily the best one,
prioritising computational efficiency is where the field of heuristics comes in
to its own.

\section{Theory}\label{sec:theory}

The heuristic approach take here will be to use a neighborhood search algorithm.
This algorithm works by considering a given potential solution, evaluating it
and then trying another potential solution \emph{close} to it. What \emph{close}
means depends on different approaches and problems: it is referred to as the
neighbourhood. As a new solution is evaluated if it is \emph{good} (this is
again a term that depends on the approach and problem) then the search
continues from the neighbourhood of this new solution.

For our problem,
the first aspect of this is to represent a given trajectory between all the
potential stops as a \emph{tour}. If we have 3 total stops and require that the
tour starts and stops at the first one then there are two possible tours:

\[
    t \in \{(1, 2, 3, 1), (1, 3, 2, 1)\}
\]

Given a distance matrix \(d\) such that \(d_{ij}\) is the distance between stop
\(i\) and \(j\) the total cost of a tour is given by:

\[
    C(t)=...  % TODO Add cost function
\]

% TODO Add example

Using this framework, the neighbourhood search can be written down as:

\begin{enumerate}
    \item Start with a given tour: \(t\).
    \item Evaluate \(C(t)\).
    \item Identify a new \(\tilde t\) from \(t\) and accept it if \(C(\tilde t)<
        C(t)\).
\end{enumerate}

This is shown diagrammatically in Figure ... % TODO Draw figure.
\section{Solving with Python}\label{sec:solving-with-python}

\section{Solving with R}\label{sec:solving-with-R}

\section{Research}\label{sec:research}

TBA
