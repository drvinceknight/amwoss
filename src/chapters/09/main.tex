\chapter[Heuristics]{Heuristics}\label{chp:heuristics}


% Introduction
\chapterinitial{I}{t} is often necessary to find the most desirable choice from
a large, or indeed, infinite set of options. Sometimes this can be done using
exact techniques but often this is not possible and finding an almost perfect
choice quickly is just as good. This is where the field of
heuristics\index{heuristics} comes in to play.

\section{Problem}\label{sec:heuristics_problem}

A delivery company needs to deliver goods to 13 different stops.
They need to find a route for a driver that stops at each of the stops
once only, then returns to the first stop, the depot.

The stops are drawn in Figure~\ref{fig:tsp}.

\begin{figure}
    \begin{center}
        \includegraphics[width=.8\textwidth]{./assets/tsp/main.pdf}
    \end{center}
    \caption{The positions of the required stops.}
    \label{fig:tsp}
\end{figure}

The relevant information is the pairwise distances between each of the stops,
which is given by the distance matrix\index{matrix} in equation (\ref{eqn:tsp}).

\tiny{
    \chapter[Introduction]{Introduction}

\chapterinitial{T}{hank} you for starting to read this book. This book aims to
bring together two fascinating topics:

\begin{itemize}
\item
  Problems that can be solved using mathematics;
\item
  Software that is free to use and change.
\end{itemize}

What we mean by both of those things will become clear through reading
this chapter and the rest of the book.

\section{Who is this book for?}\label{sec:who-is-this-book-for}

Anyone who is interested in using mathematics and computers to solve
problems will hopefully find this book helpful.

%TODO Include VENN Diagram

If you are a student of a mathematical discipline, a graduate student of
a subject like operational research, a hobbyist who enjoys solving the
travelling salesman problem or even if you get paid to do this stuff:
this book is for you. We will introduce you to the world of open source
software that allows you to do all these things freely.

If you are a student learning to write code, a graduate student using
databases for their research, an enthusiast who programmes applications
to help coordinate the neighbourhood watch, or even if you get paid to
write software: this book is for you. We will introduce you to a world
of problems that can be solved using your skill sets.

It would be helpful for the reader of this book to:

\begin{itemize}
\item
  Have access to a computer and be able to connect to the internet (at
  least once) to be able to download the relevant software.
\item
  Be prepared to read some mathematics. Technically you do not need to
  understand the specific mathematics to be able to use the tools in this book.
  The topics covered use some algebra, calculus and probability.
\end{itemize}

\section{What do we mean by applied mathematics?}\label{sec:what-do-we-mean-by-applied-mathematics}

We consider this book to be a book on applied mathematics. This is not
however a universal term, for some applied mathematics is the study of
mechanics and involves modelling projectiles being fired out of canons.
We will use the term a bit more freely here and mean any type of real
world problem that can be tackled using mathematical tools. This is
sometimes referred to as operational research, operations research,
mathematical modelling or indeed just mathematics.

One of the authors, Vince, used mathematics to plan the sitting plan at
his wedding. Using a particular area of mathematics call graph theory he
was able to ensure that everyone sat next to someone they liked and/or
knew.

The other author, Geraint, used mathematics to find the best team of
Pokemon. Using an area of mathematics call linear programming which is
based on linear algebra he was able to find the best makeup of pokemon.

Here, applied mathematics is the type of mathematics that helps us
answer questions that the real world asks.

\section{What is open source
software}\label{sec:what-is-open-source-software}

Strictly speaking open source software is software with source code that
anyone can read, modify and improve. In practice this means that you do
not need to pay to use it which is often one of the first attractions.
This financial aspect can also be one of the reasons that someone will
not use a particular piece of software due to a confusion between cost
and value: if something is free is it really going to be any good?

In practice open source software is used all of the world and powers
some of the most important infrastructure around. For example, one
should never use any cryptographic software that is not open source: if
you cannot open up and read things than you should not trust it (this is
indeed why most cryptographic systems used are open source).
% TODO Add a reference to back up this claim.

Today, open source software is a lot more than a licensing agreement:
it is a community of practice. Bugs are fixed faster, research is
implemented immediately and knowledge is spread more widely thanks to
open source software. Bugs are fixed faster because anyone can read and
inspect the source code. Most open source software projects also have a
clear mechanisms for communicating with the developers and even
reviewing and accepting code contributions from the general public.
Research is implemented immediately because when new algorithms are
discovered they are often added directly to the software by the
researchers who found them. This all contributes to the spread of
knowledge: open source software is the modern should of giants that we
all stand on.

Open source software is software that, like scientific knowledge is not
restricted in its use.

\section{How to get the most out of this
book}\label{sec:how-to-get-the-most-out-of-this-book}

The book itself is open source. You can find the source files for this
book online at \texttt{github.com/drvinceknight/ampwoss}. There will will
also find a number of \emph{Jupyter notebooks} and \emph{R markdown
files} that include code snippets that let you follow along.

We feel that you can choose to read the book from cover to cover,
writing out the code examples as you go; or it could also be used as a
reference text when faced with particular problem and wanting to know
where to start.

The book is made up of 10 chapters that are paired in two 4 parts. Each
part corresponds to a particular area of mathematics, for example
``Emergent Behaviour''. Two chapters are paired together for each
chapter, usually these two chapters correspond to the same area of
mathematics but from a slightly different scale that correspond to
different ways of tackling the problem.

Every chapter has the following structure:

\begin{enumerate}
\item
  Introduction - a brief overview of a given problem type. Here we will
  describe the problem at hand in general terms.
\item
  An Example problem. This will provide a tangible example problem that
  offers the reader some intuition for the rest of the discussion.
\item
  Solving with Python. We will describe the mathematical tools available
  to us in a programming language called Python to solve the problem.
\item
  Solving with R. Here we will do the same with the R programming
  language.
\item
  Brief theoretic background with pointers to reference texts. Some
  readers might like to delve in to the mathematics of the problem a bit
  further, we will include those details here.
\item
  Examples of research using these methods. Finally, some readers might
  even be interested in finding out a bit more of what mathematicians
  are doing on these problems. Often this will include some descriptions
  of the problem considered but perhaps at a much larger scale than the
  one presented in the example.
\end{enumerate}

For a given reader, not all sections of a chapter will be of interest.
Perhaps a reader is only interested in R and finding out more about the
research. Please do take from the book what you find useful.

}
\normalsize

The value \(d_{ij}\) gives the travel distance between
stops \(i\) and \(j\). For example, \(d_{23}=67\)
indicates that the distance between the 2nd and 3rd stop in the route is 67.


The delivery company would like to find the route around the 13 stops that gives
the smallest overall travel distance.


\section{Theory}\label{sec:heuristics_theory}

This problem is called a travelling salesman
problem\index{travelling salesman problem}, which can often be inefficient to
solve using exact methods~\cite{michalewicz2013solve}.
Heuristics are a family of methods that can be used to find a find a
\emph{sufficiently good} solution, though not necessarily the optimal solution,
where the emphasis is on prioritising computational efficiency.

The heuristic approach taken here will be to use a neighbourhood search algorithm.
This algorithm works by considering a given potential solution, evaluating it
and then trying another potential solution \emph{close} to it. What \emph{close}
means depends on different approaches and problems: it is referred to as the
neighbourhood. When a new solution is considered \emph{good}\footnote{`Good' is
again a term that depends on the approach and problem.} then the search
continues from the neighbourhood of this new solution.

For this problem, the steps are to first represent a possible solution, that is
a given route between all the potential stops as a \emph{tour}. If there are 3
total stops the tour must start and stop at the first one then there are two
possible tours:

\[
    t \in \{(1, 2, 3, 1), (1, 3, 2, 1)\}
\]

Given a distance matrix \(d\) such that \(d_{ij}\) is the distance between stop
\(i\) and \(j\) the total cost of a tour is given by:

\[
    C(t)=\sum_{i=1}^{n} d_{t_i, t_{i + 1}}
\]

Thus, with:

\[
    d = \begin{pmatrix}
        0 & 1 & 3\\
        1 & 0 & 15\\
        3 & 3 & 7
        \end{pmatrix}
\]

We have:

\begin{eqnarray*}
    C((1, 2, 3, 1)) &= d_{12} + d_{23} + d_{31} = 1 + 15 + 3 = 19\\
    C((1, 3, 2, 1)) &= d_{13} + d_{32} + d_{21} = 3 + 3 + 1 = 7
\end{eqnarray*}

In general, the neighbourhood search can be written down as:

\begin{enumerate}
    \item Start with a given tour: \(t\).
    \item Evaluate \(C(t)\).
    \item Identify a new \(\tilde t\) from \(t\) and accept it as a replacement
        for \(t\) if \(C(\tilde t)< C(t)\).
    \item Repeat the 3rd step until some stopping condition is met.
\end{enumerate}

This is shown diagrammatically in
Figure~\ref{fig:neighbourhood_search_flow_diagram}.

\begin{figure}[!hbtp]
    \begin{center}
        \includestandalone[width=.7\textwidth]{./assets/neighbourhood_search_flow_diagram/main}
    \end{center}
    \caption{The general neighbourhood search algorithm. \(N(t)\) refers to some
    neighbourhood of \(t\).}
    \label{fig:neighbourhood_search_flow_diagram}
\end{figure}

A number of stopping conditions can be used including some specific
overall cost or a number of total iterations of the algorithm.

The neighbourhood of a tour \(t\) is taken as some set of tours that can be
obtained from \(t\) using a specific and computationally efficient
\textbf{neighbourhood operator}.
To illustrate two such neighbourhoods operators, consider the following tour on
7 stops:

\[
    t = (0, 1, 2, 3, 4, 5, 6, 0)
\]

One possible neighbourhood is to choose 2 stops at random and swap. For
example, the tour \(\tilde t^{(1)}\in N(t)\) is obtained by swapping the stop
labelled 2 and the stop labelled 5.

\[
    \tilde t^{(1)} = (0, 1, 5, 3, 4, 2, 6, 0)
\]

Another possible neighbourhood is to choose 2 stops at random and reverse the
order of all stops between (including) those two stops. For example, the tour
\(\tilde t^{(2)} \in N(t)\) is obtained by reversing the order of all stops between
the stop labelled 2 and the stop labelled 5.

\[
    \tilde t^{(2)} = (0, 1, 5, 4, 3, 2, 6, 0)
\]

Examples of these tours are shown in
Figure~\ref{fig:tsp-effect-of-neighbourhood-operators}.

\begin{figure}[!hbtp]
    \begin{center}
        \includegraphics[width=0.8\textwidth]{./assets/tsp-effect-of-neighbourhood-operators/main.pdf}
    \end{center}
    \caption{The effect of two neighbourhood operators on \(t\). \(\tilde t^{(1)}\) is
    obtained by swapping stops labelled 2 and 5. \(\tilde t^{(2)}\) is obtained by reversing the
    path between stops labelled 2 and 5.}
    \label{fig:tsp-effect-of-neighbourhood-operators}
\end{figure}

\section{Solving with Python}\label{sec:heuristics_solving-with-python}

To solve this problem using Python, functions will be written that match the
first three steps in the Section~\ref{sec:heuristics_theory}.
The first step is to write the \mintinline{python}{get_initial_candidate}
function that creates an initial tour.

\begin{pyin}
import numpy as np


def get_initial_candidate(number_of_stops, seed):
    """Return an random initial tour.

    Args:
        number_of_stops: The number of stops
        seed: An integer seed.

    Returns:
        A tour starting an ending at stop with index 0.
    """
    internal_stops = list(range(1, number_of_stops))
    np.random.seed(seed)
    np.random.shuffle(internal_stops)
    return [0] + internal_stops + [0]
\end{pyin}

This gives a random tour on 13 stops:

\begin{pyin}
number_of_stops = 13
seed = 0
initial_candidate = get_initial_candidate(
    number_of_stops=number_of_stops,
    seed=seed,
)
print(initial_candidate)
\end{pyin}

\begin{pyout}
[0, 7, 12, 5, 11, 3, 9, 2, 8, 10, 4, 1, 6, 0]
\end{pyout}

To be able to evaluate any given tour its cost must be found. Here
\mintinline{python}{get_cost} does this:

\begin{pyin}
def get_cost(tour, distance_matrix):
    """Return the cost of a tour.

    Args:
        tour: A given tuple of successive stops.
        distance_matrix: The distance matrix of the problem.

    Returns:
        The cost
    """
    return sum(
        distance_matrix[current_stop, next_stop]
        for current_stop, next_stop in zip(tour[:-1], tour[1:])
    )
\end{pyin}

\begin{pyin}
distance_matrix = np.array(
    (
        (0, 35, 35, 29, 70, 35, 42, 27, 24, 44, 58, 71, 69),
        (35, 0, 67, 32, 72, 40, 71, 56, 36, 11, 66, 70, 37),
        (35, 67, 0, 63, 64, 68, 11, 12, 56, 77, 48, 67, 94),
        (29, 32, 63, 0, 93, 8, 71, 56, 8, 33, 84, 93, 69),
        (70, 72, 64, 93, 0, 101, 56, 56, 92, 81, 16, 5, 69),
        (35, 40, 68, 8, 101, 0, 76, 62, 11, 39, 91, 101, 76),
        (42, 71, 11, 71, 56, 76, 0, 15, 65, 81, 40, 60, 94),
        (27, 56, 12, 56, 56, 62, 15, 0, 50, 66, 41, 58, 82),
        (24, 36, 56, 8, 92, 11, 65, 50, 0, 39, 81, 91, 74),
        (44, 11, 77, 33, 81, 39, 81, 66, 39, 0, 77, 79, 37),
        (58, 66, 48, 84, 16, 91, 40, 41, 81, 77, 0, 20, 73),
        (71, 70, 67, 93, 5, 101, 60, 58, 91, 79, 20, 0, 65),
        (69, 37, 94, 69, 69, 76, 94, 82, 74, 37, 73, 65, 0),
    )
)
cost = get_cost(
    tour=initial_candidate,
    distance_matrix=distance_matrix,
)
print(cost)
\end{pyin}

\begin{pyout}
827
\end{pyout}

Now a function for neighbourhood operator will be written,
\mintinline{python}{swap_stops}, that swaps two stops in a given tour.

\begin{pyin}
def swap_stops(tour):
    """Return a new tour by swapping two stops.

    Args:
        tour: A given tuple of successive stops.

    Returns:
        A tour
    """
    number_of_stops = len(tour) - 1
    i, j = np.random.choice(range(1, number_of_stops), 2)
    new_tour = list(tour)
    new_tour[i], new_tour[j] = tour[j], tour[i]
    return new_tour
\end{pyin}

Applying this neighbourhood operator to the initial candidate gives:

\begin{pyin}
print(swap_stops(initial_candidate))
\end{pyin}

which swaps the 10th and 12th stops:

\begin{pyout}
[0, 7, 12, 5, 11, 3, 9, 2, 8, 1, 4, 10, 6, 0]
\end{pyout}

Now all the tools are in place to build a tool to carry out the
neighbourhood search \mintinline{python}{run_neighbourhood_search}.

\begin{pyin}
def run_neighbourhood_search(
    distance_matrix,
    iterations,
    seed,
    neighbourhood_operator=swap_stops,
):
    """Returns a tour by carrying out a neighbourhood search.

    Args:
        distance_matrix: the distance matrix
        iterations: the number of iterations for which to
                    run the algorithm
        seed: a random seed
        neighbourhood_operator: the neighbourhood operator
                                (default: swap_stops)

    Returns:
        A tour
    """
    number_of_stops = len(distance_matrix)
    candidate = get_initial_candidate(
        number_of_stops=number_of_stops,
        seed=seed,
    )
    best_cost = get_cost(
        tour=candidate,
        distance_matrix=distance_matrix,
    )
    for _ in range(iterations):
        new_candidate = neighbourhood_operator(candidate)
        cost = get_cost(
            tour=new_candidate,
            distance_matrix=distance_matrix,
        )
        if cost <= best_cost:
            best_cost = cost
            candidate = new_candidate

    return candidate
\end{pyin}

Now running this for 1000 iterations:

\begin{pyin}
number_of_iterations = 1000

solution_with_swap_stops = run_neighbourhood_search(
    distance_matrix=distance_matrix,
    iterations=number_of_iterations,
    seed=seed,
    neighbourhood_operator=swap_stops,
)
print(solution_with_swap_stops)
\end{pyin}

gives:

\begin{pyout}
[0, 7, 2, 8, 5, 3, 1, 9, 12, 11, 4, 10, 6, 0]
\end{pyout}

This has a cost:

\begin{pyin}
cost = get_cost(
    tour=solution_with_swap_stops,
    distance_matrix=distance_matrix,
)
print(cost)
\end{pyin}

\begin{pyout}
362
\end{pyout}

Therefore, using this particular algorithm, a pretty good route is found, with a
total distance of 362.

It is important to note that this may not be the optimal route, and different
algorithms may produce better solutions.
For example, one way to modify the algorithm is to use a different neighbourhood
operator.
Instead of swapping two stops, reverse the path between those two stops. This
corresponds to an algorithm called the ``2-opt''
algorithm\index{2-opt algorithm}.\footnote{The 2-opt algorithm was first
published in~\cite{croes1958method}.}
The \mintinline{python}{reverse_path} function does this:

\begin{pyin}
def reverse_path(tour):
    """Return a new tour by reversing the path between two stops.

    Args:
        tour: A given tuple of successive stops.

    Returns:
        A tour
    """
    number_of_stops = len(tour) - 1
    stops = np.random.choice(range(1, number_of_stops), 2)
    i, j = sorted(stops)
    new_tour = tour[:i] + tour[i : j + 1][::-1] + tour[j + 1 :]
    return new_tour
\end{pyin}

Applying this neighbourhood operator to the initial candidate gives:

\begin{pyin}
print(reverse_path(initial_candidate))
\end{pyin}

which reverses the order between the 3rd and the 11th stop:

\begin{pyout}
[0, 7, 4, 10, 8, 2, 9, 3, 11, 5, 12, 1, 6, 0]
\end{pyout}

Now running the neighbourhood search for 1000 iterations using the
\mintinline{python}{reverse_path} neighbourhood operator:

\begin{pyin}
solution_with_reverse_path = run_neighbourhood_search(
    distance_matrix=distance_matrix,
    iterations=number_of_iterations,
    seed=seed,
    neighbourhood_operator=reverse_path,
)
print(solution_with_reverse_path)
\end{pyin}

gives:

\begin{pyout}
[0, 8, 5, 3, 1, 9, 12, 11, 4, 10, 6, 2, 7, 0]
\end{pyout}

This now gives a different route.
Importantly, the costs differ substantially:

\begin{pyin}
cost = get_cost(
    tour=solution_with_reverse_path,
    distance_matrix=distance_matrix,
)
print(cost)
\end{pyin}

which gives:

\begin{pyout}
299
\end{pyout}

This improves on the solution found using the \mintinline{python}{swap_stops}
operator. Figure~\ref{fig:final-tsp-tours-python} shows the final obtained
routes given by both approaches.

\begin{figure}
    \begin{center}
        \includegraphics[width=\textwidth]{./assets/final-tsp-tours-with-python/main.pdf}
    \end{center}
    \caption{The final tours obtained by using the neighbourhood search in
    Python.}
    \label{fig:final-tsp-tours-python}
\end{figure}



\section{Solving with R}\label{sec:heuristics_solving-with-R}

To solve this problem using R, functions will be written that match the
first three steps in the Section~\ref{sec:heuristics_theory}.

The first step is to write the \mintinline{R}{get_initial_candidate}
function that creates an initial tour:

\begin{Rin}
#' Return an random initial tour.
#'
#' @param number_of_stops The number of stops.
#' @param seed An integer seed.
#'
#' @return A tour starting an ending at stop with index 0.
get_initial_candidate <- function(number_of_stops, seed){
  internal_stops <- 1:(number_of_stops - 1)
  set.seed(seed)
  internal_stops <- sample(internal_stops)
  c(0, internal_stops, 0)
}
\end{Rin}

This gives a random tour on 13 stops:

\begin{Rin}
number_of_stops <- 13
seed <- 1
initial_candidate <- get_initial_candidate(
  number_of_stops = number_of_stops,
  seed = seed)
print(initial_candidate)
\end{Rin}

\begin{Rout}
 [1]  0  9  4  7  1  2  5  3  8  6 11 12 10  0
\end{Rout}

To be able to evaluate any given tour its cost must be found. Here
\mintinline{R}{get_cost}  does this:

\begin{Rin}
#' Return the cost of a tour
#'
#' @param tour A given vector of successive stops.
#' @param seed The distance matrix of the problem.
#'
#' @return The cost
get_cost <- function(tour, distance_matrix){
  pairs <-  cbind(tour[-length(tour)], tour[-1]) + 1
  sum(distance_matrix[pairs])
}
\end{Rin}

\begin{Rin}
distance_matrix <- rbind(
        c(0, 35, 35, 29, 70, 35, 42, 27, 24, 44, 58, 71, 69),
        c(35, 0, 67, 32, 72, 40, 71, 56, 36, 11, 66, 70, 37),
        c(35, 67, 0, 63, 64, 68, 11, 12, 56, 77, 48, 67, 94),
        c(29, 32, 63, 0, 93, 8, 71, 56, 8, 33, 84, 93, 69),
        c(70, 72, 64, 93, 0, 101, 56, 56, 92, 81, 16, 5, 69),
        c(35, 40, 68, 8, 101, 0, 76, 62, 11, 39, 91, 101, 76),
        c(42, 71, 11, 71, 56, 76, 0, 15, 65, 81, 40, 60, 94),
        c(27, 56, 12, 56, 56, 62, 15, 0, 50, 66, 41, 58, 82),
        c(24, 36, 56, 8, 92, 11, 65, 50, 0, 39, 81, 91, 74),
        c(44, 11, 77, 33, 81, 39, 81, 66, 39, 0, 77, 79, 37),
        c(58, 66, 48, 84, 16, 91, 40, 41, 81, 77, 0, 20, 73),
        c(71, 70, 67, 93, 5, 101, 60, 58, 91, 79, 20, 0, 65),
        c(69, 37, 94, 69, 69, 76, 94, 82, 74, 37, 73, 65, 0)
)
cost <- get_cost(
  tour = initial_candidate,
  distance_matrix = distance_matrix)
print(cost)
\end{Rin}

\begin{Rout}
[1] 709
\end{Rout}

Now a function for a neighbourhood operator will be written,
\mintinline{R}{swap_stops}: swapping two stops in a given tour.

\begin{Rin}
#' Return a new tour by swapping two stops.
#'
#' @param tour A given vector of successive stops.
#'
#' @return A tour
swap_stops <- function(tour){
  number_of_stops <- length(tour) - 1
  stops_to_swap <- sample(2:number_of_stops, 2)
  new_tour <- replace(
    x = tour,
    list = stops_to_swap,
    values = rev(tour[stops_to_swap])
  )
}
\end{Rin}

Applying this neighbourhood operator to the initial candidate gives:

\begin{Rin}
new_tour <- swap_stops(initial_candidate)
print(new_tour)
\end{Rin}

which swaps the 6th and 11th stops:

\begin{Rout}
 [1]  0  9  4  7  1 11  5  3  8  6  2 12 10  0
\end{Rout}

Now all the tools are in place to build a tool to carry out the
neighbourhood search \mintinline{R}{run_neighbourhood_search}.

\begin{Rin}
#' Returns a tour by carrying out a neighbourhood search
#'
#' @param distance_matrix: the distance matrix
#' @param iterations: the number of iterations for
#'                    which to run the algorithm
#' @param seed: a random seed (default: None)
#' @param neighbourhood_operator: the neighbourhood operation
#'                                (default: swap_stops)
#'
#' @return A tour
run_neighbourhood_search <- function(
  distance_matrix,
  iterations,
  seed = NA,
  neighbourhood_operator = swap_stops
){
  number_of_stops <- nrow(distance_matrix)
  candidate <- get_initial_candidate(
    number_of_stops = number_of_stops,
    seed = seed
  )
  best_cost <- get_cost(
    tour = candidate,
    distance_matrix = distance_matrix
  )
  for (repetition in 1:iterations) {
    new_candidate <- neighbourhood_operator(candidate)
    cost <- get_cost(
        tour = new_candidate,
        distance_matrix = distance_matrix
    )
    if (cost <= best_cost) {
      best_cost <- cost
      candidate <- new_candidate
    }
  }
  candidate
}
\end{Rin}

Now running this for 1000 iterations:

\begin{Rin}
number_of_iterations <- 1000
solution_with_swap_stops <- run_neighbourhood_search(
  distance_matrix = distance_matrix,
  iterations = number_of_iterations,
  seed = seed,
  neighbourhood_operator = swap_stops
)
print(solution_with_swap_stops)
\end{Rin}

gives:

\begin{Rout}
 [1]  0 11  4 10  6  2  7 12  9  1  3  5  8  0
\end{Rout}

This has a cost:

\begin{Rin}
cost <- get_cost(
  tour = solution_with_swap_stops,
  distance_matrix = distance_matrix
)
print(cost)
\end{Rin}

which gives:

\begin{Rout}
[1] 360
\end{Rout}

Therefore, using this particular algorithm, a pretty good route is found, with a
total distance of 360.

It is important to note that this may not be the optimal route, and different
algorithms may produce better solutions.
For example, one way to modify the algorithm is to use a different neighbourhood
operator.
Instead of swapping two stops, reverse the path between those two stops. This
corresponds to an algorithm called the ``2-opt''
algorithm\index{2-opt algorithm}.\footnote{The 2 opt algorithm was first
published in~\cite{croes1958method}.}
The \mintinline{R}{reverse_path} function does this:


\begin{Rin}
#' Return a new tour by reversing the path between two stops.
#'
#' @param tour A given vector of successive stops.
#'
#' @return A tour
reverse_path <- function(tour){
  number_of_stops <- length(tour) - 1
  stops_to_swap <- sample(2:number_of_stops, 2)
  i <- min(stops_to_swap)
  j <- max(stops_to_swap)
  new_order <- c(c(1: (i - 1)), c(j:i), c( (j + 1): length(tour)))
  tour[new_order]
}
\end{Rin}

Applying this neighbourhood operator to the initial candidate gives:

\begin{Rin}
new_tour <- reverse_path(initial_candidate)
print(new_tour)
\end{Rin}

which reverses the order
between the 3rd and the 13th stop:

\begin{Rout}
 [1]  0  9 10 12 11  6  8  3  5  2  1  7  4  0
\end{Rout}

Now running the neighbourhood search for 1000 iterations using the
\mintinline{R}{reverse_path} neighbourhood operator:

\begin{Rin}
number_of_iterations <- 1000
solution_with_reverse_path <- run_neighbourhood_search(
  distance_matrix = distance_matrix,
  iterations = number_of_iterations,
  seed = seed,
  neighbourhood_operator = reverse_path
)
print(solution_with_reverse_path)
\end{Rin}

gives:

\begin{Rout}
 [1]  0  7  2  6 10  4 11 12  9  1  3  5  8  0
\end{Rout}

This now gives a different route.
Importantly, the costs differ substantially:

\begin{Rin}
cost <- get_cost(
  tour = solution_with_reverse_path,
  distance_matrix = distance_matrix
)
print(cost)
\end{Rin}

which gives:

\begin{Rout}
[1] 299
\end{Rout}

This is an improvement on the solution found using the \mintinline{R}{swap_stops}
operator. Figure~\ref{fig:final-tsp-tours-r} shows the final obtained routes
given by both approaches.


\begin{figure}
    \begin{center}
        \includegraphics[width=\textwidth]{./assets/final-tsp-tours-with-R/main.pdf}
    \end{center}
    \caption{The final tours obtained by using the neighbourhood search in R.}
    \label{fig:final-tsp-tours-r}
\end{figure}


\section{Wider context}\label{sec:heuristics_wider_context}

Heuristic methods, sometimes referred to as meta-heuristics, are a whole family
of algorithms used to find approximate solutions to combinatorial optimisation
problems. An overview is given in~\cite{bozorg2017meta}. These algorithms
include greedy searches, tabu searches\index{tabu search}, simulated
annealing\index{simulated annealing}, genetic algorithms\index{genetic algorithm},
as well \index{ant colony optimisation}. They are usually employed when the
problem is too large or complex to use exact methodologies.

The travelling salesman problem, described in this chapter, is a classic example
of one of these problems, formally described first in~\cite{menger1932},
although thought to have been discussed informally centuries before.
It is an example of a large number of types of problems collectively known as
vehicle routing problems, which often require heuristic methods for their
solutions. A survey is given in~\cite{braekers2016vehicle}. Variations of
the problem include multiple, heterogeneous and/or capacitated vehicles, and
stochastic or time-dependent travel times. A recent adaptation of the problem
is the green vehicle routing
problem\index{green vehicle routing problem}~\cite{moghdani2021green}, where the
cost function includes consideration of green house gas emissions and other
pollutants.

For more diverse applications of heuristic methods,
consider~\cite{lewis2016creating} which describes a tabu search algorithm
for finding seating plans for a wedding; and \cite{tong2013modeling} where
a genetic algorithm is used to build a prediction model for locations of
deep-sea wildlife habitats.
