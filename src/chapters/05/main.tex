\chapter[Modelling with Differential Equations]{Modelling with Differential Equations}

% Introduction
\chapterinitial{S}{ystems} often change in a way that depends on their current
state. For example, the speed at which a cup of coffee cools down depends on its
current temperature. These types of systems are called dynamical systems and are
modelled mathematically using differential equations. In this chapter we will
consider a direct solution approach using symbolic mathematics.

\section{Problem}\label{sec:problem}

Consider the following situation: the entire population of a small rural town
has caught a cold. All 100 individuals will recover at an average rate of 2 per
day.  The town leadership have noticed that being ill costs approximately
\pound10 per day, this is due to general lack of productivity, poorer mood and
other intangible aspects. They need to decide whether or not to order cold
medicine which would \textbf{double} the recover rate. The cost of of the cold
medicine is a one off cost of \pound5 per person. 

\section{Theory}\label{sec:theory}

In the case of this town, the overall rate at which people get better is
dependent on the number of people in how are ill. This can be represented
mathematically using a differential equation which is a way of relating the rate
of change of a system to the state of the system itself.

In general if we are interested in some variable \(x\) over time \(t\) the
differential function equation will be of the form:

\begin{equation}
    \frac{dx}{dt} = f(x)
\end{equation}

For some function \(f\).
In our case,
if we denote the number of infected individuals as \(I\) where we implicitly
mean that \(I\) is a function of time: \(I=I(t)\) and the rate at which
individuals recover by \(\alpha\) then the differential equation
that describes the above situation is:

\begin{equation}
    \frac{dI}{dt} = -\alpha I
\end{equation}

Finding a solution to this differential equation means finding an expression for
\(I\) that when differentiated gives \(- alpha I\).

In this particular case, one such function is:

\begin{equation}
    I(t) = e ^ {-\alpha t}
\end{equation}

However, \(I(0) = 1\) whereas for our problem we know that at time \(t=0\) there
are 100 infected individuals. Indeed a differential equation defines a family of
solutions and we need to know some sort of initial (also referred to as
boundary) condition to have the exact solution. Which in this case would be:

\begin{equation}
    I(t) = 100e ^ {-\alpha t}
\end{equation}

To evaluate the cost we then need to know the sum of the values of that function
over time. Integration gives us exactly this, so the cost would be:

\begin{equation}
    K \int_{0}^{\infty}I(t)dt
\end{equation}

where \(K\) is the cost per person per unit time.

In the upcoming sections we will confirm and use code to carry out the above
efficiently so as to answer the original question.

\section{Solving with Python}\label{sec:solving-with-python}

The first step we will take is to write a function to obtain the differential
equation. Note that here we will be using the Python library
\mintinline{python}{sympy} which allows us to carry out symbolic calculations.

\begin{pyin}
import sympy as sym

t = sym.Symbol("t")
alpha = sym.Symbol("alpha")
I_0 = sym.Symbol("I_0")
I = sym.Function("I")


def get_equation(alpha=alpha):
    """Return the differential equation.

    Args:
        alpha: a float (default: symbolic alpha)

    Returns:
        A symbolic equation
    """
    return sym.Eq(sym.Derivative(I(t), t), -alpha * I(t))
\end{pyin}

Using this we can get the equation that defines the population change over time:

\begin{pyin}
eq = get_equation()
print(eq)
\end{pyin}

which gives:

\begin{pyout}
Eq(Derivative(I(t), t), -alpha*I(t))
\end{pyout}

Note that if you are using Jupyter then your output will actually be a
well rendered mathematical equation:

\[
\frac{d}{d t} I{\left(t \right)} = - \alpha I{\left(t \right)}
\]

Note that we can pass a value to \(\alpha\) if we want to:

\begin{pyin}
eq = get_equation(alpha=1)
print(eq)
\end{pyin}

\begin{pyout}
Eq(Derivative(I(t), t), -I(t))
\end{pyout}

We will now write a function to obtain the solution to this differential

\begin{pyin}
def get_solution(I_0=I_0, alpha=alpha):
    """Return the solution to the differential equation.

    Args:
        I_0: a float (default: symbolic I_0)
        alpha: a float (default: symbolic alpha)

    Returns:
        A symbolic equation
    """
    eq = get_equation(alpha=alpha)
    return sym.dsolve(eq, I(t), ics={I(0): I_0})
\end{pyin}

We can verify the solution discussed previously:

\begin{pyin}
sol = get_solution()
print(sol)
\end{pyin}

which gives:

\begin{pyout}
Eq(I(t), I_0*exp(-alpha*t))
\end{pyout}

\[I(t) = I_0 e ^{-\alpha t}\]

We can use sympy itself to verify the result, by taking the derivative of the
right hand side of our solution.

\begin{pyin}
print(sym.diff(sol.rhs, t) == -alpha * sol.rhs)
\end{pyin}

which gives:

\begin{pyout}
True
\end{pyout}

All of the above has given us the general solution in terms of \(I(0)=I_0\) and
\(\alpha\) however we have written the code in such a way as we can pass the
actual parameters:

\begin{pyin}
sol = get_solution(alpha=2, I_0=100)
print(sol)
\end{pyin}

which gives:

\begin{pyout}
Eq(I(t), 100*exp(-2*t))
\end{pyout}

Now, to calculate the cost we will write a function to integrate our result:

\begin{pyin}
def get_cost_without_cure(
    I_0=I_0, alpha=alpha, cost_per_person=10
):
    """Return the cost.

    Args:
        I_0: a float (default: symbolic I_0)
        alpha: a float (default: symbolic alpha)
        cost_per_person: a float (default: 10)

    Returns:
        A symbolic expression
    """
    I_sol = get_solution(I_0=I_0, alpha=alpha)
    return (
        sym.integrate(I_sol.rhs, (t, 0, sym.oo))
        * cost_per_person
    )
\end{pyin}

The cost with cure can use the above but needs to modify the rate \(\alpha\) and
include the one off cost per person:

\begin{pyin}
def get_cost_with_cure(
    I_0=I_0,
    alpha=alpha,
    cost_per_person=10,
    cost_of_cure_per_person=5,
):
    """Return the cost when purchasing the cure.

    Args:
        I_0: a float (default: symbolic I_0)
        alpha: a float (default: symbolic alpha)
        cost_per_person: a float (default: 10)
        cost_of_cure_per_person: a float (default: 10)

    Returns:
        A symbolic expression
    """
    return (
        get_cost_without_cure(
            I_0=I_0,
            alpha=2 * alpha,
            cost_per_person=cost_per_person,
        )
        + cost_of_cure_per_person * I_0
    )
\end{pyin}

We can now obtain the cost without purchasing the cure:

\begin{pyin}
print(get_cost_without_cure(I_0=100, alpha=2))
\end{pyin}

which gives:

\begin{pyout}
500
\end{pyout}

and the cost of purchasing the cure:

\begin{pyin}
print(get_cost_with_cure(I_0=100, alpha=2))
\end{pyin}

which gives:

\begin{pyout}
750
\end{pyout}

So given the current parameters it is not worth purchasing the cure.

\section{Solving with R}\label{sec:solving-with-R}

% TODO Write R.

\section{Research}\label{sec:research}

TBA
