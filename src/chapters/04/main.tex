\chapter[Discrete Event Simulation]{Discrete Event Simulation}

% Introduction
\chapterinitial{C}{omplex} situations further companded by randomness appear
throughout our daily lives. For example, data flowing through a computer
network, patients being treated at an emergency services, and daily commutes to
work.
Mathematics can be used to understand these complex situations so as to
make predications which in turn can be used to make improvements. One tool used
to do this is to let a computer create a dynamic virtual representation of the
scenario in question, the particular type we are going to cover here is called
Discrete Event Simulation.

\section{Problem}\label{sec:problem}

- proposed Bike repair shop
- Inspection followed by repair (don't need repair if inspection: 20\% of bikes
don't need repair)
- 1 inspection counter
- 2 repair people
- Infinite capacity
- The shop wants to guarantee a 30 minute maximum time in the shop for all
bikes: is this feasible? (Possibly include costs/salary/money back guarantee).

\section{Theory}\label{sec:theory}

- Law of large numbers.
- Pseudo random numbers (maybe seeding?, possibly a graph showing law of large
numbers)
- Trials are important. (Smoothing, how many ones in 10 dice rolls).
- Put in context of a simple queueing system
- Two types of DES

\section{Solving with Python}\label{sec:solving-with-python}

- Build ciw model with some set of parameters
- Run ciw model (with trials) and find that you can't do it.
- Write function to build ciw model with given set of parameters
- Loops.
- Conclusion? (possibly visualisation)


\section{Solving with R}\label{sec:solving-with-R}

- Build simmer model with some set of parameters
- Run simmer model (with trials) and find that you can't do it.
- Write function to build simmer model with given set of parameters
- Loops.
- Conclusion? (possibly visualisation)

\section{Research}\label{sec:research}

TBA
