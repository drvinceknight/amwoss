\chapter[Discrete Event Simulation]{Discrete Event Simulation}

% Introduction
\chapterinitial{C}{omplex} situations further compounded by randomness appear
throughout our daily lives. For example, data flowing through a computer
network, patients being treated at an emergency services, and daily commutes to
work.
Mathematics can be used to understand these complex situations so as to
make predications which in turn can be used to make improvements. One tool used
to do this is to let a computer create a dynamic virtual representation of the
scenario in question, the particular type we are going to cover here is called
Discrete Event Simulation.

\section{Problem}\label{sec:problem}

Consider the following situation: a bicycle repair shop would like reconfigure
their set-up in order to guarantee that all bicycles processed by the repair
shop take a maximum of 30 minutes.
Their current set-up is as follows:

\begin{itemize}
  \item Bicycles arrive randomly at the shop at a rate of 15 per hour.
  \item They wait in line to be seen at an inspection counter, manned by one
  member of staff who can inspect one bicycle at a time. On average an
  inspection takes around 3 minutes.
  \item After inspection it is found that around 20\% of bicycles do not need
  repair, and they are then ready for collection.
  \item After inspection is is found that around 80\% of bicycles go on to be
  repaired. These then wait in line outside the repair workshop, which is manned
  by two members of staff who can each repair one bicycle at a time. On average
  a repair takes around 6 minutes.
  \item After repair the bicycles are ready for collection.
\end{itemize}

A diagram of the system is shown below:

\includestandalone[width=\textwidth]{./assets/bike-repair-shop}

We can also assume that there is infinite capacity at the bicycle repair shop
for waiting bicycles.
The shop will hire and extra member of staff in order to meet their target of a
maximum throughout of 30 minutes. They would like to know if they should work on
the inspection counter or in the repair workshop?


\section{Theory}\label{sec:theory}

A number of the events that govern the behaviour of the bicycle shop above are
probabilistic. For example the
times that bicycles arrive at the shop, the duration of the inspection and
repairs, and whether the bicycle would need to go on to be repaired or not.
When a number of these probabilistic events are arranged in a complex system
such as the bicycle shop, using analytical methods to manipulate these
probabilities becomes difficult. One method to deal with this is
\textit{simulation}.

Consider one probabilistic event, rolling a die.
A die has six sides numbered 1 to 6, each side is equally likely to land.
Therefore the probability of rolling a 1 is $\frac{1}{6}$, the probability of
rolling a 2 is $\frac{1}{6}$, and so on. This means that that if we roll the die
a large number of times, we would except $\frac{1}{6}$ of those rolls to be a 1.
This is called the \textit{law of large numbers}.

Now imagine we have an event in which we do not know the analytical probability
of it occurring. Consider rolling a weighted die, in this case a
die in which the probability of obtaining one number is much greater than the
others. How can we estimate the probability of obtaining a 5 on this die?

Rolling the weighted die once does not give us much information.
However due to the law of large numbers, we can roll this die a number of times,
and find the proportion of those rolls which gave a 5. The more times we roll
the die, the closer this proportion approaches the underlying probability of
obtaining a 5.

For a complex system such as the bicycle shop, we would like to estimate the
proportion of bicycles that take longer than 30 minutes to be processed. As it
is a complex system it is difficult to work this out analytically. So we would
like to observe this system a number of times and record the overall proportions
of bicycles spending longer than 30 minutes in the shop, which will converge to
the true underlying proportion.
However unlike rolling a weighted die, it it costly to observe this shop over a
number of days with identical conditions. In this case it is costly in terms of
time, as the repair shop already exists. However some scenarios, for example the
scenario where the repair shop hires and additional member of staff, do not yet
exist, so observing this this would be costly in terms of money also.
We can however build a virtual representation of this complex system on a
computer, and observe a virtual day of work much more quickly and much less
costly on the computer, similar to a video game.

% Computers not good at generating random numbers, so use pseudo random numbers. e.g. (formula)
In order to do this, the computer needs to be able to generate random outcomes
of each of the smaller events that make up the large complex system. Generating
random events are essentially doing things to random numbers, that need to be
generated.

Computers are deterministic, therefore true randomness is not always possible.
They can however generate pseudorandom numbers: sequences of numbers that look
like random numbers, but are entirely determined from the
previous numbers in the sequence. Most programming languages have methods of
doing this.

In order to simulate an event we can again manipulate the law of large numbers.
Let $X \sim U(0, 1)$, a uniformly pseudorandom variable between 0 and 1.
Let $R$ be the outcome of a roll of an unbiased die. Then $R$ can be defined as:

\begin{equation}
R =
  \begin{cases}
    1 & \text{if } 0 \leq X < \frac{1}{6}\\
    2 & \text{if } \frac{1}{6} \leq X < \frac{2}{6}\\
    3 & \text{if } \frac{2}{6} \leq X < \frac{3}{6}\\
    4 & \text{if } \frac{3}{6} \leq X < \frac{4}{6}\\
    5 & \text{if } \frac{4}{6} \leq X < \frac{5}{6}\\
    6 & \text{if } \frac{5}{6} \leq X < 1
  \end{cases}
\end{equation}

% What we gonna do: use pseudo random numbers to 'simulate' the system. - because law of large number we can simulate many times and smoothe.
The bicycle repair shop is a system made up of interactions of a number of other
simpler random events. This can be thought of as many interactions of random
variables, each generated using pseudorandom numbers.

In this case the fundemental random event that need to be generated are:
\begin{itemize}
  \item time each bicycle arrives to the repair shop,
  \item the time each bicycle spends at the inspection counter,
  \item whether each bicycle needs to go on the the repair workshop,
  \item the time each those bicycles spends at the repair shop.
\end{itemize}

As the simulation progresses these events will be generated, and will interact
together as described in Section~\ref{sec:problem}.
The proportion of customers spending longer than 30 minutes in the shop can then
be counted. This proportion itself is a random variable, and so just like the
weighted die, running this simulation once does not give us much information.
But we can run the simulation many times and take an average proportion, to
smooth out any variability.

% Two types of DES - event scheduleing in Python, process-based in R.
The process outlined above is a particular implementation of Monte Carlo
simulation called \textit{discrete event simulation}, which generates
pseudorandom numbers and observes their interactions. In practice there are two
main approaches to simulating complex probabilistic systems such as this one:
the \textit{event scheduling} approach, and \textit{process based} simulation.
It just so happens that the main implementations in Python and R use each of
these approaches, so you will see both approaches used here.


% Event scheduling
\subsection{Event Scheduling Approach}
When using the event scheduling approach, we can think of the `virtual
representation' of the system as being the facilities that the bicycles use, and
let entities (the bicycles) interact with these facilities. It is these
facilities that determine how the entities behave.

In a simulation that uses an event scheduling approach, a key concept is that
events occur that cause further events to occur in the future, either
immediately or after a delay, such as after some time in service.
In the bicycle shop examples of such events include a bicycle joining a queue, a
bicycle beginning service, and a bicycle finishing service. At each event the
event list is updated, and the clock then jumps forward to the next event in
this updated list.

% Process based
\subsection{Process Based Simulation}
When using process based simulation, we can think of the `virtual
representation' of the system as being the sequence of actions that each entity
(the bicycles) must take, and these sequences of actions might contain delays as
a number of entities seize and release a finite amount of resources. It is the
sequence of actions that determine how the entities behave.

For the bicycle repair shop an example of one possible sequence of actions would
be:

\textit{arrive $\rightarrow$ seize inspection counter $\rightarrow$ delay $\rightarrow$ release inspection counter $\rightarrow$ seize repair shop $\rightarrow$ delay $\rightarrow$ release repair shop $\rightarrow$ leave}

The scheduled delays in this sequence of events correspond to the time spend
being inspected and the time spend being repaired. Waiting in line for service
at these facilities are not included in the sequence of events; that is implicit
by the `seize' and `release' actions, as an entity will wait for a free resource
before seizing one.
Therefore in process based simulations, in addition to defining a sequence of
events, resource types and their numbers also need to be defined.


\section{Solving with Python}\label{sec:solving-with-python}

In this book we will use the Ciw library in order to conduct discrete event
simulation.
Ciw uses the event scheduling approach, which means we must define the system's
facilities, and then let customers loose to interact with them.

In this case there are two facilities to define: the inspection desk and the
repair workshop. Let's order these as so. For each of these we need to define:

\begin{itemize}
  \item the distribution of times between consecutive bicycles arriving,
  \item the distribution of times the bicycles spend in service,
  \item the number of servers available,
  \item the probability of routing to each of the other facilities after
  service.
\end{itemize}

In this case we will assume that the time between consecutive arrivals follow a
exponential distribution, and that the service times also follow exponential
distributions. These are common assumptions for this sort of queueing system.

In Ciw, these are defined in a Network object, created using the
\mintinline{python}{ciw.create_network} function. The code below uses this
function to create a Network object that defines the current bicycle repair
shop:

\begin{pyin}
import ciw

N = ciw.create_network(
    arrival_distributions=[ciw.dists.Exponential(15), ciw.dists.NoArrivals()],
    service_distributions=[ciw.dists.Exponential(20), ciw.dists.Exponential(10)],
    number_of_servers=[1, 2],
    routing=[[0.0, 0.8], [0.0, 0.0]]
)
\end{pyin}

This function takes arguments that are used to define each of the four
properties we listed above. Each of these arguments take in a list of
properties, for each facility respectively in their order.
Arguments asking for for distributions take in Ciw objects defining those
distributions, \mintinline{python}{ciw.dists.Exponential} for exponential
distributions, and \mintinline{python}{ciw.dists.NoArrivals} for when there are
no external arrivals at the repair workshop (all bicycles arriving there are
routed from the inspection desk).
The \mintinline{python}{number_of_servers} argument takes a list of integers (1
server at the inspection desk, 2 at the repair workshop).
The \mintinline{python}{routing} argument defines how bicycles are routed after
service: from the inspection desk there is zero probability of returning to the
inspection desk, and a probability of 0.8 of being routed to the repair
workshop; from the repair workshop bicycles are not routed anywhere, they leave
the system.

Now this we have defined the system, we need to use this to build the virtual
representation of the system: in Ciw this is a Simulation object:

\begin{pyin}
ciw.seed(0)
Q = ciw.Simulation(N)
\end{pyin}

Notice here a random seed is set. This is because there is some element of
randomness when initialising this object, and in order to ensure reproducible
results we force the pseudorandom number generator to produce the same sequence
of pseudorandom numbers each time.

Now we have a virtual representation of the system, we can run this for one
eight hour working day:

\begin{pyin}
Q.simulate_until_max_time(8)
\end{pyin}

Note that the simulation always begins with an empty system, so the first
bicycle to arrive will never wait for service. Depending on the situation this
may be an unwanted feature, though not in this case as it is reasonable to
assume that the bicycle shop will begin the day with no customers.
Now we'll count the number of bicycles that have finished service, and count the
number of those whose entire journey through the system lasted longer than 0.5
hours:

\begin{pyin}
left_individuals = Q.nodes[-1].all_individuals
number_over_30mins = 0
for ind in left_individuals:
    throughput = 0
    for record in ind.data_records:
        throughput += record.exit_date - record.arrival_date
    if throughput > 0.5:
        number_over_30mins += 1
print(number_over_30mins / len(left_individuals))
\end{pyin}

The first line here obtains a list of all the individuals who have left the
system, that is reach the final (index -1) node.
Each individual will have a data record for each facility they visited. This
contains many pieces of information about their time at that facility, the ones
relevant here are their \mintinline{python}{arrival_date} and their
\mintinline{python}{exit_date}, the difference of which gives their total time
spent at that facility. Summing the total time spent at every facility gives
their overall throughput in the system.

This piece of code gives

\begin{pyout}
0.26126126126126126
\end{pyout}

meaning 26.13\% of all bicycles spent longer than half an hour at the repair
shop.

However this particular day may have contained a number of extreme events.
For a more accurate proportion this experiment should be repeated, and an
average proportion taken.
In order to do so, let's write a function that performs the above experiment, so
that we can eventually repeat the function call.

\begin{pyin}
def find_percentage_over_30mins(n_inspectors, n_repairers, seed):
    """Returns the percentage of bicycles spending over 30 minutes
    at the repair shop in one run of the simulation.
    
    Args:
        n_inspectors: the number of servers at the inspection desk
        n_repairers: the number of servers in the repair workshop
        seed: the seed for the pseudorandom number generator

    Returns:
        A real.
    """
    N = ciw.create_network(
        arrival_distributions=[ciw.dists.Exponential(15), ciw.dists.NoArrivals()],
        service_distributions=[ciw.dists.Exponential(20), ciw.dists.Exponential(10)],
        number_of_servers=[n_inspectors, n_repairers],
        routing=[[0.0, 0.8], [0.0, 0.0]]
    )

    ciw.seed(seed)
    Q = ciw.Simulation(N)
    Q.simulate_until_max_time(24)

    left_individuals = Q.nodes[-1].all_individuals
    number_over_30mins = 0
    for ind in left_individuals:
        throughput = 0
        for record in ind.data_records:
            throughput += record.exit_date - record.arrival_date
        if throughput > 0.5:
            number_over_30mins += 1
    return number_over_30mins / len(left_individuals)
\end{pyin}

This can be used to find the average proportion over 100 trials:

\begin{pyin}
percentage_over_30 = []
for trial in range(100):
    percentage_over_30.append(find_percentage_over_30mins(1, 2, trial))
print(sum(percentage_over_30) / len(percentage_over_30))
\end{pyin}

which gives:

\begin{pyout}
0.15935355368513382
\end{pyout}

that is, on average 15.94\% of bicycles will spend longer than 30 minutes at the
repair shop.

Now consider the two possible future scenarios we wish top compare: hiring an
extra member of staff to serve at the inspection desk, or hiring an extra member
of staff at the repair workshop. Which scenario yields a smaller proportion of
bicycles spending over 30 minutes at the shop? Let's investigate. First look at
the situation where the additional member of staff works at the inspection desk:

\begin{pyin}
percentage_over_30 = []
for trial in range(100):
    percentage_over_30.append(find_percentage_over_30mins(2, 2, trial))
print(sum(percentage_over_30) / len(percentage_over_30))
\end{pyin}

which gives:

\begin{pyout}
0.038476805648229126
\end{pyout}

that is 3.85\% of bicycles.

Now look at the situation where the additional member of staff works at the
repair workshop:

\begin{pyin}
percentage_over_30 = []
for trial in range(100):
    percentage_over_30.append(find_percentage_over_30mins(1, 3, trial))
print(sum(percentage_over_30) / len(percentage_over_30))
\end{pyin}

which gives:

\begin{pyout}
0.10359146418929761
\end{pyout}

that is 10.36\% of bicycles.

Therefore an additional member of staff at the inspection desk would be more
beneficial than an additional member of staff at the repair workshop.


\section{Solving with R}\label{sec:solving-with-R}

In this book we will use the Simmer package in order to conduct discrete event
simulation.
Simmer uses the process based approach, which means we must define the each
bicycle's sequence of actions, and then generate a number of bicycles with these
sequences.

In Simmer these sequences of actions are made up of called trajectories. The
diagram below shows the branched trajectories than a bicycle would take at the
repair shop:

% TODO add trajectories diagram

These are defined in simmer by:

\begin{Rin}
library(simmer)

bicycle <-
  trajectory("Inspection") %>%
  seize("Inspector") %>%
  timeout(function() {rexp(1, 20)}) %>%
  release("Inspector") %>%
  branch(
    function() (runif(1) < 0.8), continue=c(F),
    trajectory("Repair") %>%
      seize("Repairer") %>%
      timeout(function() {rexp(1, 10)}) %>%
      release("Repairer"),
    trajectory("Out")
  )
\end{Rin}

Here we define a \mintinline{R}{bicycle} object which is made up of Simmer
\mintinline{R}{trajectory}s. You will see that this sequence of event matches
those in Figure~\ref{fig:simmertrajectories}.
First the bicycle seizes an ``Inspector'' resource (yet to be defined), pauses
for some service time, sampled from an exponential distribution with parameter
20, then releases the ``Inspector'' resource, so that resource is free to be
seized by another bicycle.
Then trajectory branches on the condition that an uniformly pseudorandom
variable lies above or lies 0.8: if below begin the ``Repair'' sub-trajectory,
otherwise the ``Out'' sub-trajectory which does not have any actions.
The ``Repair trajectory first seizes a ``Repairer'' resource (again, yet to be
defined), pauses for some service time, sampled from an exponential
distribution with parameter 10, then releases the ``Repairer'' resource. Once
there are no actions left in the sequence the bicycle leaves the system.

These trajectories are not very useful alone, we are yet to define the resources
used, or a way to generate bicycles with these trajectories. This is done in the
code below, where a \mintinline{R}{repair_shop} is defined:

\begin{Rin}
repair_shop <-
  simmer("Repair Shop") %>%
  add_resource("Inspector", 1) %>%
  add_resource("Repairer", 2) %>%
  add_generator("Bicycle", bicycle, function() {rexp(1, 15)})
\end{Rin}

Here we have added one resource labelled ``Inspector'', and two resources
labelled ``Repairer''. Finally we have added a generator which generates a
number of the \mintinline{R}{bicycle} objects that we defined earlier. This
generator also takes in a function to sample the delays between generating new 
\mintinline{R}{bicycle} objects, that is the inter-arrival times. 

Now we have a virtual representation of the system, we can run this for one
eight hour working day:

\begin{Rin}
set.seed(seed)
repair_shop %>% run(until=8)
\end{Rin}

Note that the simulation always begins with an empty system, so the first
bicycle to arrive will never wait for service. Depending on the situation this
may be an unwanted feature, though not in this case as it is reasonable to
assume that the bicycle shop will begin the day with no customers.
Now we'll count the number of bicycles that have finished service, and count the
number of those whose entire journey through the system lasted longer than 0.5
hours:

\begin{Rin}
recs <- repair_shop %>% get_mon_arrivals()
throughput = recs$end_time - recs$start_time
print(mean(throughput > 0.5))
\end{Rin}

The first line here obtains a data frame of information about all the arrivals
to the system.
This data frame will have columns containing a limited number of pieces of
information about the arrivals, however much more information can be recorded
and monitored by adding monitoring actions to the trajectories. The relevant
columns here are the \mintinline{R}{start_time} and \mintinline{R}{end_time},
the difference of which gives their total time spent in the system, the
throughput. A new vector is created containing throughput information.
We then calculate the proportion of entries in that vector over half an hour by
taking the mean of a vector of boolean values checking this condition (this
works because the booleans \mintinline{R}{TRUE} and \mintinlin{R}{FALSE} have
numeric values of 1 and 0 respectively).

This piece of code gives

\begin{Rout}
[1] 0.02777778
\end{Rout}

meaning 27.78\% of all bicycles spent longer than half an hour at the repair
shop.

However this particular day may have contained a number of extreme events.
For a more accurate proportion this experiment should be repeated, and an
average proportion taken.
In order to do so, let's write a function that performs the above experiment, so
that we can eventually repeat the function call.

\begin{Rin}
#' Returns the percentage of bicycles spending over 30 minutes
#' at the repair shop in one run of the simulation.
#'
#' @param n_inspectors the number of servers at the inspection desk
#' @param n_repairers the number of servers in the repair workshop
#' @param seed the seed for the pseudorandom number generator
#'
#' @return A real
find_percentage_over_30mins <- function(n_inspectors, n_repairers, seed) {
  bicycle <-
    trajectory("Inspection") %>%
    seize("Inspector") %>%
    timeout(function() {rexp(1, 20)}) %>%
    release("Inspector") %>%
    branch(
      function() (runif(1) < 0.8), continue=c(F),
      trajectory("Repair") %>%
        seize("Repairer") %>%
        timeout(function() {rexp(1, 10)}) %>%
        release("Repairer"),
      trajectory("Out")
    )

  repair_shop <-
    simmer("Repair Shop") %>%
    add_resource("Inspector", n_inspectors) %>%
    add_resource("Repairer", n_repairers) %>%
    add_generator("Bicycle", bicycle, function() {rexp(1, 15)})

  set.seed(seed)
  repair_shop %>% run(until=8)
  recs <- repair_shop %>% get_mon_arrivals()
  throughput = recs$end_time - recs$start_time
  return(mean(throughput > 0.5))
}
\end{Rin}

This can be used to find the average proportion over 100 trials:

\begin{Rin}
percentage_over_30 <- c()
for (seed in 1:100) {
  percentage_over_30[seed] <- find_percentage_over_30mins(1, 2, seed)
}
print(mean(percentage_over_30))
\end{Rin}

which gives:

\begin{Rout}
[1] 0.1551579
\end{Rout}

that is, on average 15.52\% of bicycles will spend longer than 30 minutes at the
repair shop.

Now consider the two possible future scenarios we wish top compare: hiring an
extra member of staff to serve at the inspection desk, or hiring an extra member
of staff at the repair workshop. Which scenario yields a smaller proportion of
bicycles spending over 30 minutes at the shop? Let's investigate. First look at
the situation where the additional member of staff works at the inspection desk:

\begin{Rin}
percentage_over_30 <- c()
for (seed in 1:100) {
  percentage_over_30[seed] <- find_percentage_over_30mins(2, 2, seed)
}
print(mean(percentage_over_30))
\end{Rin}

which gives:

\begin{Rout}
[1] 0.04115338
\end{Rout}

that is 4.12\% of bicycles.

Now look at the situation where the additional member of staff works at the
repair workshop:

\begin{Rin}
percentage_over_30 <- c()
for (seed in 1:100) {
  percentage_over_30[seed] <- find_percentage_over_30mins(1, 3, seed)
}
print(mean(percentage_over_30))
\end{Rin}

which gives:

\begin{Rout}
[1] 0.1000899
\end{Rout}

that is 10.01\% of bicycles.

Therefore an additional member of staff at the inspection desk would be more
beneficial than an additional member of staff at the repair workshop.


\section{Research}\label{sec:research}

TBA
