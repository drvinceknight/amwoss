\chapter[Differential Equations]{Differential Equations}

% Introduction
\chapterinitial{S}{ystems} often change in a way that depends on their current
state. For example, the speed at which a cup of coffee cools down depends on its
current temperature. These types of systems are called dynamical systems and are
modelled mathematically using differential equations. This chapter will
consider a direct solution approach using symbolic mathematics.

\section{Problem}\label{sec:problem}

Consider the following situation: the entire population of a small rural town
has caught a cold. All 100 individuals will recover at an average rate of 2 per
day. The town leadership have noticed that being ill costs approximately
\pounds10 per day, this is due to general lack of productivity, poorer mood and
other intangible aspects. They need to decide whether or not to order cold
medicine which would \textbf{double} the recovery rate. The cost of of the cold
medicine is a one off cost of \pounds5 per person.

\section{Theory}\label{sec:theory}

In the case of this town, the overall rate at which people get better is
dependent on the number of people in how are ill. This can be represented
mathematically using a differential equation which is a way of relating the rate
of change of a system to the state of the system itself.

In general the objects of interest are the variable \(x\) over time \(t\), and
the rate at which \(x\) changes with \(t\), its derivative \(\frac{dx}{dt}\).
The differential equation describing this will be of the form:

\begin{equation}
    \frac{dx}{dt} = f(x)
\end{equation}

for some function \(f\).
In this case,
the number of infected individuals will be denoted as \(I\), which will
implicitly mean that \(I\) is a function of time: \(I=I(t)\), and the rate at
which individuals recover will be denoted by \(\alpha\), then the
differential equation that describes the above situation is:

\begin{equation}
    \frac{dI}{dt} = -\alpha I
\end{equation}

Finding a solution to this differential equation means finding an expression for
\(I\) that when differentiated gives \(- \alpha I\).

In this particular case, one such function is:

\begin{equation}
    I(t) = e ^ {-\alpha t}
\end{equation}

This is a solution because:
\(\frac{dI}{dt} = -\alpha e ^ {-\alpha y} = -\alpha I\).

However here \(I(0) = 1\), whereas for this problem we know that at time \(t=0\)
there are 100 infected individuals. In general there are many such functions
that can satisfy a differential equation, known as a family of solutions. To
know which particular solution is relevant to the situation, some sort of
initial (also referred to as boundary) condition is required. Here this would
be:

\begin{equation}
    I(t) = 100e ^ {-\alpha t}
\end{equation}

To evaluate the cost the sum of the values of that function over time is needed.
Integration gives exactly this, so the cost would be:

\begin{equation}
    K \int_{0}^{\infty}I(t)dt
\end{equation}

where \(K\) is the cost per person per unit time.

In the upcoming sections code will be used to confirm to carry out the above
efficiently so as to answer the original question.

\section{Solving with Python}\label{sec:solving-with-python}

The first step is to write a function to obtain the differential
equation. The Python library SymPy is used which allows symbolic calculations.

\begin{pyin}
import sympy as sym

t = sym.Symbol("t")
alpha = sym.Symbol("alpha")
I_0 = sym.Symbol("I_0")
I = sym.Function("I")


def get_equation(alpha=alpha):
    """Return the differential equation.

    Args:
        alpha: a float (default: symbolic alpha)

    Returns:
        A symbolic equation
    """
    return sym.Eq(sym.Derivative(I(t), t), -alpha * I(t))
\end{pyin}

This gives an equation that defines the population change over time:

\begin{pyin}
eq = get_equation()
print(eq)
\end{pyin}

which gives:

\begin{pyout}
Eq(Derivative(I(t), t), -alpha*I(t))
\end{pyout}

Note that if you are using Jupyter then your output will actually be a
well rendered mathematical equation:

\[
\frac{d}{d t} I{\left(t \right)} = - \alpha I{\left(t \right)}
\]

A value of \(\alpha\) can be passed if required:

\begin{pyin}
eq = get_equation(alpha=1)
print(eq)
\end{pyin}

\begin{pyout}
Eq(Derivative(I(t), t), -I(t))
\end{pyout}

Now a function will be written to obtain the solution to this differential with
initial condition \(I(0) = I_0\):

\begin{pyin}
def get_solution(I_0=I_0, alpha=alpha):
    """Return the solution to the differential equation.

    Args:
        I_0: a float (default: symbolic I_0)
        alpha: a float (default: symbolic alpha)

    Returns:
        A symbolic equation
    """
    eq = get_equation(alpha=alpha)
    return sym.dsolve(eq, I(t), ics={I(0): I_0})
\end{pyin}

This can verify the solution discussed previously:

\begin{pyin}
sol = get_solution()
print(sol)
\end{pyin}

which gives:

\begin{pyout}
Eq(I(t), I_0*exp(-alpha*t))
\end{pyout}

\[I(t) = I_0 e ^{-\alpha t}\]

SymPy itself can be used to verify the result, by taking the derivative of the
right hand side of our solution.

\begin{pyin}
print(sym.diff(sol.rhs, t) == -alpha * sol.rhs)
\end{pyin}

which gives:

\begin{pyout}
True
\end{pyout}

All of the above has given the general solution in terms of \(I(0)=I_0\) and
\(\alpha\), however the code is written in such a way as we can pass the actual
parameters:

\begin{pyin}
sol = get_solution(alpha=2, I_0=100)
print(sol)
\end{pyin}

which gives:

\begin{pyout}
Eq(I(t), 100*exp(-2*t))
\end{pyout}

Now, to calculate the cost write a function to integrate the result:

\begin{pyin}
def get_cost(
    I_0=I_0,
    alpha=alpha,
    cost_per_person=10,
    cost_of_cure=0,
):
    """Return the cost.

    Args:
        I_0: a float (default: symbolic I_0)
        alpha: a float (default: symbolic alpha)
        cost_per_person: a float (default: 10)
        cost_of_cure: a float (default: 0)

    Returns:
        A symbolic expression
    """
    I_sol = get_solution(I_0=I_0, alpha=alpha)
    return (
        sym.integrate(I_sol.rhs, (t, 0, sym.oo))
        * cost_per_person
        + cost_of_cure * I_0
    )
\end{pyin}

The cost without purchasing the cure is:

\begin{pyin}
I_0 = 100
alpha = 2
cost_without_cure = get_cost(I_0=I_0, alpha=alpha)
print(cost_without_cure)
\end{pyin}

which gives:

\begin{pyout}
500
\end{pyout}


The cost with cure can use the above with a modified \(\alpha\) and a non zero
cost of the cure itself:

\begin{pyin}
cost_of_cure = 5
cost_with_cure = get_cost(
    I_0=I_0, alpha=2 * alpha, cost_of_cure=cost_of_cure
)
print(cost_with_cure)
\end{pyin}

which gives:

\begin{pyout}
750
\end{pyout}

So given the current parameters it is not worth purchasing the cure.

\section{Solving with R}\label{sec:solving-with-R}

R has some capability for symbolic mathematics, however at the time of writing
the options available are somewhat limited and/or not reliable. As such, in R
the problem will be solved using a numerical integration approach. For an
outline of the theory behind this approach see Chapter 5. % TODO Add reference to chapter

First write a function to give the derivative for a given value of \(I\).

\begin{Rin}
#' Returns the numerical value of the derivative.
#'
#' @param t a set of time points
#' @param y a function
#' @param parameters the set of all parameters passed to y

#' @return a float
derivative <- function(t, y, parameters) {
  with(as.list(c(y, parameters)), {
    dIdt <- -alpha * I  # nolint
    list(dIdt)  # nolint
  })
}
\end{Rin}
% TODO remove # nolint

For example, to see the value of the derivative when \(I=0\):

\begin{Rin}
derivative(t = 0, y = c(I = 100), parameters = c(alpha = 2))
\end{Rin}

This gives:

\begin{Rout}
[[1]]
[1] -200

\end{Rout}

Now the deSolve library will be used for solving differential equations
numerically:

\begin{Rin}
library(deSolve)  # nolint
#' Return the solution to the differential equation.
#'
#' @param times: a vector of time points
#' @param y_0: a float (default: 100)
#' @param alpha: a float (default: 2)

#' @return A vector of numerical values
get_solution <- function(times,
                         y0 = c(I = 100),
                         alpha = 2) {
  params <- c(alpha = alpha)
  ode(y = y0, times = times, func = derivative, parms = params)
}
\end{Rin}

% TODO remove # nolint

This will return a sequence of time point and values of \(I\) at those time
points. Using this we can compute the cost.

\begin{Rin}
#' Return the cost.
#'
#' @param I_0: a float (default: symbolic I_0)
#' @param alpha: a float (default: symbolic alpha)
#' @param cost_per_person: a float (default: 10)
#' @param cost_of_cure: a float (default: 0)
#' @param step_size: a float (default: 0.0001)
#' @param max_time: an integer (default: 10)

#' @return A numeric value
get_cost <- function(I_0 = 100,
                     alpha = 2,
                     cost_per_person = 10,
                     cost_of_cure = 0,
                     step_size = 0.0001,
                     max_time = 10) {
  times <- seq(0, max_time, by = step_size)
  out <- get_solution(times,
    y0 = c(I = I_0),
    alpha = alpha
  )
  number_of_observations <- length(out[, "I"])

  time_between_steps <- diff(out[, "time"])
  area_under_curve <- sum(
    time_between_steps *
      out[-number_of_observations, "I"]
  )
  area_under_curve *
    cost_per_person + cost_of_cure *
      I_0
}
\end{Rin}

The cost without purchasing the cure is:

\begin{Rin}
alpha <- 2
cost_without_cure <- get_cost(alpha = alpha)
print(round(cost_without_cure))
\end{Rin}


which gives:

\begin{Rout}
[1] 500
\end{Rout}

The cost with cure can use the above with a modified \(\alpha\) and a non zero
cost of the cure itself:

\begin{Rin}
cost_of_cure <- 5
cost_with_cure <- get_cost(
  alpha = 2 * alpha, cost_of_cure = cost_of_cure
)
print(round(cost_with_cure))
\end{Rin}

which gives:

\begin{Rout}
[1] 750
\end{Rout}

So given the current parameters it is not worth purchasing the cure.

\section{Research}\label{sec:research}

TBA


% Possibilities:
% + "A dynamic competition model of regime change" Syms R and Solymar L (https://www.tandfonline.com/doi/pdf/10.1057/jors.2015.28?needAccess=true)
% + "A Fluid Flow Model of Networks of Queues" Vandergraft K (https://www.jstor.org/stable/pdf/2631349.pdf?refreqid=excelsior%3A50876bb857b8a452892a441bff3e9f68)
