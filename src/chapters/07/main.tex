\chapter[Agent Based Simulation]{Agent Based Simulation}

% Introduction
\chapterinitial{S}{ometimes} we can know a lot about individuals' behaviours and
interactions, and would like to know about how a whole population of such
individuals might behave. For example psychologists and economists may know a
lot about how individual spenders and vendors behave in response to given
stimuli, and we'd like to know how these stimuli might effect the macro-economy.
Animal behaviour experts may understand individual animals' predator, prey and
mating habits, and would like know overall species population trends. Engineers
may write explicit individual instructions for self-driving cars, and would like
to investigate traffic and congestion behaviour for a city filled with such
vehicles. Agent based simulation (or agent based modelling, or ABM) is a
paradigm of thinking that allows such emergent population level behaviour to be
investigated from individual rules and interactions.

\section{Problem}\label{sec:problem}
Consider a city populated by two kinds of household, for example a household
might be fans of Cardiff City FC or Swansea City AFC. Each household has a
preference for living close to households of the same kind, and will move houses
around the city while their preferences are not satisfied.
In this situation we are interested in how segregated does the city naturally
get under these sorts of preferences.

\section{Theory}\label{sec:theory}
The model considered here is considered a `classic' one for the paradigm of
agent based simulation, and is usually called Schelling's segregation model.
It features in Thomas Schelling's book `Micromotives and Macrobehaviours', whose
title neatly summarises the world view of agent based modelling: we know,
understand, determine, or can control individual micromotives; and from this
we'd like to observe and understand macrobehaviours.

As a simplification we will model the city as a 50x50 grid.
Each box is a house that can either contain a household of Cardiff City FC
supporters, or contain a household of Swansea City AFC supporters.
Define a house's neighbours by the grid locations adjacent to it, horizontally,
vertically, and diagonally.
For mathematical simplicity, also assume that the grid is a torus, where houses
in the top row are vertically adjacent to the bottom row, and houses in the
rightmost column are horizontally adjacent to the leftmost column.

Next let's consider each household's behaviour.
Every household has a preference $p$.
This corresponds to the minimum proportion of neighbours they are happy to live
next to who support the same team as themselves.
Figure~\ref{fig:schelling_happyunhappy} shows a household of Cardiff City FC
supporters that are happy with their neighbours, and not happy with their
neighbours, when $p=0.5$. Households supporting Cardiff City FC are shaded grey,
while households supporting Swansea City AFC are white.

\begin{figure}
\begin{center}
\subfigure[A happy household, with 6 similar neighbours ($\frac{6}{8} > p = 0.5$)]{\includestandalone[width=0.4\textwidth]{./assets/schelling_happy}}
\subfigure[An unhappy household, with 2 similar neighbours ($\frac{2}{8} < p = 0.5$)]{\includestandalone[width=0.4\textwidth]{./assets/schelling_unhappy}}
\end{center}
\caption{Example of a household happy and unhappy with its neighbours, when
$p=0.5$. Households supporting Cardiff City FC are shaded grey, households
supporting Swansea City AFC are white.}
\label{fig:schelling_happyunhappy}
\end{figure}

The original problem stated that households randomly move around the city
whenever they are unhappy with their neighbours.
This long process of selling, searching for, and buying houses can be simplified
to randomly pairing two unhappy households and swapping their houses.
Let this happen to all unhappy households.

Therefore our model logic is:

\begin{enumerate}
  \item Initialise the model. Fill each house in the grid with either a
  household of Cardiff City FC or Swansea City AFC supporters with
  probability $0.5$ each.
  \item At each discrete time step, for every household:
  \begin{enumerate}
    \item Consider their neighbours, determine if the household is happy.
    \item If unhappy, swap houses with another randomly chosen unhappy
    household.
  \end{enumerate}
\end{enumerate}

After a number of time steps we can observe the overall structure of the city
and any population level behaviour that may have emerged without explicit
defining.

The above is an agent based model.
It is a model as it is an abstraction of the real system.
It is agent based as it only explicitly defines individual behaviours and
interactions, but we wish to observe overall population level behaviours not
explicitly defined.
Note that this does not require code to analyse: in fact this model was
originally run by placing and manually swapping silver and copper coins on a
chessboard.
A model isn't agent-based simply from the manner in which it is coded.
Coding the model does however allow it to be run efficiently, scaled, and allows
ease of analysis.


\section{Solving with Python}\label{sec:solving-with-python}

\section{Solving with R}\label{sec:solving-with-R}

\section{Research}\label{sec:research}
