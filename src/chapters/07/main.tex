\chapter[Agent Based Simulation]{Agent Based Simulation}

% Introduction
\chapterinitial{S}{ometimes} we can know a lot about individuals' behaviours and
interactions, and would like to know about how a whole population of such
individuals might behave. For example psychologists and economists may know a
lot about how individual spenders and vendors behave in response to given
stimuli, and we'd like to know how these stimuli might effect the macro-economy.
Animal behaviour experts may understand individual animals' predator, prey and
mating habits, and would like know overall species population trends. Engineers
may write explicit individual instructions for self-driving cars, and would like
to investigate traffic and congestion behaviour for a city filled with such
vehicles. Agent based simulation (or agent based modelling, or ABM) is a
paradigm of thinking that allows such emergent population level behaviour to be
investigated from individual rules and interactions.

\section{Problem}\label{sec:problem}
Consider a city populated by two kinds of household, for example a household
might be fans of Cardiff City FC or Swansea City AFC. Each household has a
preference for living close to households of the same kind, and will move houses
around the city while their preferences are not satisfied.
In this situation we are interested in how segregated does the city naturally
get under these sorts of preferences.

\section{Theory}\label{sec:theory}
The model considered here is considered a `classic' one for the paradigm of
agent based simulation, and is usually called Schelling's segregation model.
It features in Thomas Schelling's book `Micromotives and Macrobehaviours', whose
title neatly summarises the world view of agent based modelling: we know,
understand, determine, or can control individual micromotives; and from this
we'd like to observe and understand macrobehaviours.

As a simplification we will model the city as a 50x50 grid.
Each box is a house that can either contain a household of Cardiff City FC
supporters, or contain a household of Swansea City AFC supporters.
Define a house's neighbours by the grid locations adjacent to it, horizontally,
vertically, and diagonally.
For mathematical simplicity, also assume that the grid is a torus, where houses
in the top row are vertically adjacent to the bottom row, and houses in the
rightmost column are horizontally adjacent to the leftmost column.

Next let's consider each household's behaviour.
Every household has a preference $p$.
This corresponds to the minimum proportion of neighbours they are happy to live
next to who support the same team as themselves.
Figure~\ref{fig:schelling_happyunhappy} shows a household of Cardiff City FC
supporters that are happy with their neighbours, and not happy with their
neighbours, when $p=0.5$. Households supporting Cardiff City FC are shaded grey,
while households supporting Swansea City AFC are white.

\begin{figure}
\begin{center}
\subfigure[A happy household, with 6 similar neighbours ($\frac{6}{8} > p = 0.5$)]{\includestandalone[width=0.4\textwidth]{./assets/schelling_happy}}
\subfigure[An unhappy household, with 2 similar neighbours ($\frac{2}{8} < p = 0.5$)]{\includestandalone[width=0.4\textwidth]{./assets/schelling_unhappy}}
\end{center}
\caption{Example of a household happy and unhappy with its neighbours, when
$p=0.5$. Households supporting Cardiff City FC are shaded grey, households
supporting Swansea City AFC are white.}
\label{fig:schelling_happyunhappy}
\end{figure}

The original problem stated that households randomly move around the city
whenever they are unhappy with their neighbours.
This long process of selling, searching for, and buying houses can be simplified
to randomly pairing two unhappy households and swapping their houses.
Let this happen to all unhappy households.
In fact, we can simplify further and consider the houses themselves as agents,
and who swap households with another house.

Therefore our model logic is:

\begin{enumerate}
  \item Initialise the model: fill each house in the grid with either a
  household of Cardiff City FC or Swansea City AFC supporters with
  probability $0.5$ each.
  \item At each discrete time step, for every house:
  \begin{enumerate}
    \item Consider their household's neighbours, determine if the household is
    happy.
    \item If unhappy, swap household with another randomly chosen house with an
    unhappy household.
  \end{enumerate}
\end{enumerate}

After a number of time steps we can observe the overall structure of the city
and any population level behaviour that may have emerged without explicit
defining.

The above is an agent based model.
It is a model as it is an abstraction of the real system.
It is agent based as it only explicitly defines individual behaviours and
interactions, but we wish to observe overall population level behaviours not
explicitly defined.
Note that this does not require code to analyse: in fact this model was
originally run by placing and manually swapping silver and copper coins on a
chessboard.
A model isn't agent-based simply from the manner in which it is coded.
Coding the model does however allow it to be run efficiently, scaled, and allows
ease of analysis.


\section{Solving with Python}\label{sec:solving-with-python}
In agent based modelling we consider individual agents as their own entities,
with their own rules and behaviours.
This world view lends itself well to object-orientated programming.
Here we build a number of \textit{objects} from a set of instructions called a
\textit{class}.
These objects can both store information (in Python we call these
\textit{attributes}), and do things (in Python we call these \textit{methods}).

Python itself is written this way, and also allows users to define their own.

For this problem we will define two classes (types of object): a
\mintinline{python}{House} and a \mintinline{python}{City} for them to live in.

First let's import some useful libraries:

\begin{pyin}
import random
import itertools
import numpy as np
\end{pyin}

Now let's define the \mintinline{python}{City}:

\begin{pyin}
class City:
    def __init__(self, size, threshold):
        """Initialises the City object.

        Args:
            size: an integer number of rows and columns
            threshold: a number between 0 and 1 representing
              the minimum acceptable proportion of similar
              neighbours
        """
        self.size = size
        sides = range(size)
        self.coords = itertools.product(sides, sides)
        self.houses = {
            (x, y): House(x, y, threshold, self)
            for x, y in self.coords
        }

    def run(self, n_steps):
        """Runs the simulation of a number of time steps.

        Args:
            n_steps: an integer number of steps
        """
        for turn in range(n_steps):
            self.take_turn()

    def take_turn(self):
        """Swaps all sad households."""
        sad = [h for h in self.houses.values() if h.sad()]
        random.shuffle(sad)
        i = 0
        while i <= len(sad) / 2:
            sad[i].swap(sad[-i])
            i += 1

    def mean_satisfaction(self):
        """Finds the average household satisfaction.

        Returns:
            The average city's household satisfaction
        """
        return np.mean(
            [h.satisfaction() for h in self.houses.values()]
        )
\end{pyin}

This defines a class, a template or a set of instructions that can be used to
create instances of it, called objects.
For our model we only need one instance of the \mintinline{python}{City} class,
however it is useful to be able to produce more in order to run multiple trials
with different random seeds.
This class contains four methods: \mintinline{python}{__init__},
\mintinline{python}{run}, \mintinline{python}{take_turn} and
\mintinline{python}{mean_satisfaction}.

The \mintinline{python}{__init__} method is run whenever the object is first
created, and initialises the object.
In this case it sets a number of attributes.
First the square grid's \mintinline{python}{size} is defined, which is the
number of rows and columns of houses it contains.
Next we define \mintinline{python}{coords}, a list of tuples representing all
the possible coordinates of the grid, this uses the
\mintinline{python}{itertools} library for efficient looping.
Finally \mintinline{python}{houses} is defined, a dictionary with grid
coordinates as keys, and instances of the, yet to be defined,
\mintinline{python}{House} class representing the houses themselves.

The \mintinline{python}{run} method runs the simulation. For each
\mintinline{python}{n_steps} number of discrete time steps, the city runs the
method \mintinline{python}{take_turn}.
In this method, we first create a list of all the houses with households that
are unhappy with their neighbours; these are put in a random order using the
\mintinline{python}{random} library; and then working inwards from the ends,
houses with sad households are paired up and swap households.

The last method defined here is the \mintinline{python}{mean_satisfaction}
method, which is only used to observe any emergent behaviour.
This calculates the average satisfaction of all the houses in the grid, using
the \mintinline{python}{numpy} library for convenience.

In order to be able to create an instance of the above class, we need to define
a \mintinline{python}{House} class:

\begin{pyin}
class House:
    def __init__(self, x, y, threshold, city):
        """Initialises the House object.

        Args:
            x: the integer x-coordinate
            y: the integer y-coordinate
            threshold: a number between 0 and 1 representing
              the minimum acceptable proportion of similar
              neighbours
            city: an instance of the City class
        """
        self.x = x
        self.y = y
        self.threshold = threshold
        self.kind = random.choice(["Cardiff", "Swansea"])
        self.city = city

    def satisfaction(self):
        """Determines the household's satisfaction level.

        Returns:
            A proportion
        """
        same = 0
        for x, y in itertools.product([-1, 0, 1], [-1, 0, 1]):
            ax = (self.x + x) % self.city.size
            ay = (self.y + y) % self.city.size
            same += self.city.houses[ax, ay].kind == self.kind
        return (same - 1) / 8

    def sad(self):
        """Determines if the household is sad.

        Returns:
            a Boolean
        """
        return self.satisfaction() < self.threshold

    def swap(self, house):
        """Swaps two households.

        Args:
            house: the house object to swap household with
        """
        self.kind, house.kind = house.kind, self.kind
\end{pyin}

It contains four methods: \mintinline{python}{__init__},
\mintinline{python}{satisfaction}, \mintinline{python}{sad} and
\mintinline{python}{swap}.

The \mintinline{python}{__init__} methods sets a number of attributes at the
time the object is created: the house's \mintinline{python}{x} and
\mintinline{python}{y} coordinates (its column and row numbers on the grid);
its \mintinline{python}{threshold} which corresponds to $p$; its
\mintinline{python}{kind} which is randomly chosen between having a Cardiff City
FC supporting household or a Swansea City AFC supporting household; and finally
its \mintinline{python}{city}, an instance of the \mintinline{python}{City}
class, shared by all the houses.

The \mintinline{python}{satisfaction} method loops though each of the house's
neighbouring cells in the city grid, counts the number of neighbours that are
of the same kind as itself, and returns this as a proportion.
Then the \mintinline{python}{sad} method returns a boolean indicating of the
household's satisfaction is below the minimum threshold.

Finally the \mintinline{python}{swap} method takes another house object, and
swaps their household kinds.

Now let's create an instance of a city of size 50x50, with each household's
threshold 0.65:

\begin{pyin}
random.seed(0)
C = City(50, 0.65)
\end{pyin}

Here we have set a random seed for reproducibility.
The world initialises randomly, so let's check this initial average
satisfaction:

\begin{pyin}
print(C.mean_satisfaction())
\end{pyin}

\begin{pyout}
0.4998
\end{pyout}

This is well below the minimum threshold of $0.65$, and so on average most
households are unhappy here.
Let's run the simulation for 100 generations and see how this changes:

\begin{pyin}
C.run(100)
print(C.mean_satisfaction())
\end{pyin}

\begin{pyout}
0.9078
\end{pyout}

After 100 time steps the average satisfaction level is much higher.
In fact, is it much higher that each individual household's threshold.
Now consider that this satisfaction level is really a level of how similar
each households' neighbours are, it is actually a level of segregation.
This was the central premise of Schelling's original model, that overall
emergent segregation levels are much higher than any individuals' personal
preference for segregation.

More analysis methods can be added, including plotting functions.
Figure~\ref{fig:schelling_python_plot} shows the grid at the beginning, after 20
time steps, and after 100 time steps, with households supporting Cardiff City FC
in grey, and those supporting Swansea City AFC in white.
It visually shows the households naturally segregating over time.

\begin{figure}
\begin{center}
\subfigure[At the beginning.]{\includegraphics[width=0.32\textwidth]{./assets/python_schelling_0}}
\subfigure[After 20 time steps.]{\includegraphics[width=0.32\textwidth]{./assets/python_schelling_20}}
\subfigure[After 100 time steps.]{\includegraphics[width=0.32\textwidth]{./assets/python_schelling_100}}
\end{center}
\caption{Plotted results from the Python code.}
\label{fig:schelling_python_plot}
\end{figure}

\section{Solving with R}\label{sec:solving-with-R}
In agent based modelling we consider individual agents as their own entities,
with their own rules and behaviours.
This world view lends itself well to object-orientated programming.
Here we build a number of \textit{objects} from a set of instructions called a
\textit{class}.
These objects can both store information (in the R library we will use we call
these \textit{fields}), and do things (called \textit{methods}).

There are a number of ways of doing object orientated programming in R.
In this chapter, we will use a package called \mintinline{R}{R6}.

For this problem we will define two classes (types of object): a
\mintinline{R}{House} and a \mintinline{R}{City} for them to live in.

Now let's define the \mintinline{R}{City}:

\begin{Rin}
library(R6)
city <- R6Class("City", list(
  size = NA,
  houses = NA,
  initialize = function(size, threshold) {
    self$size <- size
    self$houses <- c()
    for (x in 1:size) {
      row <- c()
      for (y in 1:size) {
        row <- c(row, house$new(x, y, threshold, self))
      }
      self$houses <- rbind(self$houses, row)
    }
  },
  run = function(n_steps) {
    for (turn in 1:n_steps) {
      self$take_turn()
    } },
  take_turn = function() {
    sad <- c()
    for (house in self$houses) {
      if (house$sad()) {
        sad <- c(sad, house)
      }
    }
    sad <- sample(sad)
    num_sad <- length(sad)
    i <- 1
    while (i <= num_sad / 2) {
      sad[[i]]$swap(sad[[num_sad - i]])
      i <- i + 1
    }
  },
  mean_satisfaction = function() {
    mean(sapply(self$houses, function(x) x$satisfaction()))
  })
)
\end{Rin}

This defines an R6 class, a template or a set of instructions that can be used
to create instances of it, called objects.
For our model we only need one instance of the \mintinline{R}{City} class,
although it may be useful to be able to produce more in order to run multiple
trials with different random seeds.
This class contains four methods: \mintinline{R}{initialize},
\mintinline{R}{run}, \mintinline{R}{take_turn} and
\mintinline{R}{mean_satisfaction}.

The \mintinline{R}{initialize} method is run at the time the object is first
created.
It initialises the object by setting a number of its fields.
First the square grid's \mintinline{R}{size} is defined, which is the
number of rows and columns of houses it contains.
Then it's \mintinline{R}{houses} is defined by iteratively repeating the
\mintinline{R}{rbind} function to create a two-dimensional vector of instances
of the, yet to be defined, \mintinline{R}{House} class, representing the houses
themselves.

The \mintinline{R}{run} method runs the simulation. For each discrete time
step from 1 to \mintinline{R}{n_steps}, the world runs the method
\mintinline{R}{take_turn}.
In this method, we first create a list of all the houses with households that
are unhappy with their neighbours; these are put in a random order using the
\mintinline{R}{sample} function; and then working inwards from the ends,
houses with sad households are paired up and swap households.

The last method defined here is the \mintinline{R}{mean_satisfaction}
method, which is used to observe any emergent behaviour.
This calculates the average satisfaction of all the houses in the grid, using
the the \mintinline{R}{sapply} function to create a vector of all the houses'
satisfaction levels.

In order to be able to create an instance of the above class, we need to define
a \mintinline{R}{House} class:

\begin{Rin}
house <- R6Class("House", list(
  x = NA,
  y = NA,
  threshold = NA,
  city = NA,
  kind = NA,
  initialize = function(x = NA,
                        y = NA,
                        threshold = NA,
                        city = NA) {
    self$x <- x
    self$y <- y
    self$threshold <- threshold
    self$city <- city
    self$kind <- sample(c("Cardiff", "Swansea"), 1)
  },
  satisfaction = function() {
    same <- 0
    for (x in -1:1) {
      for (y in -1:1) {
        ax <- ( (self$x + x - 1) %% self$city$size) + 1
        ay <- ( (self$y + y - 1) %% self$city$size) + 1
        if (self$city$houses[[ax, ay]]$kind == self$kind) {
          same <- same + 1
        }
      }
    }
    (same - 1) / 8
  },
  sad = function() {
    self$satisfaction() < self$threshold
  },
  swap = function(house) {
    old <- self$kind
    self$kind <- house$kind
    house$kind <- old
  })
)
\end{Rin}

It contains four methods: \mintinline{R}{initialize},
\mintinline{R}{satisfaction}, \mintinline{R}{sad} and \mintinline{R}{swap}.

The \mintinline{R}{initialize} methods sets a number of the class' fields when
the object is created: the house's \mintinline{r}{x} and \mintinline{R}{y}
coordinates (its column and row numbers on the grid); its
\mintinline{R}{threshold} which corresponds to $p$; its \mintinline{R}{kind}
which is randomly chosen between having a Cardiff City
FC supporting household or a Swansea City AFC supporting household; and finally
its \mintinline{R}{city}, an instance of the \mintinline{R}{City} class, shared
by all the houses.

The \mintinline{R}{satisfaction} method loops though each of the house's
neighbouring cells in the city grid, counts the number of neighbours that are of
the same kind as itself, and returns this as a proportion.
The \mintinline{R}{sad} method returns a boolean indicating of the
household's satisfaction is below its minimum threshold.

Finally the \mintinline{R}{swap} method takes another house object, and
swaps their household kinds.

Now let's create an instance of a city of size 50x50, with each household's
threshold 0.65:

\begin{Rin}
set.seed(0)
our_city <- city$new(50, 0.65)
\end{Rin}

Here we have set a random seed for reproducibility.
The world initialises randomly, so let's check this initial average
satisfaction:

\begin{Rin}
print(our_city$mean_satisfaction())
\end{Rin}

\begin{Rout}
[1] 0.4956
\end{Rout}

This is well below the minimum threshold of $0.65$, and so on average most
households are unhappy here.
Let's run the simulation for 100 generations and see how this changes:

\begin{Rin}
our_city$run(100)
print(our_city$mean_satisfaction())
\end{Rin}

\begin{Rout}
[1] 0.9338
\end{Rout}

After 100 time steps the average satisfaction has increased.
It is now actually much higher that each individual household's threshold.
We can consider this satisfaction level as a level of how similar each
households' neighbours are, and so it is actually a level of segregation.
This was the central premise of Schelling's original model, that overall
emergent segregation levels are much higher than any individuals' personal
preference for segregation.

More analysis methods can be added, including plotting functions.
Figure~\ref{fig:schelling_R_plot} shows the grid at the beginning, after 20
time steps, and after 100 time steps, with households supporting Cardiff City FC
in grey, and those supporting Swansea City AFC in white.
It visually shows the households naturally segregating over time.

\begin{figure}
\begin{center}
\subfigure[At the beginning.]{\includegraphics[width=0.32\textwidth]{./assets/schelling_R_0}}
\subfigure[After 20 time steps.]{\includegraphics[width=0.32\textwidth]{./assets/schelling_R_20}}
\subfigure[After 100 time steps.]{\includegraphics[width=0.32\textwidth]{./assets/schelling_R_100}}
\end{center}
\caption{Plotted results from the R code.}
\label{fig:schelling_R_plot}
\end{figure}

\section{Research}\label{sec:research}
