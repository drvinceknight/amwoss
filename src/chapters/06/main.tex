\chapter[Systems dynamics]{Systems Dynamics}

% Introduction
\chapterinitial{I}{n} many situations systems are dynamical, in that the state
or population of a number of entities or classes change according the the
current state or population of the system. For example population dynamics,
chemical reactions, and systems of macroeconomics. It is often useful to be able
to predict how these systems will behave over time, though the rules which
govern these changes can be complex, and are not necessarily solvable
analytically. In these cases numerical methods can be used, which is the focus
of this chapter.

\section{Problem}\label{sec:problem}
Consider the following scenario, where a population of 3000 people are
susceptible to infection by some disease. This population can be described by
the following parameters:

\begin{itemize}
  \item They have a birth rate $b$ of 0.01 per day;
  \item They have a death rate $d$ of 0.01 per day;
  \item For every infectious individual, the infection rate $\alpha$ is 0.3 per
  day;
  \item Infectious people recover naturally (and thus gain an immunity from the
  disease), at a recovery rate $r$ of 0.02 per day;
  \item For each day an individual is infected, they must take medication which
  costs a public healthcare system $\pounds 10$ per day.
\end{itemize}

A vaccine is produced, which allows new born individuals to become recovered
immediately, and thus gain an immunity. This vaccine costs the public health
care system a one-off cost of $\pounds 220$ per vaccine. The healthcare
providers would like to know if achieving a vaccination rate $v$ of $85\%$ would
be beneficial financially.

\section{Theory}\label{sec:theory}
% Setup - diagram & equations
The above scenario can be described is called a compartmental model of disease, and can be shown in the stock and flow diagram in Figure~\ref{fig:stockflow}.

% TODO INCLUDE DIAGRAM

The system has three `stocks' of different types of individuals, those
susceptible to disease ($S$), those infected with the disease ($I$), and those
who have recovered from the disease and so have gained immunity ($R$). The
levels on these stocks change according to the flows in, out, and between them,
controlled by `taps'. The amount of flow the taps let through are influenced
multiplicatively (either negatively or positively), by other factors, such as
external parameters (e.g. birth rate, infection rate) and the stock levels.

In this system the following taps exist, influences by the following parameters:

\begin{itemize}
  \item \textit{external $\rightarrow S$:} Influenced positively by the birth
  rate, and negatively by the vaccine rate.
  \item \textit{$S \rightarrow I$:} Influenced positively by the infection rate,
  and the number of infected individuals.
  \item \textit{$S \rightarrow$ external:} Influenced positively by the death
  rate.
  \item \textit{$I \rightarrow R$:} Influenced positively by the recovery rate.
  \item \textit{$I \rightarrow$ external:} Influenced positively by the death
  rate.
  \item \textit{$R \rightarrow$ external:} Influenced positively by the birth
  rate and the vaccine rate.
  \item \textit{external $\rightarrow R$:} Influenced positively by the death
  rate.
\end{itemize}

Mathematically the change in stock levels are written as the derivatives, for
example the change in the number of susceptible individuals over time is denoted
by $\frac{dS}{dt}$. This can be equated tot he sum of the taps in or out of that
stock. Thus the system is described by the following system of differential
equations:

\begin{align}
\frac{dS}{dt} &= -\frac{\alpha SI}{N} + (1 - v)bN - dS\\
\frac{dI}{dt} &= \frac{\alpha SI}{N} - (r + d)I\\
\frac{dR}{dt} &= rI - dR + vbN
\end{align}

Notice here some taps need to be scaled by $N$, where $N = S + I + R$ the total
number of individuals in the system.

We would like to understand the behaviour of the functions $S$, $I$ and $R$
under these rules, that is we would like to solve this system of differential
equations. This system contains some non-linear terms, implying that this may be
difficult to solve analytically. In fact this particular system is very
difficult to solve analytically, so we will use a numerical method instead.

There are a number of numerical methods, and the solvers we will use in Python
and R cleverly choose the most appropriate for the problem at hand. In general
these methods use the principle that the derivative denotes the rate of
instantaneous change. Thus for a differential equation $\frac{dy}{dt} = f(t,y)$,
consider the function $y$ as a discrete sequence of points
$\{y_0, y_1, y_2, y_3, \dots\}$ on $\{t_0, t_0 + h, t_0 + 2h, t_0 + 3h, \dots\}$
then

\begin{equation}
y_{n+1} = h \times f(t_0 + nh, y_n).
\end{equation}

This sequence approaches the true solution $y$ as $h \rightarrow 0$.
Thus numerical methods, including the Runge-Kutta methods and the Euler method,
step through this sequence $\{y_n\}$, choosing appropriate values of $h$ and
employing other methods of error reduction.


\section{Solving with Python}\label{sec:solving-with-python}



\section{Solving with R}\label{sec:solving-with-R}
\section{Research}\label{sec:research}
