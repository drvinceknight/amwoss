\chapter[Game Theory]{Game Theory}

% Introduction
\chapterinitial{M}{ost} of the time when modelling certain situations two
approaches are valid: to make assumptions about the overall behaviour or to make
assumptions about the detailed behaviour. The later can be thought of as
measuring emergent behaviour. One tool used to do this is the study of
interactive decision making: game theory.

\section{Problem}\label{sec:game_theory_problem}

Consider a city council. Two electric taxi companies, company A and company B,
are going to move
in to the city and the city wants to ensure that the customers are best served
by this new duopoly. The two taxi firms will be deciding how many vehicles to
deploy: one, two or three. The city wants to encourage them to both use three as
this ensures the highest level of availability to the population.

Some exploratory data analysis gives the following insights:

\begin{itemize}
    \item If both companies use the same number of taxis then they make the same
        profit which will go down slightly as the number of taxis goes up.
    \item If one company uses more taxis than the other then they make more
        profit.
\end{itemize}

The expected profits for the companies are given in
Table~\ref{tbl:profit-of-taxi-companies}.

\newcolumntype{P}[1]{>{\centering\arraybackslash}p{#1}}
\begin{table}[!hbtp]
    \begin{center}
    \begin{tabular}{cl|P{0.08\textwidth}P{0.08\textwidth}P{0.08\textwidth}}
                                                                  &    & \multicolumn{3}{c}{Company B}\\
                                                                  &    & 1               & 2                 & 3\\
        \toprule
        \multirow{3}{*}{\rotatebox[origin=c]{90}{Company A}} & 1  & 1               & \(\frac{1}{2}\)   & \(\frac{1}{3}\)\\[4.5mm]
                                                                  & 2  & \(\frac{3}{2}\) & \(\frac{19}{20}\) & \(\frac{1}{2}\)\\[4.5mm]
                                                                  & 3  & \(\frac{5}{3}\) & \(\frac{4}{5}\)   & \(\frac{17}{20}\)\\[4.5mm]
    \end{tabular}
    \qquad
    \begin{tabular}{cl|P{0.08\textwidth}P{0.08\textwidth}P{0.08\textwidth}}
                                                                  &    &
                                                                  \multicolumn{3}{c}{Company B}\\
                                                                  &    & 1               & 2                 & 3\\
        \toprule
        \multirow{3}{*}{\rotatebox[origin=c]{90}{Company A}} & 1  & 1 & \(\frac{3}{2}\)   & \(\frac{5}{3}\)\\[4.5mm]
                                                                  & 2  & \(\frac{1}{2}\) & \(\frac{19}{20}\) & \(\frac{4}{5}\)\\[4.5mm]
                                                                  & 3  & \(\frac{1}{3}\) & \(\frac{1}{2}\)   & \(\frac{17}{20}\)\\[4.5mm]
    \end{tabular}
    \end{center}
    \caption{Profits (in GBP per hour) of each Taxi company based on
    the choice of vehicle number by all companies. The first table shows the profits for
    company A. The second table shows the profits for company B.}
    \label{tbl:profit-of-taxi-companies}
\end{table}

Given these expected profits, the council wants to understand what is likely to
happen and potentially give a financial incentive to each company to ensure
their behaviour is in the population's interest.
This would take the form of a fixed increase to the companies' profits,
\(\epsilon\), to be found, if they put on three taxis, shown in
Table~\ref{tbl:profit-of-taxi-companies-offset}

\begin{table}[!hbtp]
    \begin{center}
    \begin{tabular}{cl|P{0.08\textwidth}P{0.08\textwidth}P{0.08\textwidth}}
                                                                  &    & \multicolumn{3}{c}{Company B}\\
                                                                  &    & 1               & 2                 & 3\\
        \toprule
        \multirow{3}{*}{\rotatebox[origin=c]{90}{Company A}} & 1  & 1               & \(\frac{1}{2}\)   & \(\frac{1}{3}\)\\[4.5mm]
                                                                  & 2  & \(\frac{3}{2}\) & \(\frac{19}{20}\) & \(\frac{1}{2}\)\\[4.5mm]
                                                                  & 3  & \(\frac{5}{3} + \epsilon\) & \(\frac{4}{5} + \epsilon\) & \(\frac{17}{20} + \epsilon\)\\[4.5mm]
    \end{tabular}
    \qquad
    \begin{tabular}{cl|P{0.08\textwidth}P{0.08\textwidth}P{0.08\textwidth}}
                                                                  &    &
                                                                  \multicolumn{3}{c}{Company B}\\
                                                                  &    & 1               & 2                 & 3\\
        \toprule
        \multirow{3}{*}{\rotatebox[origin=c]{90}{Company A}} & 1  & 1 & \(\frac{3}{2}\)   & \(\frac{5}{3} + \epsilon\)\\[4.5mm]
                                                                  & 2  & \(\frac{1}{2}\) & \(\frac{19}{20}\) & \(\frac{4}{5} + \epsilon\)\\[4.5mm]
                                                                  & 3  & \(\frac{1}{3}\) & \(\frac{1}{2}\) & \(\frac{17}{20} + \epsilon\)\\[4.5mm]
    \end{tabular}
    \end{center}
    \caption{Profits (in GBP per hour) of each Taxi company based on
    the choice of vehicle number by all companies. The first table shows the profits for
    company A. The second table shows the profits for company B. The
    council's financial incentive \(\epsilon\) is included.}
    \label{tbl:profit-of-taxi-companies-offset}
\end{table}

From Table~\ref{tbl:profit-of-taxi-companies-offset} it can be seen that if Company B
chooses to use 3 vehicles while Company A chooses to only use 2 then Company B
would get \(\frac{17}{20} + \epsilon\) and Company A would get \(\frac{1}{2}\)
profits per hour. The question is: what value of \(\epsilon\) ensures both
companies use 3 vehicles.

\section{Theory}\label{sec:game_theory_theory}

In the case of this city, the interaction can be modelled using a mathematical
object called a game, which here requires:

\begin{enumerate}
    \item A given collection of actors that make decisions (players);
    \item Options available to each player (actions);
    \item A numerical value associated to each player for every possible
        choice of action made by all the players. This is the utility or reward.
\end{enumerate}

This is called a normal form game and is formally defined by:

\begin{enumerate}
    \item A finite set of \(N\) players;
    \item Action spaces for each player: \(\{A_1, A_2, A_3, \dots, A_N\}\);
    \item Utility functions that for each player \(u_1, u_2, u_3, \dots, u_N\)
        where \(u_i:A_1\times A_2 \times A_3 \dots A_N \to \mathbb{R}\).
\end{enumerate}

When \(N=2\) the utility function is often represented by a pair of matrices (1
for each player) of with the same number of rows and columns. The rows
correspond to the actions available to the first player and the columns to the
actions available to the second player.

Given a pair of actions (a row and column) we can read the utilities to both
player by looking at the corresponding entry of the corresponding matrix.

For this example, the two matrices would be:

\[
   M =
        \begin{pmatrix}
            1     & 1 / 2   & 1 / 3 \\
            3 / 2 & 19 / 20 & 1 / 2 \\
            5 / 3 & 4 / 5   & 17 / 20\\
        \end{pmatrix}
   \qquad
   N = M ^T =
        \begin{pmatrix}
            1     & 3 / 2   & 5 / 3 \\
            1 / 2 & 19 / 20 & 4 / 5 \\
            1 / 3 & 1 / 2   & 17 / 20\\
        \end{pmatrix}
\]

A diagram of the system is shown in Figure~\ref{fig:taxi-firm-game}

\begin{figure}
\begin{center}
\includestandalone[width=.8\textwidth]{./assets/taxi-firm-game/main}
\end{center}
\caption{Diagrammatic representation of the action sets and payoff matrices for
    the game.}
\label{fig:taxi-firm-game}
\end{figure}

A strategy corresponds to a way of choosing actions, this is represented by a
probability vector. For the \(i\)th player, this vector \(v\) would be of size
\(|A_i|\) (the size of the action space) and \(v_i\) corresponds to the
probability of choosing the \(i\)th action.

Both taxis always choosing to use 2 taxis (the second row/column) would
correspond to the strategy: \((0, 1, 0)\).
If both companies use this strategy and the row player (who controls the
rows) wants to improve their outcome it is evident by inspecting the second
column that the highest number is \(19 / 20\): thus the row player has no reason
to change what they are doing.

This is called a Nash equilibrium: when both players are playing a
strategy that is the best response against the other.

An important fact is that a Nash equilibrium is guaranteed to exist.
This was actually the theoretic result for which John Nash received a noble
prize\footnote{
John Nash proved the fact that any game has a Nash equilibrium
in~\cite{nash1950equilibrium}. Discussions and proofs for particular cases
can be found in good Game Theory text books such as~\cite{maschler2013game}
}.
There are various algorithms that can be used for finding Nash equilibria, they involve
a search through the pairs of spaces of possible strategies until pairs of best
responses are found. Mathematical insight allows this do be done somewhat efficiently
using algorithms that can be thought of as modifications of the algorithms used in linear programming.
One such example is called the Lemke-Howson algorithm.
A Nash equilibrium is not necessarily guaranteed to be arrived at through
dynamic decision making. However, any stable
behaviour that does emerge will be a Nash
equilibrium, such emergent processes are the topics of evolutionary game
theory\footnote{
Evolutionary game theory was formalised in~\cite{smith1974theory} although
some of the work of Robert Axelrod is evolutionary in principle~\cite{axelrod1981evolution}
},
learning algorithms\footnote{
An excellent text on learning in games is~\cite{fudenberg1998theory}
}
and/or agent based modelling which will be covered in
Chapter~\ref{chp:agent_based_simulation}.

\section{Solving with Python}\label{sec:game_theory_solving-with-python}

The first step we will take is to write a function to create a game using the
matrix expected profits and any offset. The Nashpy~\cite{knight2018nashpy}
and numpy~\cite{harris2020array} libraries will be used for this.

\begin{pyin}
import nashpy as nash
import numpy as np


def get_game(profits, offset=0):
    """Return the game object with a given offset when 3 taxis are
    provided.

    Args:
        profits: a matrix with expected profits
        offset: a float

    Returns:
        A nashpy game object
    """
    new_profits = np.array(profits)
    new_profits[2] += offset
    return nash.Game(new_profits, new_profits.T)
\end{pyin}


This gives the game for the considered problem:

\begin{pyin}
import numpy as np

profits = np.array(
    (
        (1, 1 / 2, 1 / 3),
        (3 / 2, 19 / 20, 1 / 2),
        (5 / 3, 4 / 5, 17 / 20),
    )
)
game = get_game(profits=profits)
print(game)
\end{pyin}

which gives:

\begin{pyout}
Bi matrix game with payoff matrices:

Row player:
[[1.         0.5        0.33333333]
 [1.5        0.95       0.5       ]
 [1.66666667 0.8        0.85      ]]

Column player:
[[1.         1.5        1.66666667]
 [0.5        0.95       0.8       ]
 [0.33333333 0.5        0.85      ]]
\end{pyout}

The function \mintinline{python}{get_equilibria} which will directly compute the
equilibria:

\begin{pyin}
def get_equilibria(profits, offset=0):
    """Return the equilibria for a given offset when 3 taxis are
    provided.

    Args:
        profits: a matrix with expected profits
        offset: a float

    Returns:
        A tuple of Nash equilibria
    """
    game = get_game(profits=profits, offset=offset)
    return tuple(game.support_enumeration())
\end{pyin}

This can be used to obtain the equilibria in the game.

\begin{pyin}
equilibria = get_equilibria(profits=profits)
\end{pyin}

The equilibria are:

\begin{pyin}
for eq in equilibria:
    print(eq)
\end{pyin}

giving:

\begin{pyout}
(array([0., 1., 0.]), array([0., 1., 0.]))
(array([0., 0., 1.]), array([0., 0., 1.]))
(array([0. , 0.7, 0.3]), array([0. , 0.7, 0.3]))
\end{pyout}

There are 3 Nash equilibria: 3 possible pairs of behaviour that the 2
companies could stabilise at:

\begin{itemize}
    \item The first equilibrium \(((0, 1, 0), (0, 1, 0))\) corresponds to both
          firms always using 2 taxis;
    \item The second equilibrium \(((0, 0, 1), (0, 0, 1))\) corresponds to both
          firms always using 3 taxis;
    \item The third equilibrium \(((0, 0.7, 0.3), (0, 0.7, 0.3))\) corresponds to
          both firms using 2 taxis 70\% of the time and 3 taxis otherwise.
\end{itemize}

A good thing to note is that the two taxi companies will never only provide a
single taxi (which would be harmful to the customers).

This can be used to find the number of Nash equilibria for a given offset and
stop when there is a single equilibrium:

\begin{pyin}
offset = 0
while len(get_equilibria(profits=profits, offset=offset)) > 1:
    offset += 0.01
\end{pyin}

This gives a final offset value of:

\begin{pyin}
print(round(offset, 2))
\end{pyin}


\begin{pyout}
0.15
\end{pyout}

and now confirm that the Nash equilibrium is where both taxi firms provide
three vehicles:

\begin{pyin}
print(get_equilibria(profits=profits, offset=offset))
\end{pyin}

giving:

\begin{pyout}
((array([0., 0., 1.]), array([0., 0., 1.])),)
\end{pyout}

Therefore, in order to ensure that the maximum amount of taxis are used, the
council should offer a \(\pounds 0.15\) per hour incentive to both taxi
companies for putting on 3 taxis.

\section{Solving with R}\label{sec:game_theory_solving-with-R}

R does not have a single appropriate library for the game considered here, we
will choose to use Recon~\cite{oliveira2019recon} which has functionality for finding the
Nash equilibria for two player games when only considering pure strategies
(where the players only choose to use a single action at a time).

\begin{Rin}
library(Recon)

#' Returns the equilibria in pure strategies
#' for a given offset
#'
#' @param profits: a matrix with expected profits
#' @param offset: a float
#'
#' @return a list of equilibria
get_equilibria <- function(profits, offset = 0){
  new_profits <- rbind(
    profits[c(1, 2), ],
    profits[3, ] + offset
  )
  sim_nasheq(new_profits, t(new_profits))
}
\end{Rin}


This gives the pure Nash equilibria:

\begin{Rin}

profits <- rbind(
  c(1, 1 / 2, 1 / 3),
  c(3 / 2, 19 / 20, 1 / 2),
  c(5 / 3, 4 / 5, 17 / 20)
)
eqs <- get_equilibria(profits = profits)
print(eqs)
\end{Rin}

which gives:

\begin{Rout}
$`Equilibrium 1`
[1] "2" "2"

$`Equilibrium 2`
[1] "3" "3"

\end{Rout}

There are 2 pure Nash equilibria: 2 possible pairs of behaviour that the two
companies might converge to.

\begin{itemize}
    \item The first equilibrium \(((0, 1, 0), (0, 1, 0))\) corresponds to both
          firms always using 2 taxis;
    \item The second equilibrium \(((0, 0, 1), (0, 0, 1))\) corresponds to both
          firms always using 3 taxis.
\end{itemize}

There is in fact a third Nash equilibrium where both taxi firms use 2 taxis 70\%
of the time and 3 taxis the rest of the time but \mintinline{R}{Recon} is unable
to find Nash equilibria with mixed behaviour for games with more than two
strategies.

As discussed, the council would like to offset the cost of 3
taxis so as to encourage the taxi company to provide a better service.

This gives the number of equilibria for a given offset and stops when there is a
single equilibrium:

\begin{Rin}
offset <- 0
while (length(
  get_equilibria(profits = profits, offset = offset)
  ) > 1){
  offset <- offset + 0.01
}
\end{Rin}

This gives a final offset value of:

\begin{Rin}
print(round(offset, 2))
\end{Rin}

\begin{Rout}
[1] 0.15
\end{Rout}

now confirm that the Nash equilibrium is where both taxi firms provide
three vehicles:

\begin{Rin}
eqs <- get_equilibria(profits = profits, offset = offset)
print(eqs)
\end{Rin}

giving:

\begin{Rout}
$`Equilibrium 1`
[1] "3" "3"
\end{Rout}

Therefore, in order to ensure that the maximum amount of taxis are used, the
council should offer a \(\pounds 0.15\) per hour incentive to both taxi
companies for putting on 3 taxis.

\section{Wider context}\label{sec:game_theory_wider_context}

The definition of a normal form game here as well as the solution concept of
Nash equilibrium are common starting points for the use of game theory as a
study of emergent behaviour. Other topics include different forms of games, for
example extensive form which are represented by trees and more explicitly model
asynchronous decision making. Other solution concepts involve the specific study
of the emergence mechanisms through models based on natural evolutionary
process: Moran processes or replicator dynamics. A good text book to read on
these topics is~\cite{maschler2013game}.

John Nash whose life was portrayed in the 2001 movie ``a beautiful mind'' (which
is an adaptation of~\cite{nasar2011beautiful}) won the Noble prize
for\cite{nash1950equilibrium} in which he proved that a Nash equilibrium
always exists. However, in~\cite{nash1951non} John Nash gives an
application of Game Theory to a specific version of Poker.

Another application of the concept of Nash equilibrium
is~\cite{deo2011centralized} where the authors identify worst case
scenarios for ambulance diversion: a practice where an emergency room will
declare itself too full to accept new patients. When they are multiple emergency
units serving a same population strategic behaviour becomes relevant. The
authors of this paper identify the effect of this decentralised decision making
and also propose an approach that is socially optimal: similarly to the Taxi
problem considered in this chapter.

A specific area of a lot of research in Game theory is the study of cooperative
behaviour. \cite{axelrod1981evolution} started this work with his original
computer tournaments with more recent work involving so called Zero-Determinant
strategies which considered extortion as a mathematical
concept~\cite{press2012iterated}.
A review and systemic analysis of some of the research on behaviour, of which
game theory is a subset,
is given in~\cite{press2012iterated}.
