\chapter[Game Theory]{Game Theory}

% Introduction
% TODO Fix this section (language and missing step)
\chapterinitial{M}{ost} when modelling certain situations two approaches are
valid: to make assumptions about the overall behaviour or to make assumptions
about the detailed behaviour. The later falls is akin to measuring
emergent behaviour. One tool used to do this is the study of interactive
decision making: Game Theory.

\section{Problem}\label{sec:problem}

Consider a city council. Two electric taxi companies are going to move in to
the city and the city wants to ensure that the customers are best served by this
new duopoly. The two taxi firms will be deciding how many vehicles to deploy:
one, two or three. The city wants to encourage them to both use three as this
ensures the highest level of availability to the population.

Some exploratory data analysis gives the following insights:

\begin{itemize}
    \item If both companies use the same number of taxis then they make the same
        profit which will go down slightly as the number of taxis goes up.
    \item If one company uses more taxis than the other then they make more
        profit.
\end{itemize}

The expected profits are given in Table~\ref{tbl:profit-of-taxi-companies}.

\begin{table}[!hbtp]
    \begin{center}
    \begin{tabular}{l|ccc}
        \diagbox{Taxi numbers}{Other company taxi numbers} & 1     & 2      & 3\\
        \toprule
        1                                                  & 1     & \(\frac{1}{2}\) & \(\frac{1}{3}\)\\
        2                                                  & \(\frac{3}{2}\)   & \(\frac{19}{20}\)& \(\frac{1}{2}\)\\
        3                                                  & \(\frac{5}{3}\) & \(\frac{4}{5}\)  & \(\frac{17}{20}\)\\
    \end{tabular}
    \end{center}
    \caption{Profits (in GBP per hour) of a given company based on
    their vehicle numbers and the other companies vehicle numbers.}
    \label{tbl:profit-of-taxi-companies}
\end{table}


Given these expected profits, the council wants to understand what is likely to
happen and potentially give a financial incentive to each company to ensure
their behaviour is in the population's interest.

The mathematical tool used to find the expected behaviour is Game Theory.

\section{Theory}\label{sec:theory}

In the case of this City, the interaction can be modelled using a mathematical
object called a game which in the field of game theory is defined as follows.
There are a number of games, the ones we will consider here require:

\begin{enumerate}
    \item A given collection of actors that make decisions (players).
    \item Options available to each player (actions).
    \item A numerical value associated to each player for every possible
        choice of action made by all the players. This is the utility or reward.
\end{enumerate}

There are called normal form games and are formally defined by:

\begin{enumerate}
    \item A finite set of \(N\) players;
    \item Action spaces for each player: \(\{A_1, A_2, A_3, \dots, A_N\}\);
    \item Utility functions that for each player \(u_1, u_2, u_3, \dots, u_N\)
        where \(u_i:A_1\times A_2 \times A_3 \dots A_N \to \mathbb{R}\).
\end{enumerate}

When \(N=2\) the utility function is often represented by a pair of matrices (1
for each player) of with the same number of rows and columns. The rows
correspond to the actions available to the first player and the columns to the
actions available to the second player.

Given a pair of actions (a row and column) we can read the utilities to both
player by looking at the corresponding entry of the corresponding matrix.

A strategy corresponds to an way of choosing actions, this is represented by a
probability vector. For the \(i\)th player, this vector \(v\) would be of size
\(|A_i|\) (the size of the action space) and \(v_i\) corresponds to the
probability of choosing the \(i\)th action.

For the example of our City, the two matrices would be:

\[
   M =
        \begin{pmatrix}
            1     & 1 / 2   & 1 / 3 \\
            3 / 2 & 19 / 20 & 1 / 2 \\
            5 / 3 & 4 / 5   & 17 / 20\\
        \end{pmatrix}
   \qquad
   N = M ^T =
        \begin{pmatrix}
            1     & 3 / 2   & 5 / 3 \\
            1 / 2 & 19 / 20 & 4 / 5 \\
            1 / 3 & 1 / 2   & 17 / 20\\
        \end{pmatrix}
\]

A diagram of the system is shown in Figure~\ref{fig:taxi-firm-game}

\begin{figure}
\begin{center}
\includestandalone[width=.8\textwidth]{./assets/taxi-firm-game/main}
\end{center}
\caption{Diagrammatic representation of the action sets and payoff matrices for
    the game.}
\label{fig:taxi-firm-game}
\end{figure}

Both taxis always choosing to use 2 taxis (the second row/column) would
correspond to the strategy: \((0, 1, 0)\).
If the both companies use this strategy and the row player (who controls the
rows) wants to
improve their outcome it's evident by inspecting the second column that the
highest number is \(19 / 20\): thus the row player has no reason to change what
they are doing.

This is in fact called a Nash equilibrium: when both players are playing a
strategy that is the best response against the other.

Whilst a Nash equilibria is not necessarily a set of strategies that players
will converge towards, once they are there they have no reason to move away from
it. It is the particular concept we will use to understand the emergent
behaviour in our city.

\section{Solving with Python}\label{sec:solving-with-python}

The first step we will take is to write a function to create a game using the
matrix expected profits. We will use the \mintinline{python}{nashpy} library for
this.

\begin{pyin}
import nashpy as nash


def get_game(profits):
    """Return the game object.

    Args:
        profits: a matrix with expected profits

    Returns:
        A nashpy game object
    """
    return nash.Game(profits, profits.T)
\end{pyin}

Using this we can obtain the game for the our problem:

\begin{pyin}
import numpy as np

profits = np.array(
    (
        (1, 1 / 2, 1 / 3),
        (3 / 2, 19 / 20, 1 / 2),
        (5 / 3, 4 / 5, 17 / 20),
    )
)
game = get_game(profits=profits)
print(game)
\end{pyin}

which gives:

\begin{pyout}
Bi matrix game with payoff matrices:

Row player:
[[1.         0.5        0.33333333]
 [1.5        0.95       0.5       ]
 [1.66666667 0.8        0.85      ]]

Column player:
[[1.         1.5        1.66666667]
 [0.5        0.95       0.8       ]
 [0.33333333 0.5        0.85      ]]
\end{pyout}

We can now use this to investigate what stable behaviours might emerge:

\begin{pyin}
for eq in game.support_enumeration():
    print(eq)
\end{pyin}

which gives:

\begin{pyout}
(array([0., 1., 0.]), array([0., 1., 0.]))
(array([0., 0., 1.]), array([0., 0., 1.]))
(array([0. , 0.7, 0.3]), array([0. , 0.7, 0.3]))
\end{pyout}

We see that there are 3 Nash equilibria: 3 possible pairs of behaviour that the
two companies might converge to.

A good thing to note is that the two taxi
companies will never only provide a single taxi (which would be harmful to the
customers).

However, the Council would like to offset the cost of 3
taxis so as to encourage the taxi company to provide a better service. This
involves modifying the \mintinline{python}{get_game} function as follows:

\begin{pyin}
def get_game(profits, offset):
    """Return the game object with a given offset when 3 taxis
    are provided.

    Args:
        profits: a matrix with expected profits
        offset: a float

    Returns:
        A nashpy game object
    """
    new_profits = np.array(profits)
    new_profits[2] += offset
    return nash.Game(new_profits, new_profits.T)
\end{pyin}

we will write a function \mintinline{python}{get_equilibria} which will directly
compute the equilibria:

\begin{pyin}
def get_equilibria(profits, offset):
    """Return the equilibria for a given offset when 3 taxis
    are provided.

    Args:
        profits: a matrix with expected profits
        offset: a float

    Returns:
        A nashpy game object
    """
    game = get_game(profits=profits, offset=offset)
    return tuple(game.support_enumeration())
\end{pyin}


Using this we can obtain the number of equilibria for a given offset and stop
when there is a single equilibria:

\begin{pyin}
offset = 0
while len(get_equilibria(profits=profits, offset=offset)) > 1:
    offset += 0.01
\end{pyin}

This gives a final offset value of:

\begin{pyin}
print(round(offset, 2))
\end{pyin}


\begin{pyout}
0.15
\end{pyout}

and we can confirm that the Nash equilibria is where both taxi firms provide
three vehicles:

\begin{pyin}
print(tuple(get_equilibria(profits=profits, offset=offset)))
\end{pyin}

giving:

\begin{pyout}
((array([0., 0., 1.]), array([0., 0., 1.])),)
\end{pyout}

\section{Solving with R}\label{sec:solving-with-R}

R does not have a single appropriate library for the game considered here, we
will choose to use \mintinline{R}{Recon} which has functionality for finding the
Nash equilibria for two player games when only considering pure strategies (
where the players only choose to use a single action at a time).

\begin{Rin}
library(Recon)

#' Returns the equilibria in pure strategies
#'
#' @param profits: a matrix with expected profits
#'
#' @return a list of equilibria
get_equilibria <- function(profits){
    sim_nasheq(profits, t(profits))
}
\end{Rin}

Using this we can obtain the pure Nash equilibria:

\begin{Rin}

profits <- rbind(
        c(1, 1 / 2, 1 / 3),
        c(3 / 2, 19 / 20, 1 / 2),
        c(5 / 3, 4 / 5, 17 / 20)
    )
eqs <- get_equilibria(profits = profits)
print(eqs)
\end{Rin}

which gives:

\begin{Rout}
$`Equilibrium 1`
[1] "2" "2"

$`Equilibrium 2`
[1] "3" "3"

\end{Rout}

We see that there are 2 pure Nash equilibria: 2 possible pairs of behaviour that the
two companies might converge to.

A good thing to note is that the two taxi
companies will not only provide a single taxi (which would be harmful to the
customers).

As discussed, the Council would like to offset the cost of 3
taxis so as to encourage the taxi company to provide a better service. This
involves modifying the \mintinline{R}{get_equilibria} function as follows:


\begin{Rin}
#' Returns the equilibria in pure strategies
#' for a given offset
#'
#' @param profits: a matrix with expected profits
#' @param offset: a float
#'
#' @return a list of equilibria
get_equilibria <- function(profits, offset){
    new_profits <- rbind(
                profits[c(1, 2), ],
                profits[3, ] + offset)
    sim_nasheq(new_profits, t(new_profits))
}
\end{Rin}

Using this we can obtain the number of equilibria for a given offset and stop
when there is a single equilibria:

\begin{Rin}
offset <- 0
while (length(
            get_equilibria(profits = profits, offset = offset)
            ) > 1){
    offset <- offset + 0.01
}
\end{Rin}

This gives a final offset value of:

\begin{Rin}
print(round(offset, 2))
\end{Rin}

\begin{Rout}
[1] 0.15
\end{Rout}

and we can confirm that the Nash equilibria is where both taxi firms provide
three vehicles:

\begin{Rin}
print(get_equilibria(profits = profits, offset = offset))
\end{Rin}

giving:

\begin{Rout}
$`Equilibrium 1`
[1] "3" "3"
\end{Rout}


\section{Research}\label{sec:research}

TBA
