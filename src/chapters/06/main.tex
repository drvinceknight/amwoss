\chapter[Game Theory]{Game Theory}

% Introduction
\chapterinitial{M}{ost} when modelling certain situations two approaches are
valid: to make assumptions about the overall behaviour or to make assumptions
about the detailed behaviour. The later falls is akin to measuring
emergent behaviour. One tool used to do this is the study of interactive
decision making: Game Theory.

\section{Problem}\label{sec:problem}

Consider a city council. Two electric taxi companies are going to move in to
the city and the city wants to ensure that the customers are best served by this
new duopoly. The two taxi firms will be deciding how many vehicles to deploy:
one, two or three. The city wants to encourage them to both use three as this
ensures the highest level of availability to the population.

Some exploratory data analysis gives the following insights:

\begin{itemize}
    \item If both companies use the same number of taxis then they make the same
        profit which will go down slightly as the number of taxis goes up.
    \item If one company uses more taxis than the other then they make more
        profit.
\end{itemize}

The expected profits are given in Table~\ref{tbl:profit-of-taxi-companies}.

\begin{table}[!hbtp]
    \begin{center}
    \begin{tabular}{l|ccc}
        \diagbox{Taxi numbers}{Other company taxi numbers} & 1     & 2      & 3\\
        \toprule
        1                                                  & 1     & \(\frac{1}{2}\) & \(\frac{1}{3}\)\\
        2                                                  & \(\frac{3}{2}\)   & \(\frac{19}{20}\)& \(\frac{1}{2}\)\\
        3                                                  & \(\frac{5}{3}\) & \(\frac{4}{5}\)  & \(\frac{17}{20}\)\\
    \end{tabular}
    \end{center}
    \caption{Profits (in GBP per hour) of a given company based on
    their vehicle numbers and the other companies vehicle numbers.}
    \label{tbl:profit-of-taxi-companies}
\end{table}


Given these expected profits, the council wants to understand what is likely to
happen and potentially give a financial incentive to each company to ensure
their behaviour is in the population's interest.

The mathematical tool used to find the expected behaviour is Game Theory.

\section{Theory}\label{sec:theory}

\begin{itemize}
    \item What is a game.
    \item What is a Nash equilibrium.
\end{itemize]

\section{Solving with Python}\label{sec:solving-with-python}

The first step we will take is to write a function to create a game using the
matrix expected profits. We will use the \mintinline{python}{nashpy} library for
this.

\begin{pyin}
import nashpy as nash


def get_game(profits):
    """Return the game object

    Args:
        profits: a matrix with expected profits

    Returns:
        A nashpy game object
    """
    return nash.Game(profits, profits.T)
\end{pyin}

Using this we can obtain the game for the our problem:

\begin{pyin}
import numpy as np

profits = np.array(
    (
        (1, 1 / 2, 1 / 3),
        (3 / 2, 19 / 20, 1 / 2),
        (5 / 3, 4 / 5, 17 / 20)
    )
)
game = get_game(profits=profits)
print(game)
\end{pyin}

which gives:

\begin{pyout}
Bi matrix game with payoff matrices:

Row player:
[[1.         0.5        0.33333333]
 [1.5        0.95       0.5       ]
 [1.66666667 0.8        0.85      ]]

Column player:
[[1.         1.5        1.66666667]
 [0.5        0.95       0.8       ]
 [0.33333333 0.5        0.85      ]]
\end{pyout}

We can now use this to investigate what stable behaviours might emerge:

\begin{pyin}
for eq in game.support_enumeration():
    print(eq)
\end{pyin}

which gives:

\begin{pyout}
(array([0., 1., 0.]), array([0., 1., 0.]))
(array([0., 0., 1.]), array([0., 0., 1.]))
(array([0. , 0.7, 0.3]), array([0. , 0.7, 0.3]))
\end{pyout}

We see that there are 3 Nash equilibria: 3 possible pairs of behaviour that the
two companies might converge to.

A good thing to note is that the two taxi
companies will never onlu provide a single taxi (which would be harmful to the
customers).

However, the Council would like to offset the cost of 3
taxis so as to encourage the taxi company to provide a better service. This
involves modifying the \mintinline{python}{get_game} function as follows:

\begin{pyin}
def get_game(profits, offset):
    """Return the game object with a given offset
    when 3 taxis are provided

    Args:
        profits: a matrix with expected profits
        offset: a float

    Returns:
        A nashpy game object
    """
    new_profits = np.array(profits)
    new_profits[2] += offset
    return nash.Game(new_profits, new_profits.T)
\end{pyin}

we will write a function \mintinline{python}{get_equilibria} which will directly
compute the equilibria:

\begin{pyin}
def get_equilibria(profits, offset):
    """Return the equilibria for a given offset
    when 3 taxis are provided

    Args:
        profits: a matrix with expected profits
        offset: a float

    Returns:
        A nashpy game object
    """
    game = get_game(profits=profits, offset=offset)
    return tuple(game.support_enumeration())
\end{pyin}


Using this we can obtain the number of equilibria for a given offset and stop
when there is a single equilibria:

\begin{pyin}
offset = 0
while len(get_equilibria(profits=profits, offset=offset)) > 1:
    offset += 0.01
\end{pyin}

This gives a final offset value of:

\begin{pyin}
print(round(offset, 2))
\end{pyin}


\begin{pyout}
0.15
\end{pyout}

and we can confirm that that Nash equilibria is where both taxi firms provide
three vehicles:

\begin{pyin}
print(tuple(get_equilibria(profits=profits, offset=offset)))
\end{pyin}

giving:

\begin{pyout}
((array([0., 0., 1.]), array([0., 0., 1.])),)
\end{pyout}

\section{Solving with R}\label{sec:solving-with-R}
\section{Research}\label{sec:research}

TBA
